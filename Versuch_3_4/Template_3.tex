% Copyright © 2013 Martin Ueding <dev@martin-ueding.de>

\input{header.tex}

\usepackage{placeins}

\ihead{physik313 – Versuch 3/4}
\ifoot{Lino Lemmer}

\hypersetup{
	pdftitle={Transistor}
}

\subject{Praktikumsprotokoll}
\title{Transistor}
\subtitle{physik313 – Versuch 3}
\author{
	Lino Lemmer \footnote{\href{mailto:s6lilemm@uni-bonn.de}{s6lilemm@uni-bonn.de}}
}

%\setcounter{tocdepth}{2}

\newcommand\IB{I_\text{B}}
\newcommand\IC{I_\text{C}}
\newcommand\ID{I_\text{D}}
\newcommand\IE{I_\text{E}}
\newcommand\IS{I_\text{S}}
\newcommand\RC{R_\text{C}}
\newcommand\RD{R_\text{D}}
\newcommand\RE{R_\text{E}}
\newcommand\UBE{U_\text{BE}}
\newcommand\UB{U_\text{B}}
\newcommand\UC{U_\text{C}}
\newcommand\UCE{U_\text{CE}}
\newcommand\UE{U_\text{E}}
\newcommand\UG{U_\text{G}}
\newcommand\UGS{U_\text{GS}}
\newcommand\UDS{U_\text{DS}}

\begin{document}

\maketitle

Der \LaTeX-Quelltext zu allen Protokollen in diesem Praktikum kann auf
\ref{it:mu} eingesehen werden. Die Quellen für dieses Protokoll können auf
\ref{it:github/alles} eingesehen werden. Die \LaTeX-Datei wird aus
\ref{it:github/template} generiert.

\begin{enumerate}
	\item
		\label{it:mu}
		\url{http://martin-ueding.de/de/university/physik313/}
	\item
		\label{it:github/alles}
		\url{https://github.com/martin-ueding/physik313-3_4/}
	\item
		\label{it:github/template}
		\url{https://github.com/martin-ueding/physik313-3_4/blob/master/Template_4.tex}
\end{enumerate}

\newpage
\tableofcontents
\newpage

%%%%%%%%%%%%%%%%%%%%%%%%%%%%%%%%%%%%%%%%%%%%%%%%%%%%%%%%%%%%%%%%%%%%%%%%%%%%%%%
%                                 Einleitung                                  %
%%%%%%%%%%%%%%%%%%%%%%%%%%%%%%%%%%%%%%%%%%%%%%%%%%%%%%%%%%%%%%%%%%%%%%%%%%%%%%%

\FloatBarrier
\section{Einleitung}

In diesem Versuch betrachten wir zwei Arten von Transistoren, Bipolare und FET.

Wir vermessen die Kennlinien der beiden Arten und bestimmen daraus
Verstärkungswerte sowie Arbeitspunkte. Bei einer Emitterfolgerschaltung
bestimmen wir die Strom- und Spannungsverstärkung für beide Arten. Zuletzt
bestimmen wir den Eingangswiderstand des FET.

%%%%%%%%%%%%%%%%%%%%%%%%%%%%%%%%%%%%%%%%%%%%%%%%%%%%%%%%%%%%%%%%%%%%%%%%%%%%%%%
%                                  Aufgaben                                   %
%%%%%%%%%%%%%%%%%%%%%%%%%%%%%%%%%%%%%%%%%%%%%%%%%%%%%%%%%%%%%%%%%%%%%%%%%%%%%%%

\FloatBarrier
\section{Aufgaben}

\FloatBarrier
\subsection{Aufgabe A}

\begin{problem}
	Welche Ströme treten beim Transistor außer dem Basis-Emitter-Durchlassstrom
	und dem Emitter-Kollektor-Strom auf?
\end{problem}

Es treten noch Diffusionsströme auf, die jedoch so klein sind, dass sie
vernachlässigt werden können.

\FloatBarrier
\subsection{Aufgabe B}

\begin{problem}
	Wie sieht der Potenzialverlauf im npn-Transistor aus
	\begin{enumerate}
		\item
			ohne äußere Spannung
		\item
			bei außen angelegter Spannung?
	\end{enumerate}
\end{problem}

Der Potenzialverlauf ist in Abbildung~\ref{fig:B_Zeichnung} zu sehen. Dabei ist
der Verlauf bei angelegter Spannung durchgezogen.

\begin{figure}
	\centering
	\includegraphics{Zeichnungen/B_Zeichnung.pdf}
	\caption{%
		Potenzialverlauf im npn-Transistor
	}
	\label{fig:B_Zeichnung}
\end{figure}

\FloatBarrier
\subsection{Aufgabe C}

\begin{problem}
	Wie sehen die Ladungsträgerkonzentrationen für Löcher und Elektronen im
	npn-Transistor aus?
\end{problem}

Die Ladungsträgerkonzentration im npn-Transistor ist in
Abbildung~\ref{fig:C_Zeichnung} zu sehen.

\begin{figure}
	\centering
	\includegraphics{Zeichnungen/C_Zeichnung.pdf}
	\caption{%
		Ladungsträgerkonzentration im npn-Transistor
	}
	\label{fig:C_Zeichnung}
\end{figure}

\FloatBarrier
\subsection{Aufgabe D}

\begin{problem}
	Verifizieren Sie die Relationen zwischen $\alpha$, $\beta$ und $\gamma$.
\end{problem}

Zwischen $\alpha$ und $\beta$:

\begin{align*}
	\beta &= \dod{\IC}{\IB}\\
	&= \dod{\IC}{\del{\IE-\IC}}\\
	&= \frac{\dif\IC}{\dif\IE-\dif\IC}\\
	&= \frac{\od\IC\IE}{\od\IE\IE-\od\IC\IE}\\
	&=\frac{\alpha}{1-\alpha}
	\intertext{Zwischen $\beta$ und $\gamma$:}
	\beta &= \dod{\IC}{\IB}\\
	&=\dod{\del{\IE-\IB}}\IB\\
	&= \frac{\dif\IE-\dif\IB}{\dif\IB}\
	&= \gamma-1
\end{align*}

\FloatBarrier
\subsection{Aufgabe E}
\label{ssec:Aufgabe_E}

\begin{problem}
	Welchen Ausgangsspannunsbereich ($U_\text{out min}, U_\text{out max}$)
	(Aussteuerbereich) hat die Schaltung in Abbildung~\ref{fig:3_4-5}?
	Vernachlässigen Sie hier $U_\text{CE sat}$.
\end{problem}

\begin{figure}[htbp]
	\centering
	\includegraphics[width=.4\textwidth]{Anleitung/3_4-5.png}
	\caption{%
		\cite[Abbildung~3/4.5]{physik313-Anleitung}
	}
	\label{fig:3_4-5}
\end{figure}

Wir vernachlässigen $U_\text{CE sat}$, so dass der Transistor auch bei beliebig
kleinen Spannungen $\UCE$ noch funktioniert. Die Ausgangsspannung, die gemessen
wird, ist die Spannung zwischen Abgriff und Masse. Diese ist die
Kollektorspannung $\UC$. Bei einem gegebenen Kollektorstrom $\IC$ kann die
Ausgangsspannung auch berechnet werden durch:
\[
	U_\text{out} = U_0 - \IC\RC
\]

Die Frage ist nun also, welche Kollektorströme in dieser Schaltung fließen
können.

In der Anleitung steht, dass der Aussteuerbereich durch folgende
Eckpunkte begrenzt wird:

\begin{quote}
	Die niedrigste Ausgangsspannung wird beim größtmöglichen Transistorstrom
	$\IC$ und beim kleinstmöglichen Spannungsabfall am Transsistor $\UCE$
	erreicht, und umgekehrt. \cite[§3/4.3.1]{physik313-Anleitung}
\end{quote}

Der größte Kollektorstrom, der fließen kann, wird durch $\RC$ und $\RE$
begrenzt, wenn der Transistor eine unendliche Leitfähigkeit erreicht. Somit
ist:
\[
	{\IC}_\text{max} = \frac{U_0}{\RC + \RE}
\]

Dadurch fällt eine minimale Spannung ab, nämlich:
\[
	{U_\text{out}}_\text{min} = U_0 \cdot \del{1 - \frac{\RC}{\RC + \RE}}
\]

Wenn wenn der Basisstrom $\IB$ minimal wird, weil durch das Signal $\UB$
unter \SI{0.6}{\volt} fällt, wird der Widerstand des Transistors sehr groß.
Wenn dieser dabei deutlich größer als $\RC$ wird, wird die rechte Seite der
Schaltung zu einem unbelasteten Spannungsteiler, die Spannung $U_0$ fällt
komplett über dem Transistor ab. Dadurch fällt keine Spannung über $\RC$ ab,
die Ausgangsspannung ist die volle Eingangsspannung. Also:
\[
	{U_\text{out}}_\text{max} = U_0
\]

Somit ist der Aussteuerbereich:
\[
	\intcc{U_0; U_0 \cdot \del{1 - \frac{\RC}{\RC + \RE}}}
\]

\FloatBarrier
\subsection{Aufgabe F}

\begin{problem}
	Welche Form hat die \emph{Eingangskennlinie} eines Transistors in
	Emitterschaltung ($\IB$ als Funktion von $\UBE$)?
\end{problem}

Die Eingangskennlinie ist die Kennlinie einer Diode. Sie ist in
Abbildung~\ref{fig:F_Zeichnung} zu sehen.

\begin{figure}[htbp]
	\centering
	\includegraphics{Zeichnungen/F_Zeichnung.pdf}
	\caption{%
		Einganskennlinie eines Transistors in Emitterschaltung
	}
	\label{fig:F_Zeichnung}
\end{figure}

\FloatBarrier
\subsection{Aufgabe G}

\begin{problem}
	Leiten Sie \eqref{eq:3_4-7} her!
\end{problem}

Die zitierte Gleichung ist:
\begin{equation}
	\label{eq:3_4-7}
	\UB = U_0\frac{R_2}{R_1 + R_2} - \IB\frac{R_1 R_2}{R_1 + R_2}
\end{equation}

$\UB$ und $\IB$ sind in Abbildung~\ref{fig:3_4-5} die Spannung, die über $R_2$
abfällt, bzw. der Strom der durch $R_2$ fließt. $U_1$ fällt über $R_1$ ab und
der Strom $I_1$ fließt durch $R_2$ hindurch.

\begin{align*}
	U_0 &= U_1 + \UB\\
		&= R_1I_1 + \UB\\
	 &= R_1 (I_2 + \IB) + \UB\\
	 &= R_1 \IB + R_1\frac{\UB}{R_2} + \UB\\
	 &= \UB \frac{R_1+R_2}{R_2} + R_1 \IB
	\intertext{hieraus folgt sofort}
	\UB &= U_0\frac{R_2}{R_1 + R_2} - \IB\frac{R_1 R_2}{R_1 + R_2}
\end{align*}

\FloatBarrier
\subsection{Aufgabe H}

\begin{problem}
	Was passiert, wenn man den Spannungsteiler zu niederohmig macht?
\end{problem}

Wenn der Spannungsteiler zu niederohmig ist, können große Ströme durch ihn
fließen. Wenn die Betriebsspannung als ideale Spannungsquelle angenommen wird,
ändert dies nichts an der Betriebsspannung. Jedoch wird so Basis und Emitter
mit einer zu hohen Admittanz verbunden, das Eingangssignal wird geschwächt. Im
extremen Fall liegt am Transistor gar kein Signal mehr an, die Verstärkung
funktioniert nicht mehr.

\FloatBarrier
\subsection{Aufgabe I}

\begin{problem}
	Wie sieht die entsprechende Kennlinie beim bipolaren Transistor aus?
	Welcher Spannung entspricht dort $U_\text{thr}$?
\end{problem}

Die „entsprechende Kennlinie“ ist der Graph von $\ID$ gegen $\UGS$. Bei einem
bipolaren Transistor ist dies $\IC$ gegen $\UBE$.

Die Kennlinie sieht genauso aus, wie die einer normalen Diode, da Basis und
Emitter eine normale Diode sind. Der wichtige Unterschied ist, dass die Kurve
eines FETs quadratisch ist, während die Kurve einer normalen Diode exponentiell
ist. Die Kennlinie ist in Abbildung~\ref{fig:beuth-bild-16-9} gezeigt.

\begin{figure}[htbp]
	\centering
	\includegraphics[width=.5\textwidth]{beuth-bild-16-9.jpg}
	\caption{%
		\cite[Bild~16.9]{beuth/elementare_elektronik}
	}
	\label{fig:beuth-bild-16-9}
\end{figure}

Die Spannung $U_\text{thr}$ entspricht hier der Spannung, die die Diode
braucht, bis sie leitend wird, also $\SI{.6}\volt$.

\FloatBarrier
\subsection{Aufgabe J}

\begin{problem}
	Was ändert sich, wenn man $\IS$ anstelle von $\ID$ aufträgt?
\end{problem}

Da $I_\text{G}\approx0$, ändert sich nichts signifikant.

\FloatBarrier
\subsection{Aufgabe K}

\begin{problem}
	Zeigen Sie, dass genauer gilt:
	\begin{equation}
		\label{eq:3_4-11}
		v = \frac{\gamma \RE}{r_\text{BE} + \gamma \RE},
	\end{equation}
	wobei der differentielle Widerstand der Emitter-Basis-Diode $r_\text{BE} =
	\dif  \UBE / \dif  \IB$ ist.
\end{problem}

Aus
\begin{align*}
	\UB &= \UBE + \UE
	\intertext{folgt}
	v &= \dod\UE\UB\\
	&= \frac{\dif\UE}{\dif\UBE+\dif\UE}\\
	&= \frac{\dif\IE\RE}{\dif\UBE+\dif\IE\RE}\\
	&= \frac{\od\IE\IB\RE}{\od\UBE\IB+\od\IE\IB\RE}\\
	&= \frac{\gamma\RE}{r_\text{BE}+\gamma\RE}
\end{align*}

\FloatBarrier
\subsection{Aufgabe L}

\begin{problem}
	Welchen Zweck könnte der Kollektorwiderstand $\RC$ beim Emitterfolger
	haben? Hinweis: Am Ausgang könnte eine niederohmige Last angeschlossen
	sein.
\end{problem}

Angenommen, laut Hinweis, dass der die Last, $\RE$ niederohmig ist. Der
Transistor wird bei großen Basisströmen auch niederohmig. Wenn dann kein
Schutzwiderstand $\RC$ vorgeschaltet ist, fließt ein hoher Strom durch die
Last.

\FloatBarrier
\subsection{Aufgabe M}

\begin{problem}
	Beweisen Sie \eqref{eq:3_4-12}.
\end{problem}

Die zitierte Gleichung ist:
\begin{equation}
	\label{eq:3_4-12}
	\frac{r_\text{out}}{r_\text{in}}
	= \frac{\gamma \RE}{r_\text{BE} + \gamma \RE}
	\approx \frac 1\gamma
\end{equation}

Aus $r_\text{in}=\frac\UB\IB$ und $r_\text{out}=\frac\UE\IE$ folgt:
\begin{align*}
	\frac{r_\text{out}}{r_\text{in}} &= \dod\UE\IE \dod\IB\UB\\
	&= \frac{\gamma\RE}{r_\text{BE}+\gamma\RE}\frac1\gamma\\
	&= \frac{\RE}{r_\text{BE}+\gamma\RE}\\
	&\approx \frac 1\gamma
\end{align*}

\FloatBarrier
\subsection{Aufgabe N}

\begin{problem}
	Wie groß ist der Eingangswiderstand des unbelasteten Emitterfolgers?
\end{problem}

\begin{align*}
	R_\text{in} &= \dod\UB\IB\\
	&= \frac {\dif\UE+\dif\UBE}{\dif\IB}\\
	&= r_\text{BE} + \dod \IE\IB \RE\\
	&= r_\text{BE} + \gamma\RE
\end{align*}

%%%%%%%%%%%%%%%%%%%%%%%%%%%%%%%%%%%%%%%%%%%%%%%%%%%%%%%%%%%%%%%%%%%%%%%%%%%%%%%
%                   Durchführung: Transistoreigenschaften                    %
%%%%%%%%%%%%%%%%%%%%%%%%%%%%%%%%%%%%%%%%%%%%%%%%%%%%%%%%%%%%%%%%%%%%%%%%%%%%%%%

\FloatBarrier
\section{Durchführung}

\FloatBarrier
\subsection{Kennlinien und Arbeitspunkt}

\subsubsection{Kennlinienschreiber}

Um direkt mehrere Kennlinien auf dem Oszilloskop darstellen zu können, müssen
wir schnell hintereinander verschiedene Basisströme $\IB$ auf den Transistor
gegeben werden. Diese Umschaltung der Basisströme übernimmt der
Kennlinienschreiber für uns. Das Schaltbild des Kennlinienschreibers ist in
Abbildung~\ref{fig:3-1} dargestellt.

\begin{figure}[htbp]
	\centering
	\includegraphics[width=\textwidth]{Anleitung/3-1.png}
	\caption{
		Kennlinienschreiber \cite[Abbildung~3.1]{physik313-Anleitung}
	}
	\label{fig:3-1}
\end{figure}

Am linken Triggereingang wird ein
Rechtecksignal zugeführt. Dieses wird durch den Kondensator differenziert. Die
positiven Pulse, die durch den Transistor verstärkt werden, werden im Zählwerk
gezählt. Dieses gibt die aktuelle Anzahl als vier Binärstellen durch vier
Widerstandsketten. Die einzelnen Widerstände unterscheiden sich um einen Faktor
2, so dass der Ausgangsstrom in 16 Stufen erhöht wird.

\subsubsection{Inbetriebnahme des Kennlinienschreibers}

Am rechten Ende ist ein Steckplatz für einen Transistor, der mit den oben
beschriebene den Basisströmen versorgt wird. Der zweite Ausgang des
Funktionsgenerators, der ein Dreiecksignal liefert, gibt eine kontinuierlich
ändernde Betriebsspannung. Mit dem Oszilloskop nehmen wir die
Kollektor-Emitter-Spannung $\UCE$ ab und geben es in Kanal~2 rein. Auf Kanal~1
wird die Betriebsspannung vom Funktionsgenerator angelegt. Dieser Aufbau ist in
Abbildung~\ref{fig:3-2} gezeigt.

\begin{figure}[htbp]
	\centering
	\includegraphics[width=\textwidth]{Anleitung/3-2.png}
	\caption{
		\cite[Abbildung~3.2]{physik313-Anleitung}
	}
	\label{fig:3-2}
\end{figure}

Damit auf dem Oszilloskop die Spannungsdifferenz angezeigt wird, invertieren
wir Kanal~2 und addieren ihn auf Kanal~1. Für die Kennlinien stellen wir
außerdem den XY-Betrieb ein.

Wir schließen den Transistor vom \emph{Schaltbrett~1} (siehe
Abbildung~\ref{fig:3-4}) an den Kennlinienschreiber an.

\begin{figure}[htbp]
	\centering
	\includegraphics[width=\textwidth]{Anleitung/3-4.png}
	\caption{
		\emph{Schaltbrett~1} \cite[Abbildung~3.4]{physik313-Anleitung}
	}
	\label{fig:3-4}
\end{figure}

\FloatBarrier
\subsubsection{Bipolarer Transistor}

\paragraph{Aufnahme der Kennlinen}

Wir lassen die Schaltung von unserem Assistenten überprüfen. Danach schalten
wir das Netzgerät ein und erhalten nach Justierung das Kennlinienfeld, siehe
Abbildung~\ref{fig:3}.

\begin{figure}
	\centering
	\begin{minipage}{.45\linewidth}
		\includegraphics[width=\linewidth]{Oszi_Hand/3-03.jpg}
	\end{minipage}
	\hfill
	\begin{minipage}{.45\linewidth}
		\includegraphics[width=\linewidth]{Oszi_Foto/3-03.jpg}
	\end{minipage}
	\caption{%
		Kennlinienfeld, $\IC$ gegen $\UCE$. Verstärkung
		\SI{1}{\volt\per\division}, XY-Modus. Durch die Umwandlung des Stromes
		$\IC$ in eine Spannung mit $\RC = \SI{500}\ohm$ entspricht ein
		\si{\division} auf der $\IC$-Achse einem Strom von
		\SI{2}{\milli\ampere}.
	}
	\label{fig:3}
\end{figure}

Den Kollektorstrom $\IC$ haben wir indirekt gemessen, in dem wir die
Spannungsdifferenz vor und nach dem Widerstand $\RC$ mit dem Oszilloskop
sichtbar gemacht haben. Da $\RC = \SI{500}\ohm$ ist, können wir die Spannung so
in einen Strom umwandeln.

\paragraph{Bestimmung von $\beta$}

Aus dem Graph in Abbildung~\ref{fig:3} lesen den Abstand
zwischen zwei Kennlinien, $\dif\IC$, ab. Wir wählen Kennline 5 und
6 von unten, und erhalten eine Differenz von \SI{<< dIC >>}{\ampere}. Mit
$\dif\IB = \SI{<< dIB >>}{\ampere}$ erhalten wir:
\[
	\beta = \dod\IC\IB = \num{<< beta >>}
\]

\paragraph{Arbeitspunkt}

Die Arbeitsgerade ist in der Anleitung gegeben als:
\cite[Formel~3/4.6]{physik313-Anleitung}
\begin{equation}
	\label{eq:3_4-6}
	\IC = \frac{U_0 - \UCE}{\RC + \RE}
\end{equation}

Mit $\RC = \RE = \SI{390}\ohm$ und einer Betriebspannung von $U_0 =
\SI{15}\volt$ können wir eine Gerade einzeichnen. Die Spurpunkte sind:
\[
	\del{\UCE = \SI0\volt; \IC = \frac{U_0}{\RC + \RE} = \SI{19.2}{\milli\ampere}}
	\eqnsep
	\del{\UCE = U_0 = \SI{15}\volt; \IC = \SI0\ampere}
\]

Aus diesem Grund sollten wir bereits beim Abzeichnen des Kennlinienfeldes
darauf achten, dass $\UCE$ einen Bereich bis $\SI{15}\volt$ und $\IC$ einen
Bereich bis $\SI{20}{\milli\ampere}$ umfasst.

Diese Gerade ist in Abbildung \ref{fig:3} schon eingezeichnet.

In der Anleitung war gegeben, dass die Kennlinien $(\SI{12}\volt -
\SI{.7}\volt)/(4 \cdot \SI{470}{\kilo\ohm})$, also \SI{6.01}{\micro\ampere}
auseinander liegen. Es soll die Spannung $\UCE$ für den Arbeitspunkt $\IB =
\SI{60}{\micro\ampere}$ abgeschätzt werden, in dem die Linien im Kennlinienfeld
verlängert werden. Die gewünschte Linie ist dann die zehnte Linie. Da wir
davon ausgehen können, dass im Sättigungsbereich der Kollektorstrom
näherungsweise konstant bleibt, lesen wir die Spannung
$\UCE = \SI{7.4 \pm 0.1}{\volt}$ und den Strom $\IC = \SI{9.8 \pm
0.2}{\milli\ampere}$ ab.

\subsubsection{FET}

\paragraph{Aufnahme der Kennlinien}

Wir benutzen den gleichen Aufbau wie in der vorherigen Aufgabe. Den bipolaren
Transistor ersetzen wir durch einen FET. Da dieser mit einer Spannung gesteuert
wird, müssen wir den Basisstrom $\IB$ erst noch in eine Basis-Emitter-Spannung
$\UBE$ umwandeln. Bei FETs heißt diese jedoch Gate-Source-Spannung, $\UGS$. Wir
schließen also einen Widerstand $R$ zwischen Gate und Source an. Dazu benutzen
wir ein \SI{470}{\kilo\ohm}-Potentiometer. Wir justieren das Potentiometer so,
dass möglichst viele Linien auf dem Oszilloskop zu sehen sind. Leider haben wir
es bei der Durchführung versäumt, den eingestellten Wert von $R$ zu messen.
Daher können wir jetzt nur einen Wert zwischen \SIrange{0}{470}{\kilo\ohm}
annehmen und damit rechnen.

Den Spannungsabstand $\dif\UGS$ erhalten wir durch $R \cdot
\SI{6}{\micro\ampere}$, da, wie schon im vorherigen Abschnitt berechnet, die
Stromunterschiede $\dif\IB$ den Wert \SI{6}{\micro\ampere} haben. Über einem
Widerstand $R$ fällt dann die Spannung $U = IR$ ab. Dies ist natürlich nicht
ganz exakt, da der Widerstand nicht unendlich groß ist und so alle Ströme etwas
verändert.

\begin{figure}
	\centering
	\label{fig:4}
	\begin{minipage}{0.45\linewidth}
		\includegraphics[width=\linewidth]{Oszi_Hand/3-04.jpg}
	\end{minipage}
	\hfill
	\begin{minipage}{.45\linewidth}
		\includegraphics[width=\linewidth]{Oszi_Foto/3-04.jpg}
	\end{minipage}
	\caption{%
		Kennlinienfeld, $\RD \ID$ gegen $\UDS$. Verstärkung $\SI
		1{\volt\per\division}$, XY-Modus.
	}
	\label{fig:4}
\end{figure}

Das Kennlinienfeld des FET ist in Abbildung~\ref{fig:4} zu sehen.

\paragraph{Bestimmung der Schwellenspannung und Transkonduktanz}
\label{par:Schwellenspannung}

Um die Schwellenspannung und die Transkonduktanz bestimmen zu können muss die
Abhängigkeit des Drainstromes $\ID$ von $\UGS$ bei konstantem $\UDS$ untersucht
werden.

Als feste $\UDS$ wähle ich \SI 2\volt. Ich lese aus dem Plot die Werte für
$\ID$ für die verschiedenen Linien ab, siehe Tabelle~\ref{tab:trans_raw}.

\begin{table}[htbp]
	\centering
	\begin{tabular}{SS}
		{Liniennummer $n$} & {$\ID / \si\division$} \\
		\hline
		%< for n, id in trans_raw: >%
		<< n >> & << id >> \\
		%< endfor >%
	\end{tabular}
	\caption{%
		Abgelesene Daten aus Abbildung~\ref{fig:4}
	}
	\label{tab:trans_raw}
\end{table}

Mit einem angenommenen Widerstand von \SI{<< R >>}{\ohm} kann ich für die Linie
$n$ die Gatespannung $\UG$ wie folgt ausrechnen:
\[
	\UG = R n \cdot \SI{6}{\micro\ampere}
\]

Die umgerechneten Daten sind in Tabelle~\ref{tab:trans}.

\begin{table}[htbp]
	\centering
	\begin{tabular}{SS}
		{$\UG / \si\volt$} & {$\ID / \si\ampere$} \\
		\hline
		%< for ug, id in trans: >%
		<< ug >> & << id >> \\
		%< endfor >%
	\end{tabular}
	\caption{%
		Berechnete Daten aus Tabelle~\ref{tab:trans_raw}
	}
	\label{tab:trans}
\end{table}

Die Daten plotte ich einmal $\ID$ gegen $\UG$ in Abbildung~\ref{fig:ID_UG} und
einmal als $\sqrt\ID$ gegen $\UG$ in Abbildung~\ref{fig:ID_UG_sqrt}.

\begin{figure}[htbp]
	\centering
	\includegraphics[width=\textwidth]{ID_UG.pdf}
	\caption{%
		Drainstrom gegen Gatespannung bei konstanter Drainspannung
	}
	\label{fig:ID_UG}
\end{figure}

\begin{figure}[htbp]
	\centering
	\includegraphics[width=\textwidth]{ID_UG_sqrt.pdf}
	\caption{%
		Drainstrom gegen Gatespannung bei konstanter Drainspannung
	}
	\label{fig:ID_UG_sqrt}
\end{figure}

Dies fitte ich mit $\ID = k (\UG - U_\text{thr})^2$ und erhalte $k = \SI{<< k
>>}{\per\ohm}$ und $U_\text{thr} = \SI{<< Uthr >>}{\volt}$. Die Fits sind
schon in den Abbildungen~\ref{fig:ID_UG} und \ref{fig:ID_UG_sqrt} eingetragen.

%< if Uthr | float < 0: >%
Es sollte keine negative Thresholdspannung herauskommen. In den weiteren
Aufgaben hatten wir eine Spannung von \SI{1.4}{\volt} gemessen. Auch wenn wir
den Widerstand $R$ nicht bestimmt hatten, sollte mindestens eine positive
Spannung $U_\text{thr}$ herauskommen.
%< endif >%

\FloatBarrier
\subsection{Emitterfolger}

\subsubsection{Aufbau}

Auf dem \emph{Schaltbrett~1} (Abbildung~\ref{fig:3-4}) bauen wir einen
Emitterfolger (Kollektorschaltung) auf. Eine solche ist, mit Kapazitäten
erweitert, in Abbildung~\ref{fig:beuth-bild-16-21} dargestellt.

\begin{figure}[htbp]
	\centering
	\includegraphics[width=.6\textwidth]{beuth-bild-16-21.jpg}
	\caption{%
		Verstärkerstufe in Kollektorschaltung (Emitterfolgerstufe)
		\cite[Bild~16.21]{beuth/elementare_elektronik}
	}
	\label{fig:beuth-bild-16-21}
\end{figure}

Das Netzgerät mit $U_0 = \SI{15}\volt$ wird an das Schaltbrett angeschlossen.
Wir setzen $\RC = \RE = \SI{390}\ohm$ ein, wie in der Anleitung beschrieben.
Als Signal stellen wir ein Sinussignal mit $\dif\UBE = \SI{.5}{\voltss}$ und
einer Frequenz $f = \SI{500}\hertz$ ein. Dazu kommt ein Offset von
\SI{2}{\volt}.

Das Oszilloskop wird mit Kanal~1 an das Signal des Funktionsgenerators
angeschlossen. Kanal~2 kommt an den Emitter.

Wir vergleichen die Eingangs- und Ausgangsspannung, $\UB$ bzw. $\UE$. auf dem
Oszilloskop. Die Signale sind in Abbildung~\ref{fig:7} zu sehen.

\begin{figure}
	\centering
	\begin{minipage}{.45\linewidth}
		\includegraphics[width=\linewidth]{Oszi_Hand/3-07.jpg}
	\end{minipage}
	\hfill
	\begin{minipage}{.45\linewidth}
		\includegraphics[width=\linewidth]{Oszi_Foto/3-07.jpg}
	\end{minipage}
	\caption{%
		Eingangs- und Ausgangssignal beim Emitterfolger. Verstärkung auf beiden
	Kanälen identisch.
	}
	\label{fig:7}
\end{figure}

Es ist deutlich zu erkennen, dass keine Phasenverschiebung stattgefunden hat.
Dies ist klar, da die Ausgangsspannung $\UE$ mit der Eingangsspannung $\UB$
über die Beziehung $\UB = \UE - \UBE = \UB - \SI {0.6}{\volt}$ zusammenhängt.

\subsubsection{Spannungsverstärkung}

Wir messen die Spannungsverstärkung $\dif\UE/\dif\UB$. Dazu lesen wir vom
Oszilloskop die Spitze-Spitze-Spannung von beiden Kanälen ab und erhalten:
\[
	\dif\UE = \SI{<< dU_E >>}\voltss
	\eqnsep
	\dif\UB = \SI{<< dU_B >>}\voltss
\]

Daraus folgt die Spannungsverstärkung $\dif\UE/\dif\UB = \num{<<
spannungsverstaerkung >>}$.

\subsubsection{Aussteuergrenzen}

Am Signalgenerator wird nun die Amplitude variiert um die untere und obere
Aussteuergrenze zu ermitteln. Fällt das Eingangssignal dabei unter
\SI{0.3}{\volt} wird das Ausgangssignal abgeschnitten.

Durch weiteres Aufdrehen bis zur maximalen Amplitude des Generators konnten
wir keine Signalverzerrung feststellen. Der Theorie nach hätten wir ab einer
bestimmten Eingangssignalstärke eine einsetzende Gleichrichtung beobachten
können müssen.

Die untere Grenze konnten wir durch Vergrößerung des Offsets erhöhen.

Nach unseren Vorüberlegungen in \ref{ssec:Aufgabe_E} sollte die untere Grenze
bei $U/2$ liegen. Dies ist eindeutig nicht der Fall.

\subsubsection{Arbeitspunkteinstellung}

Wir ersetzen nun den DC-Offset am Funktionsgenerator mit einer
Basisvorspannung, die wir mit dem Spannungsteiler erzeugen. In der Anleitung
steht: „Fügen Sie dazu einen geeigneten Kondensator in Serie hinter dem
\SI{10}{\kilo\ohm}-Widerstand ein.“ Auf \emph{Schaltbrett~1} gibt es jedoch
drei solche Widerstände, wobei einer davon ein Potentiometer ist. Der
Kondensator soll hinter den linken Widerstand. Dies habe ich in blau in
Abbildung~\ref{fig:3-4_Arbeitspunkt} dargestellt. In rot und grün sind
einfache Kabelbrücken.

\begin{figure}[htbp]
	\centering
	\includegraphics[width=\textwidth]{Anleitung/3-4_Arbeitspunkt.png}
	\caption{
		\emph{Schaltbrett~2} mit Beschaltung. Originalbild ist in
		Abbildung~\ref{fig:3-4} und aus
		\cite[Abbildung~3.4]{physik313-Anleitung}.
	}
	\label{fig:3-4_Arbeitspunkt}
\end{figure}

Der Kondensator hat nicht direkt etwas mit der Einstellung der Basisvorspannung
zu tun. Diese ist mit dem roten Kontakt schon eingestellt. Der Kondensator
dient dazu, den Funktionsgenerator von der Gleichspannung zu entkoppeln, sowie
eine eventuell verbleibende Offsetspannung vom Signal zu trennen.

Wenn der Kondensator zu klein ist, ist seine Impedanz zu hoch, das
Eingangssignal wird zu stark geschwächt. Es findet eine Phasenverschiebung
statt, die größer wird je kleiner der Kondensator ist.

Bei einem \SI{10}{\micro\farad}-Kondensator war weder eine Dämpfung, noch eine
Phasenverschiebung erkennbar. Beim \SI{100}{\nano\farad}-Kondensator waren
beide zwar sichtbar, aber noch sehr klein.

Wir verändern den Spannungsteiler um zu schauen, wann der Aussteuerungsbereich
der Schaltung am größten ist.

Beim \SI{100}{\nano\farad}-Kondensator war der Bereich am größten bei einer
Basisspannung von \SI{1.2}{\volt}. Dies ist ein kleiner Wert im Gegensatz zu
anliegenden Gesamtspannung von \SI{15}{\volt}. Daraus kann man folgern je
größer $\RC$, desto größer der Aussteuerbereich.

\FloatBarrier
\subsection{FET}

Auf dem \emph{Schaltbrett~2} (Abbildung~\ref{fig:3-5}) bauen wir eine
Emitterfolgerschaltung (Kollektorschaltung) mit dem Bipolartransistor auf.
Dabei setzen wir $\RC = \SI0\ohm$ und den Emitterwiderstand im Bereich
\SIrange{1}{2}{\kilo\ohm}. Dazu setzen wir einen \SI{1.8}{\kilo\ohm} hinter den
schon vorhandenen \SI{100}{\ohm} Widerstand.

Die Schaltung wird mit \SI{10}{\volt} Gleichspannung versorgt. Als
Eingangssignal in die Basis des Transistors geben wir ein Sinussignal mit
\SI1{\kilo\hertz} und \SI{2}{\voltss}. Außerdem stellen wir einen Offset von
etwa \SI5{\volt} ein.

\begin{figure}[htbp]
	\centering
	\includegraphics[width=\textwidth]{Anleitung/3-5.png}
	\caption{
		\emph{Schaltbrett~2} \cite[Abbildung~3.5]{physik313-Anleitung}
	}
	\label{fig:3-5}
\end{figure}

Wir beobachten das Eingangssignal sowie das Signal, das aus dem Emitter kommt,
mit den beiden Kanälen des Oszilloskops. Dazu stellen wir auf beiden Kanälen
die gleiche Verstärkung und den gleichen $y$-Offset ein. Die Kanäle werden auf
DC gekoppelt, damit auch die Basisvorspannung zu sehen ist.

Der Offset wird variiert, dabei beobachten wir, wie die Ausgangsspannung
ungefähr \SI{.6}{\volt} unterhalb der Eingangsspannung liegt. Außerdem erhalten
wir nur ein Ausgangssignal, wenn die Spannung oberhalb dieser Grenzspannung
liegt. Siehe Abbildung~\ref{fig:Folger}.

\begin{figure}[htbp]
	\centering
	\begin{minipage}{.3\linewidth}
		\includegraphics[width=\linewidth]{Oszi_Foto/3-08.jpg}
	\end{minipage}
	\hfill
	\begin{minipage}{.3\linewidth}
		\includegraphics[width=\linewidth]{Oszi_Foto/3-10.jpg}
	\end{minipage}
	\hfill
	\begin{minipage}{.3\linewidth}
		\includegraphics[width=\linewidth]{Oszi_Foto/3-11.jpg}
	\end{minipage}
	\caption{%
		Variation des Offsets. Verstärkung \SI{1}{\volt\per\division}.
	}
	\label{fig:Folger}
\end{figure}

Als nächstes bauen wir eine vergleichbare Schaltung, diesmal allerdings mit dem
FET. Die Schaltung heißt nicht mehr Emitterfolgerschaltung sondern
Sourcefolgerschaltung, da beim FET die englischen Begriffe benutzt werden. Wir
überprüfen die Funktion der Schaltung und verändern den DC-Offset und
beobachten, dass die Schaltung im wesentlichen gleich funktioniert.

Wir bestimmen $U_\text{thr}$ zu \SI{1.4}{\volt}. Im Vergleich zu der Messung
der Schwellenspannung, die wir auf Seite~\pageref{par:Schwellenspannung}
bestimmt haben, kommt hier etwas völlig anderes heraus. Dies liegt jedoch
daran, dass wir den Widerstand $R$ nicht bestimmt haben, und dass der dort
erhaltene Wert von \SI{<< Uthr >>}{\volt} negativ ist.

\subsubsection{Eingangswiderstand}

In dieser Versuchsaufgabe wollen wir die Eingangswiderstände der Transistoren
als unbelastete Emitterfolger abschätzen.

Den Sinusgenerator nehmen wir aus der Buchse~B1 heraus. Mit einer Drahtbrücke
koppeln wir den \SI1{\micro\farad}-Kondensator ein. Das Oszilloskop hat einen
Eingangswiderstand von \SI1{\mega\ohm}. Dadurch sollte die Zeitkonstante für
die Entladung des Kondensators sein:
\[
	\tau = RC = \SI1{\second}
\]

Die Halbwertszeit $T_{1/2}$ ist um $1/\ln(2)$ kleiner, also $\SI{.69}\second$.
Wir beobachten eine Halbwertszeit von \SI{0.64}{\second}.

Der bipolare Transistor wird wieder als Emitterfolger eingesetzt. Buchse~1 wird
mit einem kurzen BNC Kabel an die Basis angeschlossen. Mit dem Oszilloskop
beobachten wir die Spannung am Emitter, also am Ausgang des Verstärkers. Der
Transistor wird am Kollektor mit Strom versorgt, am Emitter haben wir keinen
Widerstand eingesetzt. Die Entladung hat nur über den \SI{1}{\mega\ohm}
Eingangswiderstand am Oszilloskop stattgefunden. In der Anleitung steht, dass
man einen Widerstand im Bereich \SIrange{1}{2}{\kilo\ohm} als $\RE$ benutzen
soll. Dies haben wir versucht, jedoch war die Entladung dann so schnell, dass
sie nicht mehr sinnvoll zu beobachten war.

Wir beobachten die neue Abfallzeit. Diese ist anders, da der Kondensator jetzt
über einen anderen Widerstand entlädt, den Transistor. Wir beobachten eine
Abfallzeit von \SI{76}{\second}.

Aus diesem Verhältnis bestimmen wir die Stromverstärkung. Diese ist definiert
als:
\[
	B := \frac\IC\IB
	\eqnsep
	\beta := \frac{\dif\IC}{\dif\IB}
\]

Um die Halbwertszeit zu halbieren, muss ein doppelt so hoher Strom fließen. Da
jedoch die Halbwertszeit beim Bipolartransistor größer geworden ist, floss
weniger Strom. Dies bedeutet, dass der Strom nicht verstärkt worden ist,
sondern abgeschwächt. Dies wäre mit einem $\RE$ im Bereich \SI{1}{\kilo\ohm}
wahrscheinlich anders gewesen, jedoch haben wir dort die Halbwertszeit nicht
mehr sinnvoll messen können.

Beim FET sind die Zeiten so groß wie beim Oszilloskop alleine, der Strom wurde
also nicht wirklich verändert. Dies ist gerade beim FET sehr seltsam.

Bei kleineren Kapazitäten wurde die Halbwertszeit deutlich geringer. Bei den
kleinsten Kapazitäten soweit, dass sie mit dem Auge als instantan wahrgenommen
worden sind. Wir haben die Zeitbasis am Oszilloskop so eingestellt, dass wir
die Entladekurve beobachten konnte. Da keine Triggerung eingestellt worden ist,
war die Entladung an einer beliebigen Stelle auf dem Schirm und hat nur
aufgeblitzt. Wir haben versucht abzuschätzen, wie lange die Halbwertszeit ist,
jedoch sind unsere Messungen nur Größenordnungen. Siehe
Tabelle~\ref{tab:Halbwertszeiten}.

Zuletzt bauen wir mit dem FET einen Sourcefolger auf, der ebenfalls am Drain
mit Spannung versorgt wird. Die Entladung findet nur über das Oszilloskop
statt, auch hier haben wir keinen Emitterwiderstand benutzt. Die Entladezeiten
waren alle extrem kurz. Zu erwarten wäre, laut Tutor, dass sich die Spannung
selbst bei der kleinsten Kapazität nicht verändert. Bei unserem FET
unterscheiden sich die Entladezeiten allerdings nicht von den Entladezeiten des
Oszilloskops alleine.

\begin{table}[htbp]
	\centering
	\begin{tabular}{S|SSS}
		{Kapazität / \si{\farad}} & {Oszilloskop alleine / \si\second} &
		{Bipoltransistor / \si\second} & {FET / \si\second} \\
		\hline
		%< for c, o, b, f in halbwertszeiten: >%
		<< c >> & << o >> & << b >> & << f >> \\
		%< endfor >%
	\end{tabular}
	\caption{
		Halbwertszeiten bei den verschiedenen Transistoren
	}
	\label{tab:Halbwertszeiten}
\end{table}

Aus den Halbwertszeiten können wir die Widerstände errechnen:
\[
	R = \frac{T_{1/2}}{C \ln(2)}
\]

Die errechneten Widerstände sind in Tabelle~\ref{tab:Eingangswiderstaende}. Der
\SI{1}{\mega\ohm}-Widerstand des Oszilloskop ist in der entsprechenden Spalte
zu erkennen. Bei der letzten Kapazität ist unser Wert allerdings weniger genau.
Der Bipolartransistor hat einen \emph{höheren} Widerstand als das Oszilloskop
alleine. Dies ist zu erwarten, da wir nur über das Oszilloskop entladen haben.
Um den eigentlichen Widerstand zu erhalten, muss man den gemessenen Wert vom
Widerstand des Oszilloskops abziehen. Allerdings bleibt auch dann immer noch
ein Widerstand in der Größenordnung \SI{1e8}{\ohm} übrig.

Der gemessene Widerstand beim FET ist jedoch nur wenig größer als der des
Oszilloskops. Dies bedeutet, dass der FET keinen besonders hohen
Eingangswiderstand hat. In der Anleitung sind Widerstände von
\SIrange{e12}{e15}{\ohm} angegeben. \cite[§3/4.4]{physik313-Anleitung}. In
anderen Gruppen war die Entladezeit deutlich höher. Der von uns benutzte FET
hatte wahrscheinlich einen defekt.

\begin{table}[htbp]
	\centering
	\begin{tabular}{S|SSS}
		{Kapazität / \si{\farad}} & {Oszilloskop alleine / \si\ohm} &
		{Bipoltransistor / \si\ohm} & {FET / \si\ohm} \\
		\hline
		%< for c, o, b, f in widerstaende: >%
		<< "%.3g" % c >> & << "%.3g" % o >> & << "%.3g" % b >> & << "%.3g" % f >> \\
		%< endfor >%
	\end{tabular}
	\caption{
		Eingangswiderstände bei den verschiedenen Transistoren
	}
	\label{tab:Eingangswiderstaende}
\end{table}

Die Entladung mit der Kapazität vom BNC Kabel war noch deutlich schneller als
bei der \SI{100}{\pico\farad}-Kapazität. In Simons Gruppe war zu sehen, dass
die Kapazität des Kabels wohl in der gleichen Größenordnung wie
\SI{100}{\pico\farad} ist. Sie hatten das Kabel auf Seite des Kondensators
abgesteckt, entladen und wieder angesteckt. Die Spannung auf dem Oszilloskop
war noch ungefähr halb so groß.

In der Tabelle~1.1 aus der Anleitung ist eine spezifische Kapazität von
\SI{100}{\pico\farad\per\meter} für RG-58C/U-Kabel gegeben. Das kurze Kabel war
jedoch nur ungefähr \SI{15}{\centi\meter} lang. Wahrscheinlich haben Stecker
und das Gate selbst noch eine Kapazität, so dass die Summe in der Größenordnung
von \SI{100}{\pico\farad} liegt.

%%%%%%%%%%%%%%%%%%%%%%%%%%%%%%%%%%%%%%%%%%%%%%%%%%%%%%%%%%%%%%%%%%%%%%%%%%%%%%%
%                                  Literatur                                  %
%%%%%%%%%%%%%%%%%%%%%%%%%%%%%%%%%%%%%%%%%%%%%%%%%%%%%%%%%%%%%%%%%%%%%%%%%%%%%%%

\FloatBarrier
\IfFileExists{\bibliographyfile}{
	\bibliography{\bibliographyfile}
}{}

\end{document}

% vim: spell spelllang=de tw=79
