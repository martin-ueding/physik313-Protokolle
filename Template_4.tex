% Copyright © 2013 Martin Ueding <dev@martin-ueding.de>

\input{header.tex}

\usepackage{placeins}

\ihead{physik313 – Versuch 3/4}
\ifoot{Lino Lemmer}

\hypersetup{
	pdftitle={Transistorverstärker}
}

\subject{Praktikumsprotokoll}
\title{Transistorverstärker}
\subtitle{physik313 – Versuch 4}
\author{
	Lino Lemmer \footnote{\href{mailto:s6lilemm@uni-bonn.de}{s6lilemm@uni-bonn.de}}
}

%\setcounter{tocdepth}{2}

\newcommand\IB{I_\text{B}}
\newcommand\IC{I_\text{C}}
\newcommand\ID{I_\text{D}}
\newcommand\IE{I_\text{E}}
\newcommand\IS{I_\text{S}}
\newcommand\RC{R_\text{C}}
\newcommand\RE{R_\text{E}}
\newcommand\UBE{U_\text{BE}}
\newcommand\UB{U_\text{B}}
\newcommand\UC{U_\text{C}}
\newcommand\UCE{U_\text{CE}}
\newcommand\UE{U_\text{E}}
\newcommand\UGS{U_\text{GS}}

\usepackage{tocloft}

\newlistof{todo}{lotd}{TODO Liste}

\newcommand{\FIXME}[1]{\printTODO{FIXME: #1}}
\newcommand{\TODO}[1]{\printTODO{TODO: #1}}
\newcommand{\XXX}[1]{\printTODO{XXX: #1}}
\newcommand{\FRAGE}[1]{\printTODO{Rückfrage: #1}}

\newcommand{\printTODO}[1]{
	\par%
	\textcolor{OrangeRed}{\textsf{#1}}%
	\par%
	\refstepcounter{todo}
	\addcontentsline{lotd}{todo}{#1}
}


\begin{document}

\maketitle

Der \LaTeX-Quelltext zu allen Protokollen in diesem Praktikum kann auf
\ref{it:mu} eingesehen werden. Die Quellen für dieses Protokoll können auf
\ref{it:github/alles} eingesehen werden. Die \LaTeX-Datei wird aus
\ref{it:github/template} generiert.

\begin{enumerate}
	\item
		\label{it:mu}
		\url{http://martin-ueding.de/de/university/physik313/}
	\item
		\label{it:github/alles}
		\url{https://github.com/martin-ueding/physik313-3_4/}
	\item
		\label{it:github/template}
		\url{https://github.com/martin-ueding/physik313-3_4/blob/master/Template_4.tex}
\end{enumerate}

%\newpage
\tableofcontents

\listoftodo
\newpage

%%%%%%%%%%%%%%%%%%%%%%%%%%%%%%%%%%%%%%%%%%%%%%%%%%%%%%%%%%%%%%%%%%%%%%%%%%%%%%%
%                                 Einleitung                                  %
%%%%%%%%%%%%%%%%%%%%%%%%%%%%%%%%%%%%%%%%%%%%%%%%%%%%%%%%%%%%%%%%%%%%%%%%%%%%%%%

\FloatBarrier
\section{Einleitung}

\TODO{
	Einleitung für den zweiten Tag schreiben.
}

%%%%%%%%%%%%%%%%%%%%%%%%%%%%%%%%%%%%%%%%%%%%%%%%%%%%%%%%%%%%%%%%%%%%%%%%%%%%%%%
%                                  Aufgaben                                   %
%%%%%%%%%%%%%%%%%%%%%%%%%%%%%%%%%%%%%%%%%%%%%%%%%%%%%%%%%%%%%%%%%%%%%%%%%%%%%%%

\FloatBarrier
\section{Aufgaben}

\FloatBarrier
\subsection{Aufgabe O}

\begin{problem}
	Zeigen Sie, dass genauer gilt:
	\begin{equation}
		\label{eq:3_4-14}
		v = - \frac{\beta \RC}{r_\text{BE} + \gamma \RE}
	\end{equation}
\end{problem}

\begin{align*}
	v &= \dod \UC\UB\\
	  &= -\frac{\dif\IC\RC}{\dif\UBE + \dif\UE}\\
	  &= -\frac{\dif\IC\RC}{\dif\UBE + \dif\IE\RE}\\
	  &= -\frac{\od\IC\IB\RC}{\od\UBE\IB+\od\IE\IB\RE}\\
	  &= -\frac{\beta\RC}{r_\text{BE}+\gamma\RE}
\end{align*}

\FloatBarrier
\subsection{Aufgabe P}

\begin{problem}
	Beweisen Sie \eqref{eq:3_4-17}.
\end{problem}

Die zitierte Gleichung ist:
\begin{equation}
	\label{eq:3_4-17}
	\frac{\dif  v} v
	= \frac{\dif  v_0}{v_0} \frac{1}{k v_0 + 1}
	= \frac{\dif  v_0}{v_0} \frac{v}{v_0}
\end{equation}

Aus 

\begin{align*}
	\frac 1v &= \frac 1{v_0} + k
	\intertext{folgt:}
	v &= \frac {v_0}{1+kv_0}
	\intertext{Daraus ergibt sich}
	\dod {v}{v_0} &= \frac {1+kv_0-kv_0}{\del{1+kv_0}^2}\\
	&= \frac 1{\del{1+kv_0}^2}\\
	&= \frac {v_0}{1+kv_0} \frac 1{v_0\del{1+kv_0}}\\
	&= \frac v{v_0} \frac 1{1+kv_0}
	\intertext{Hieraus erhält man die gesuchte Gleichung}
	\frac {\dif v}v &= \frac {\dif v_0}{v_0} \frac 1{1+kv_0} = \frac {\dif
v_0}{v_0} \frac v{v_0}
\end{align*}

\FloatBarrier
\subsection{Aufgabe Q}

\begin{problem}
	Erklären sie, wieso die Kapazität $C_\text{CB}$ Einfluss auf die
	Verstärkung hat.
\end{problem}

Bei einer Verstärkung $\beta$ wird durch den Miller-Effekt wird die anfänglich
kleine Kapazität $C_\text{BC}$ um den Faktor $1+\abs\beta$ vergrößert. Die
hierdurch vergrößerte Kapazität wirkt nun als Tiefpass. Die durch diesen
Tiefpass unterdrückten Frequenzen werden zwar weiterhin verstärkt, aber die
Unterdrückung wirkt stärker.

\FloatBarrier
\subsection{Aufgabe R}

\begin{problem}
	Erklären Sie die Funktionsweise der Schaltung in
	Abbildung~\ref{fig:3_4-15}! Wie groß ist die Spannungsänderung im Punkt P
	bei einer Stromänderung $\dif  \IE(T2)$ und welche
	Transistorgröße bestimmt diesen Wert?
\end{problem}

\begin{figure}[htbp]
	\centering
	\includegraphics[width=.6\textwidth]{Anleitung/3_4-15.png}
	\caption{%
		\cite[Abbildung~3/4.15]{physik313-Anleitung}
	}
	\label{fig:3_4-15}
\end{figure}

An $T_2$ liegt eine konstante Spannung an. Durch die Diodeneigenschaft des
Transistors wird dadurch die Spannung am Punkt $P$ auf ca. $\SI 2{\volt} -
\SI{0.6}{\volt} = \SI{1.4}{\volt}$ fixiert, da eine kleine Änderung des Stroms
$\dif\IE$ nahezu keine Änderung der Basis-Emitter-Spannung $\UBE$ in $T_2$
bewirkt.

\TODO{Aufgabe R: welche Transistoreigneschaft?}

\FloatBarrier
\subsection{Aufgabe S}

\begin{problem}
	Leiten Sie \eqref{eq:3_4-18} her. Hinweise: Da die Gegenkopplung bei
	Betrachtung im Frequenzraum auf der Addition von Sinusschwingungen beruht
	und wie in diesem Kapitel die Phasen ignoriert haben, ist hier
	$v(f_\text{grenz gk}) = 2v(f=0)$ statt korrekterweise $\sqrt 2 v (f = 0)$.
\end{problem}

Die zitierte Gleichung ist:
\begin{equation}
	\label{eq:3_4-18}
	f_\text{grenz gk} = f_\text{grenz} \frac{v_0}{v(f=0)}
\end{equation}

\TODO{Aufgabe S fertigstellen}

\FloatBarrier
\subsection{Aufgabe T}

\begin{problem}
	Erläutern Sie die Wirkungsweise der Art der Stabilisierung des
	Basispotentials durch den Widerstand $R$ in Abbildung~\ref{fig:3_4-16}.
	Überlegen Sie dazu, was passiert, wenn das Basispotential aus irgend einem
	(äußeren) Grund „wegläuft“!
\end{problem}

\begin{figure}[htbp]
	\centering
	\includegraphics[width=.6\textwidth]{Anleitung/3_4-16.png}
	\caption{%
		\cite[Abbildung~3/4.16]{physik313-Anleitung}
	}
	\label{fig:3_4-16}
\end{figure}

Da kein Emitterwiderstand in dieser Schaltung vorhanden ist, trennt Basis und
Emitter nur \SI{.6}{\volt}. Der Widerstand zwischen Basis und Masse ist recht
gering ab dieser Spannung, es kann also ein hoher Basisstrom fließen. Dieser
wird jedoch durch $R_\text{in}$ aus $U_\text{in}$ begrenzt.

Durch den Widerstand $R$ wird eine Basisvorspannung angelegt, die jedoch vom
Kollektorstrom abhängt. Im Arbeitspunkt stellt sich folgende Basisspannung ein:
\[
	\UB = U_\text{in} - R_\text{in} I_\text{B in}
	+ U_0 - \RC \IC - R I_R
\]

Der Strom $I_\text{B in}$ sollte unabhängig vom Strom $I_R$ sein und somit:
\[
	I_\text{B in} = \frac{U_\text{in} - \SI{.6}\volt}{R_\text{in}}
\]

Eingesetzt bleibt von $\UB$ nur noch folgendes übrig:
\[
	\UB = \SI{.6}\volt + U_0 - \RC \IC - R I_R
\]

Das liegt daran, dass $U_\text{in}$ einen so hohen Basistrom erzeugen würde,
bis der Widerstand $R_\text{in}$ die meiste Spannung wieder verbrannt hat. Die
Basisvorspannung muss also über $R$ laufen. Da über $R_\text{in}$ nur das
kleine Wechselstromsignal kommen soll, passt dies auch.

Je nach Bemessung von $R$ und $\RC$ wird sich ein gewisser Arbeitspunkt mit
einer gewissen Basisspannung $\UB$ einstellen. Darauf hin wird ein Basisstrom
$\IB$ fließen, der jedoch auf einen recht kleinen Widerstand trifft, solange
$\UB \gtrsim \SI{.6}\volt$ gilt. Würde $\UB$ konstant bleiben, käme es hier zu
einem beliebigen Anwachsen von $\IB$.

Jedoch bedeutet ein höherer Basisstrom, der durch $R$ fließen muss, dass mehr
Spannung über diesem abfällt und somit die Basisvorspannung gesenkt wird.
Außerdem schaltet der Transistor einen höhreren Kollektorstrom $\IC$ frei,
womit auch wieder mehr Strom über $\RC$ abfällt, das Basispotential rückt
wieder näher zur Masse, es fließt weniger Basisstrom.

Auf diese Weise wird ein unbeschränktes Anwachsen von $\IB$, und somit die
Zerstörung des Transistors, verhindert werden.

\FloatBarrier
\subsection{Aufgabe U}

\begin{problem}
	An welche Stelle der Schaltung [Abbildung~\ref{fig:3_4-16}] würden Sie den
	Kondensator setzen [, um die Rückkopplung wechselspannungsmäßig
	aufzuheben]? (Tip: Man kann einen Widerstand teilen!)
\end{problem}

Um die Rückkopplung für Wechselspannungen aufzuheben, muss die Leitfähigkeit
von $R$ für Wechselspannungen herabgesetzt werden. Es soll also ein
Wechselstromwiderstand zu $R$ hinzukommen. Falls wir eine Induktivität zur
Verfügung hätten, konnte man diese direkt hinter $R$ schalten. Dies würde für
ausreichend hohe Frequenzen die Leitfähigkeit klein halten.

Mit einem Kondensator, der seriell geschaltet wird, wird nur die
Gleichspannung gesperrt. Mit einem Kondensator, der parallel geschaltet wird,
wird der Wechselstromwiderstand gesenkt.

\TODO{Aufgabe U fertigstellen}

%%%%%%%%%%%%%%%%%%%%%%%%%%%%%%%%%%%%%%%%%%%%%%%%%%%%%%%%%%%%%%%%%%%%%%%%%%%%%%%
%                    Durchführung: Transistorverstärker                     %
%%%%%%%%%%%%%%%%%%%%%%%%%%%%%%%%%%%%%%%%%%%%%%%%%%%%%%%%%%%%%%%%%%%%%%%%%%%%%%%

\FloatBarrier
\section{Durchführung}

\subsection{Fortsetung Emitterfolger}

\subsubsection{Spannungsverstärkung des Emitterfolgers}

\fehlt

\subsubsection{Emitterfolger als Impedanzwandler}

\fehlt

\subsection{Invertierender Transistorverstärker (Emitterschaltung)}

\subsubsection{Phasenbeziehung zwischen Ein- und Ausgang}

\fehlt

\subsubsection{Spannungsverstärkung des inverteierenden Verstärkers}

\fehlt

\subsubsection{Bestimmung des Transistoreingangswiderstands}

\fehlt

\subsection{Wechselstrommäßige Aufhebung der Gegenkopplung}

\fehlt

\subsection{Frequenzverhalten und Kaskodenschaltung}

\fehlt

\subsection{Verstärker mit Spannungsgegenkopplung}

\fehlt

\begin{figure}[htbp]
	\centering
	\includegraphics[width=.6\textwidth]{Anleitung/4-1.png}
	\caption{
		\cite[Abbildung~4.1]{physik313-Anleitung}
	}
	\label{fig:4-1}
\end{figure}

%%%%%%%%%%%%%%%%%%%%%%%%%%%%%%%%%%%%%%%%%%%%%%%%%%%%%%%%%%%%%%%%%%%%%%%%%%%%%%%
%                                  Literatur                                  %
%%%%%%%%%%%%%%%%%%%%%%%%%%%%%%%%%%%%%%%%%%%%%%%%%%%%%%%%%%%%%%%%%%%%%%%%%%%%%%%

\FloatBarrier
\IfFileExists{\bibliographyfile}{
	\bibliography{\bibliographyfile}
}{}

\end{document}

% vim: spell spelllang=de tw=79
