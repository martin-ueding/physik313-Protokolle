% Copyright © 2013 Martin Ueding <dev@martin-ueding.de>

\input{header.tex}

\usepackage{placeins}

\ihead{physik313 – Versuch 3/4}
\ifoot{Lino Lemmer}

\hypersetup{
	pdftitle={Transistorverstärker}
}

\subject{Praktikumsprotokoll}
\title{Transistorverstärker}
\subtitle{physik313 – Versuch 4}
\author{
	Lino Lemmer \footnote{\href{mailto:s6lilemm@uni-bonn.de}{s6lilemm@uni-bonn.de}}
}

%\setcounter{tocdepth}{2}

\newcommand\IB{I_\text{B}}
\newcommand\IC{I_\text{C}}
\newcommand\ID{I_\text{D}}
\newcommand\IE{I_\text{E}}
\newcommand\IS{I_\text{S}}
\newcommand\RC{R_\text{C}}
\newcommand\RE{R_\text{E}}
\newcommand\UBE{U_\text{BE}}
\newcommand\UB{U_\text{B}}
\newcommand\UC{U_\text{C}}
\newcommand\UCE{U_\text{CE}}
\newcommand\UE{U_\text{E}}
\newcommand\UGS{U_\text{GS}}

\usepackage{tocloft}

\newlistof{todo}{lotd}{TODO Liste}

\newcommand{\FIXME}[1]{\printTODO{FIXME: #1}}
\newcommand{\TODO}[1]{\printTODO{TODO: #1}}
\newcommand{\XXX}[1]{\printTODO{XXX: #1}}
\newcommand{\FRAGE}[1]{\printTODO{Rückfrage: #1}}

\newcommand{\printTODO}[1]{
	\par%
	\textcolor{OrangeRed}{\textsf{#1}}%
	\par%
	\refstepcounter{todo}
	\addcontentsline{lotd}{todo}{#1}
}


\begin{document}

\maketitle

Der \LaTeX-Quelltext zu allen Protokollen in diesem Praktikum kann auf
\ref{it:mu} eingesehen werden. Die Quellen für dieses Protokoll können auf
\ref{it:github/alles} eingesehen werden. Die \LaTeX-Datei wird aus
\ref{it:github/template} generiert.

\begin{enumerate}
	\item
		\label{it:mu}
		\url{http://martin-ueding.de/de/university/physik313/}
	\item
		\label{it:github/alles}
		\url{https://github.com/martin-ueding/physik313-3_4/}
	\item
		\label{it:github/template}
		\url{https://github.com/martin-ueding/physik313-3_4/blob/master/Template_4.tex}
\end{enumerate}

%\newpage
\tableofcontents

\listoftodo
\newpage

%%%%%%%%%%%%%%%%%%%%%%%%%%%%%%%%%%%%%%%%%%%%%%%%%%%%%%%%%%%%%%%%%%%%%%%%%%%%%%%
%                                 Einleitung                                  %
%%%%%%%%%%%%%%%%%%%%%%%%%%%%%%%%%%%%%%%%%%%%%%%%%%%%%%%%%%%%%%%%%%%%%%%%%%%%%%%

\FloatBarrier
\section{Einleitung}

\TODO{
	Einleitung für den zweiten Tag schreiben.
}

%%%%%%%%%%%%%%%%%%%%%%%%%%%%%%%%%%%%%%%%%%%%%%%%%%%%%%%%%%%%%%%%%%%%%%%%%%%%%%%
%                                  Aufgaben                                   %
%%%%%%%%%%%%%%%%%%%%%%%%%%%%%%%%%%%%%%%%%%%%%%%%%%%%%%%%%%%%%%%%%%%%%%%%%%%%%%%

\FloatBarrier
\section{Aufgaben}

\FloatBarrier
\subsection{Aufgabe O}

\begin{problem}
	Zeigen Sie, dass genauer gilt:
	\begin{equation}
		\label{eq:3_4-14}
		v = - \frac{\beta \RC}{r_\text{BE} + \gamma \RE}
	\end{equation}
\end{problem}

\begin{align*}
	v &= \dod \UC\UB\\
	  &= -\frac{\dif\IC\RC}{\dif\UBE + \dif\UE}\\
	  &= -\frac{\dif\IC\RC}{\dif\UBE + \dif\IE\RE}\\
	  &= -\frac{\od\IC\IB\RC}{\od\UBE\IB+\od\IE\IB\RE}\\
	  &= -\frac{\beta\RC}{r_\text{BE}+\gamma\RE}
\end{align*}

\FloatBarrier
\subsection{Aufgabe P}

\begin{problem}
	Beweisen Sie \eqref{eq:3_4-17}.
\end{problem}

Die zitierte Gleichung ist:
\begin{equation}
	\label{eq:3_4-17}
	\frac{\dif  v} v
	= \frac{\dif  v_0}{v_0} \frac{1}{k v_0 + 1}
	= \frac{\dif  v_0}{v_0} \frac{v}{v_0}
\end{equation}

Aus 

\begin{align*}
	\frac 1v &= \frac 1{v_0} + k
	\intertext{folgt:}
	v &= \frac {v_0}{1+kv_0}
	\intertext{Daraus ergibt sich}
	\dod {v}{v_0} &= \frac {1+kv_0-kv_0}{\del{1+kv_0}^2}\\
	&= \frac 1{\del{1+kv_0}^2}\\
	&= \frac {v_0}{1+kv_0} \frac 1{v_0\del{1+kv_0}}\\
	&= \frac v{v_0} \frac 1{1+kv_0}
	\intertext{Hieraus erhält man die gesuchte Gleichung}
	\frac {\dif v}v &= \frac {\dif v_0}{v_0} \frac 1{1+kv_0} = \frac {\dif
v_0}{v_0} \frac v{v_0}
\end{align*}

\FloatBarrier
\subsection{Aufgabe Q}

\begin{problem}
	Erklären sie, wieso die Kapazität $C_\text{CB}$ Einfluss auf die
	Verstärkung hat.
\end{problem}

Bei einer Verstärkung $\beta$ wird durch den Miller-Effekt wird die anfänglich
kleine Kapazität $C_\text{CB}$ um den Faktor $1+\abs\beta$ vergrößert. Die
hierdurch vergrößerte Kapazität wirkt nun als Tiefpass. Die durch diesen
Tiefpass unterdrückten Frequenzen werden zwar weiterhin verstärkt, aber die
Unterdrückung wirkt stärker.

\FloatBarrier
\subsection{Aufgabe R}

\begin{problem}
	Erklären Sie die Funktionsweise der Schaltung in
	Abbildung~\ref{fig:3_4-15}! Wie groß ist die Spannungsänderung im Punkt P
	bei einer Stromänderung $\dif  \IE(T2)$ und welche
	Transistorgröße bestimmt diesen Wert?
\end{problem}

\begin{figure}[htbp]
	\centering
	\includegraphics[width=.6\textwidth]{Anleitung/3_4-15.png}
	\caption{%
		\cite[Abbildung~3/4.15]{physik313-Anleitung}
	}
	\label{fig:3_4-15}
\end{figure}

An $T_2$ liegt eine konstante Spannung an. Durch die Diodeneigenschaft des
Transistors wird dadurch die Spannung am Punkt $P$ auf ca. $\SI 2{\volt} -
\SI{0.6}{\volt} = \SI{1.4}{\volt}$ fixiert, da eine kleine Änderung des Stroms
$\dif\IE$ nahezu keine Änderung der Basis-Emitter-Spannung $\UBE$ in $T_2$
bewirkt.

\TODO{Aufgabe R: welche Transistoreigneschaft?}

\FloatBarrier
\subsection{Aufgabe S}

\begin{problem}
	Leiten Sie \eqref{eq:3_4-18} her. Hinweise: Da die Gegenkopplung bei
	Betrachtung im Frequenzraum auf der Addition von Sinusschwingungen beruht
	und wir in diesem Kapitel die Phasen ignoriert haben, ist hier
	$v(f_\text{grenz gk}) = 2v(f=0)$ statt korrekterweise $\sqrt 2 v (f = 0)$.
\end{problem}

Die zitierte Gleichung ist:
\begin{equation}
	\label{eq:3_4-18}
	f_\text{grenz gk} = f_\text{grenz} \frac{v_0}{v(f=0)}
\end{equation}

Die Transitfrequenz $f_\text{T}$ bleibt, trotz der durch die Gegenkopplung
verringerten Verstärkung, die gleiche. Daher gilt
\[f_\text{grenz}v_0 = f_\text{T} = f_\text{grenz gk} v(f=0)\]
Hieraus folgt sofort die gesuchte Gleichung \eqref{eq:3_4-18}.

\FloatBarrier
\subsection{Aufgabe T}

\begin{problem}
	Erläutern Sie die Wirkungsweise der Art der Stabilisierung des
	Basispotentials durch den Widerstand $R$ in Abbildung~\ref{fig:3_4-16}.
	Überlegen Sie dazu, was passiert, wenn das Basispotential aus irgend einem
	(äußeren) Grund „wegläuft“!
\end{problem}

\begin{figure}[htbp]
	\centering
	\includegraphics[width=.6\textwidth]{Anleitung/3_4-16.png}
	\caption{%
		\cite[Abbildung~3/4.16]{physik313-Anleitung}
	}
	\label{fig:3_4-16}
\end{figure}

Da kein Emitterwiderstand in dieser Schaltung vorhanden ist, trennt Basis und
Emitter nur \SI{.6}{\volt}. Der Widerstand zwischen Basis und Masse ist recht
gering ab dieser Spannung, es kann also ein hoher Basisstrom fließen. Dieser
wird jedoch durch $R_\text{in}$ aus $U_\text{in}$ begrenzt.

Durch den Widerstand $R$ wird eine Basisvorspannung angelegt, die jedoch vom
Kollektorstrom abhängt. Im Arbeitspunkt stellt sich folgende Basisspannung ein:
\[
	\UB = U_\text{in} - R_\text{in} I_\text{B in}
	+ U_0 - \RC \IC - R I_R
\]

Der Strom $I_\text{B in}$ sollte unabhängig vom Strom $I_R$ sein und somit:
\[
	I_\text{B in} = \frac{U_\text{in} - \SI{.6}\volt}{R_\text{in}}
\]

Eingesetzt bleibt von $\UB$ nur noch folgendes übrig:
\[
	\UB = \SI{.6}\volt + U_0 - \RC \IC - R I_R
\]

Das liegt daran, dass $U_\text{in}$ einen so hohen Basistrom erzeugen würde,
bis der Widerstand $R_\text{in}$ die meiste Spannung wieder verbrannt hat. Die
Basisvorspannung muss also über $R$ laufen. Da über $R_\text{in}$ nur das
kleine Wechselstromsignal kommen soll, passt dies auch.

Je nach Bemessung von $R$ und $\RC$ wird sich ein gewisser Arbeitspunkt mit
einer gewissen Basisspannung $\UB$ einstellen. Darauf hin wird ein Basisstrom
$\IB$ fließen, der jedoch auf einen recht kleinen Widerstand trifft, solange
$\UB \gtrsim \SI{.6}\volt$ gilt. Würde $\UB$ konstant bleiben, käme es hier zu
einem beliebigen Anwachsen von $\IB$.

Jedoch bedeutet ein höherer Basisstrom, der durch $R$ fließen muss, dass mehr
Spannung über diesem abfällt und somit die Basisvorspannung gesenkt wird.
Außerdem schaltet der Transistor einen höhreren Kollektorstrom $\IC$ frei,
womit auch wieder mehr Strom über $\RC$ abfällt, das Basispotential rückt
wieder näher zur Masse, es fließt weniger Basisstrom.

Auf diese Weise wird ein unbeschränktes Anwachsen von $\IB$, und somit die
Zerstörung des Transistors, verhindert werden.

\FloatBarrier
\subsection{Aufgabe U}

\begin{problem}
	An welche Stelle der Schaltung [Abbildung~\ref{fig:3_4-16}] würden Sie den
	Kondensator setzen [, um die Rückkopplung wechselspannungsmäßig
	aufzuheben]? (Tip: Man kann einen Widerstand teilen!)
\end{problem}

Um die Rückkopplung für Wechselspannungen aufzuheben, muss die Leitfähigkeit
von $R$ für Wechselspannungen herabgesetzt werden. Dies wird ermöglicht, indem
man den Widerstand $R$ aufteilt und in die Mitte einen Kondensator zur Masse
einsetzt. Der so entstandene Tiefpass bewirkt eine Unterdrückung hoher
Frequenzen und wirkt in beide Richtungen.

%%%%%%%%%%%%%%%%%%%%%%%%%%%%%%%%%%%%%%%%%%%%%%%%%%%%%%%%%%%%%%%%%%%%%%%%%%%%%%%
%                    Durchführung: Transistorverstärker                     %
%%%%%%%%%%%%%%%%%%%%%%%%%%%%%%%%%%%%%%%%%%%%%%%%%%%%%%%%%%%%%%%%%%%%%%%%%%%%%%%

\FloatBarrier
\section{Durchführung}

\subsection{Fortsetung Emitterfolger}

\subsubsection{Spannungsverstärkung des Emitterfolgers}

Auf Schaltbrett 1 (Abbildung \ref{fig:3-4}) wird ein Emitterfolger mit
externem Offset aufgebaut. Dazu wird der Funktionsgenerator auf Sinussignal mit
$U_\text{SS} = \SI{0.5}{\volt}$ und Offset $\UB = \SI 2{\volt}$ eingestellt. An
die Stellen der waagerechten Doppelpfeile kommen einfache Kabelbrücken. Kanal~2
des Osziloskops werden mit dem unteren Anschluss verbunden.

\begin{figure}
	\centering
	\includegraphics[width=\textwidth]{Anleitung/3-4.png}
	\caption{%
		Schaltbrett 1
	}
	\label{fig:3-4}
\end{figure}

Nun wird zunächst der Emitterwiderstand $\RE$ konstant gelassen und die
Spannungsverstärkung für drei verschiedene Kollektorwiderstände $\RC$ gemessen,
danach umgekehrt.

Die Messung ist in Tabelle~\ref{tab:4-1-1}.

\begin{table}
	\centering
	\begin{tabular}{SS|SSS}
		{$\RE / \si{\ohm}$} & {$\RC / \si{\ohm}$} & {$\dif\UB / \si{\division}$}
		& {$\dif\UE / \si{\division}$} & {$\del{\dif\UE / \dif\UB}$} \\
		\hline
	\end{tabular}
	\caption{%
		Abhängigkeit der Spannungsverstärkung von $\RE$ und $\RC$
	}
\label{tab:4-1-1}
\end{table}

Die Erwartung, dass sich die Verstärkung nicht nennenswert verändert, da sie in
erster Näherung
\[v = \frac{\gamma\RE}{r_\text{BE}+\gamma\RE} \approx 1\]
unabhängig sein soll von $\RE$ und $\RC$ wurde $\messwert$.

\TODO{Was passiert bei kleinen $\RE$, großen $\RC$?}

\subsubsection{Emitterfolger als Impedanzwandler}

Nun wird ein invertierender Transistorverstärker aufgebaut, indem in
Abbildung~\ref{fig:3-4} $\RC = \SI {22}{\kilo\ohm}$ und $\RE = \SI
1{\kilo\ohm}$ gesetzt werden. Als Eingangssignal wird ein Sinussignal mit
$U_\text{SS} = \SI {0.5}{\volt}$, $\UB = \SI 1{\volt}$ und $f =
\SI{800}{\hertz}$ gewählt.

Nun wird an den Kollektor ein Lautsprecher mit $R \approx \SI{300}{\ohm}$
angeschlossen. 

\TODO{Warum ist nichts zu hören?}

Mit dem Schaltbrett~2 (Abbildung~\ref{fig:3-5} wird ein Emitterfolger zwischen
Transistorverstärker und Lautsprecher eingebaut. Dazu wird die
Gleichspannungsversorgung von Brett~1 übernommen und der Lautsprecher wird an
den Emitter der npn-Bipolar-Transistors geschlossen.

\begin{figure}
	\centering
	\includegraphics[width=\textwidth]{Anleitung/3-5.png}
	\caption{%
		Schaltbrett 2
	}
	\label{fig:3-5}
\end{figure}

\TODO{Warum ist hier ein Signal zu hören}

\subsection{Invertierender Transistorverstärker (Emitterschaltung)}

\subsubsection{Phasenbeziehung zwischen Ein- und Ausgang}

Auf Schaltbrett 1 (Abbildung~\ref{fig:3-4}) wird nun eine Emitterschaltung
aufgebaut, indem Kanal~2 des Oszilloskops an den oberen Ausgang geschlossen
wird. Für Kollektor- und Emitterwiderstand soll gelten $\RC = \RE =
\SI{390}{\ohm}$.

Das Basispotenzial wird mit dem Spannungsteiler (senkrechter Doppelpfeil
überbrückt) so eingestellt, dass $\UB = \SI {1.5}{\volt}$ ist.  Eine der
waagerechten Kabelbrücken aus dem vorherigen Versuch muss durch einen möglichst
großen Kondensator ersetzt werden, damit der Gleichspannungsanteil aus dem
Signal des Generators gefiltert wird. Letzterer soll ein
$U_\text{SS}=\SI{0.5}{\volt}$-Sinussignal ausgeben.

Die Phasenbeziehung zwischen Eingangs- und Ausgangssignal ist hierbei $\pi$.

\subsubsection{Spannungsverstärkung des inverteierenden Verstärkers}

\begin{table}
	\centering
	\begin{tabular}{SSSSS}
		{$\RE / \si\ohm$} &
		{$\RC / \si\ohm$} &
		{$\dif\UB / \si\volt$} &
		{$\dif\UE / \si\volt$} &
		{$\beta$} \\
		\hline
		%< for re, rc, dub, due, beta in table412: >%
		<< re >> & << rc >> & << dub >> & << due >> & << beta >> \\
		%< endfor >%
	\end{tabular}
\end{table}

\fehlt

\subsubsection{Bestimmung des Transistoreingangswiderstands}

\fehlt

\subsection{Wechselstrommäßige Aufhebung der Gegenkopplung}

\fehlt

\subsection{Frequenzverhalten und Kaskodenschaltung}

\begin{table}
	\centering
	\begin{tabular}{SSS|S}
		{$f / \si\hertz$} &
		{$\dif\UB / \si\volt$} &
		{$\dif\UE / \si\volt$} &
		{$|v|$} \\
		\hline
		%< for f, dub, due, v in table414normal: >%
		<< f >> & << dub >> & << due >> & << v >> \\
		%< endfor >%
	\end{tabular}
	\caption{%
		Messwerte für die Wechselspannungsverstärkung.
	}
	\label{tab:414normal}
\end{table}

\begin{table}
	\centering
	\begin{tabular}{SSS|S}
		{$f / \si\hertz$} &
		{$\dif\UB / \si\volt$} &
		{$\dif\UE / \si\volt$} &
		{$|v|$} \\
		\hline
		%< for f, dub, due, v in table414Kaskode: >%
		<< f >> & << dub >> & << due >> & << v >> \\
		%< endfor >%
	\end{tabular}
	\caption{%
		Messwerte für die Wechselspannungsverstärkung der Kaskodenschaltung
	}
	\label{tab:414Kaskode}
\end{table}

\fehlt

\subsection{Verstärker mit Spannungsgegenkopplung}

\fehlt

\begin{figure}[htbp]
	\centering
	\includegraphics[width=.6\textwidth]{Anleitung/4-1.png}
	\caption{
		\cite[Abbildung~4.1]{physik313-Anleitung}
	}
	\label{fig:4-1}
\end{figure}

%%%%%%%%%%%%%%%%%%%%%%%%%%%%%%%%%%%%%%%%%%%%%%%%%%%%%%%%%%%%%%%%%%%%%%%%%%%%%%%
%                                  Literatur                                  %
%%%%%%%%%%%%%%%%%%%%%%%%%%%%%%%%%%%%%%%%%%%%%%%%%%%%%%%%%%%%%%%%%%%%%%%%%%%%%%%

\FloatBarrier
\IfFileExists{\bibliographyfile}{
	\bibliography{\bibliographyfile}
}{}

\end{document}

% vim: spell spelllang=de tw=79
