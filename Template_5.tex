% Copyright © 2013 Martin Ueding <dev@martin-ueding.de>

\input{header.tex}

\usepackage{placeins}

\ihead{physik313 – Versuch 5}
\ifoot{Lino Lemmer}

\hypersetup{
	pdftitle={Transistor}
}

\subject{Praktikumsprotokoll}
\title{Transistor}
\subtitle{physik313 – Versuch 5}
\author{
	Lino Lemmer \footnote{\href{mailto:s6lilemm@uni-bonn.de}{s6lilemm@uni-bonn.de}}
}

%\setcounter{tocdepth}{2}

\newcommand\IB{I_\text{B}}
\newcommand\IC{I_\text{C}}
\newcommand\ID{I_\text{D}}
\newcommand\IE{I_\text{E}}
\newcommand\IS{I_\text{S}}
\newcommand\RC{R_\text{C}}
\newcommand\RD{R_\text{D}}
\newcommand\RE{R_\text{E}}
\newcommand\UBE{U_\text{BE}}
\newcommand\UB{U_\text{B}}
\newcommand\UC{U_\text{C}}
\newcommand\UCE{U_\text{CE}}
\newcommand\UE{U_\text{E}}
\newcommand\UG{U_\text{G}}
\newcommand\UGS{U_\text{GS}}
\newcommand\UDS{U_\text{DS}}

\begin{document}

\maketitle

Der \LaTeX-Quelltext zu allen Protokollen in diesem Praktikum kann auf
\ref{it:mu} eingesehen werden. Die Quellen für dieses Protokoll können auf
\ref{it:github/alles} eingesehen werden. Die \LaTeX-Datei wird aus
\ref{it:github/template} generiert.

\begin{enumerate}
	\item
		\label{it:mu}
		\url{http://martin-ueding.de/de/university/physik313/}
	\item
		\label{it:github/alles}
		\url{https://github.com/martin-ueding/physik313-3_4/}
	\item
		\label{it:github/template}
		\url{https://github.com/martin-ueding/physik313-3_4/blob/master/Template_4.tex}
\end{enumerate}

\newpage
\tableofcontents
\newpage

%%%%%%%%%%%%%%%%%%%%%%%%%%%%%%%%%%%%%%%%%%%%%%%%%%%%%%%%%%%%%%%%%%%%%%%%%%%%%%%
%                                 Einleitung                                  %
%%%%%%%%%%%%%%%%%%%%%%%%%%%%%%%%%%%%%%%%%%%%%%%%%%%%%%%%%%%%%%%%%%%%%%%%%%%%%%%

\FloatBarrier
\section{Einleitung}

In diesem Versuch beschäftigen wir uns mit verschiedenen Arten von
Operationsverstärkern. Dabei untersuchen wir einen nicht invertierenden
Verstärker, Addierer, Integratoren und Differenzverstärker.

%%%%%%%%%%%%%%%%%%%%%%%%%%%%%%%%%%%%%%%%%%%%%%%%%%%%%%%%%%%%%%%%%%%%%%%%%%%%%%%
%                                  Theorie                                    %
%%%%%%%%%%%%%%%%%%%%%%%%%%%%%%%%%%%%%%%%%%%%%%%%%%%%%%%%%%%%%%%%%%%%%%%%%%%%%%%

\FloatBarrier
\section{Theorie}

\TODO{Theorieteil}

%%%%%%%%%%%%%%%%%%%%%%%%%%%%%%%%%%%%%%%%%%%%%%%%%%%%%%%%%%%%%%%%%%%%%%%%%%%%%%%
%                                  Aufgaben                                   %
%%%%%%%%%%%%%%%%%%%%%%%%%%%%%%%%%%%%%%%%%%%%%%%%%%%%%%%%%%%%%%%%%%%%%%%%%%%%%%%

\FloatBarrier
\section{Aufgaben}

\FloatBarrier
\subsection{Aufgabe A}

\problem{Berechnen Sie $v$ für $k = 0.1$, $v_0 = 10^4$ und $v_0 = 10^5$. Um
wieviel Prozent weicht $v$ jeweils vom angestrebten Wert $1 / k$ ab?}

\begin{align*}
    v_\text{opt} &= \frac 1k = 10\\
    v_{v_0 = 10^4} &= \frac 1{0.1 + 10^{-4} \approx 9.990\\
    v_{v_0 = 10^5} &= \frac 1{0.1 + 10^{-5} \approx 9.999
\end{align*}

Dies entspricht einer Abweichung von $0.1\%$ für $v_0 = 10^4$ und $0.01\%$ für
$v_0 = 10^5$.

\FloatBarrier
\subsection{Aufgabe B}

\problem{Zeigen Sie, dass die Eingangsspannung des Verstärkers allgemein $U_x =
    \Uin / (1+kv_0)$ ist. Wie groß ist sie für den oben betrachteten
Verstärker mit $k = 0.1$, $v_0 = 10^5$, $\Uin = \SI 1{\volt}$?}



%%%%%%%%%%%%%%%%%%%%%%%%%%%%%%%%%%%%%%%%%%%%%%%%%%%%%%%%%%%%%%%%%%%%%%%%%%%%%%%
%                                  Literatur                                  %
%%%%%%%%%%%%%%%%%%%%%%%%%%%%%%%%%%%%%%%%%%%%%%%%%%%%%%%%%%%%%%%%%%%%%%%%%%%%%%%

\FloatBarrier
\IfFileExists{\bibliographyfile}{
	\bibliography{\bibliographyfile}
}{}

\end{document}

% vim: spell spelllang=de tw=79
