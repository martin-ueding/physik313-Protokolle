% Copyright © 2013 Martin Ueding <dev@martin-ueding.de>

\input{header.tex}

\usepackage{placeins}

\ihead{physik313 – Versuch 5}
\ifoot{Lino Lemmer}

\hypersetup{
	pdftitle={Operationsverstärker}
}

\subject{Praktikumsprotokoll}
\title{Operationsverstärker}
\subtitle{physik313 – Versuch 5}
\author{
	Lino Lemmer \footnote{\href{mailto:s6lilemm@uni-bonn.de}{s6lilemm@uni-bonn.de}}
}

%\setcounter{tocdepth}{2}

\newcommand\IB{I_\text{B}}
\newcommand\IC{I_\text{C}}
\newcommand\ID{I_\text{D}}
\newcommand\IE{I_\text{E}}
\newcommand\IS{I_\text{S}}
\newcommand\RC{R_\text{C}}
\newcommand\RD{R_\text{D}}
\newcommand\RE{R_\text{E}}
\newcommand\UBE{U_\text{BE}}
\newcommand\UB{U_\text{B}}
\newcommand\UC{U_\text{C}}
\newcommand\UCE{U_\text{CE}}
\newcommand\UE{U_\text{E}}
\newcommand\UG{U_\text{G}}
\newcommand\UGS{U_\text{GS}}
\newcommand\UDS{U_\text{DS}}
\newcommand\Uin{U_\text{in}}
\newcommand\Uout{U_\text{out}}

\begin{document}

\maketitle

Der \LaTeX-Quelltext zu allen Protokollen in diesem Praktikum kann auf
\ref{it:mu} eingesehen werden. Die Quellen für dieses Protokoll können auf
\ref{it:github/alles} eingesehen werden. Die \LaTeX-Datei wird aus
\ref{it:github/template} generiert.

\begin{enumerate}
	\item
		\label{it:mu}
		\url{http://martin-ueding.de/de/university/physik313/}
	\item
		\label{it:github/alles}
		\url{https://github.com/martin-ueding/physik313-3_4/}
	\item
		\label{it:github/template}
		\url{https://github.com/martin-ueding/physik313-3_4/blob/master/Template_4.tex}
\end{enumerate}

\newpage
\tableofcontents
\newpage

%%%%%%%%%%%%%%%%%%%%%%%%%%%%%%%%%%%%%%%%%%%%%%%%%%%%%%%%%%%%%%%%%%%%%%%%%%%%%%%
%                                 Einleitung                                  %
%%%%%%%%%%%%%%%%%%%%%%%%%%%%%%%%%%%%%%%%%%%%%%%%%%%%%%%%%%%%%%%%%%%%%%%%%%%%%%%

\FloatBarrier
\section{Einleitung}

In diesem Versuch beschäftigen wir uns mit verschiedenen Arten von
Operationsverstärkern. Dabei untersuchen wir einen nicht invertierenden
Verstärker, Addierer, Integratoren und Differenzverstärker.

%%%%%%%%%%%%%%%%%%%%%%%%%%%%%%%%%%%%%%%%%%%%%%%%%%%%%%%%%%%%%%%%%%%%%%%%%%%%%%%
%                                  Theorie                                    %
%%%%%%%%%%%%%%%%%%%%%%%%%%%%%%%%%%%%%%%%%%%%%%%%%%%%%%%%%%%%%%%%%%%%%%%%%%%%%%%

\FloatBarrier
\section{Theorie}

%\TODO{Theorieteil}

%%%%%%%%%%%%%%%%%%%%%%%%%%%%%%%%%%%%%%%%%%%%%%%%%%%%%%%%%%%%%%%%%%%%%%%%%%%%%%%
%                                  Aufgaben                                   %
%%%%%%%%%%%%%%%%%%%%%%%%%%%%%%%%%%%%%%%%%%%%%%%%%%%%%%%%%%%%%%%%%%%%%%%%%%%%%%%

\FloatBarrier
\section{Aufgaben}

\FloatBarrier
\subsection{Aufgabe A}

\begin{problem}
    Berechnen Sie $v$ für $k = 0.1$, $v_0 = 10^4$ und $v_0 = 10^5$. Um
    wieviel Prozent weicht $v$ jeweils vom angestrebten Wert $1 / k$ ab?
\end{problem}

\begin{align*}
    v_\text{opt} &= \frac 1k = 10\\
    v_{v_0 = 10^4} &= \frac 1{0.1 + 10^{-4}} \approx 9.990\\
    v_{v_0 = 10^5} &= \frac 1{0.1 + 10^{-5}} \approx 9.999
\end{align*}

Dies entspricht einer Abweichung von $0.1\%$ für $v_0 = 10^4$ und $0.01\%$ für
$v_0 = 10^5$.

\FloatBarrier
\subsection{Aufgabe B}

\begin{problem}
    Zeigen Sie, dass die Eingangsspannung des Verstärkers allgemein
    $U_x = \Uin / (1+kv_0)$ ist. Wie groß ist sie für den oben betrachteten
    Verstärker mit $k = 0.1$, $v_0 = 10^5$, $\Uin = \SI 1{\volt}$?
\end{problem}

Es gilt
\begin{align*}
    U_x &= \Uin - k\Uout \\
        &= \Uin - kv_0U_x
    \intertext{Hieraus folgt}
    \Uin &= \del{1+kv_0}U_x
    \intertext{und daraus erhält man sofort die gesuchte Beziehung}
    U_x &= \frac {\Uin}{1+kv_0}
    \intertext{Für den oben betrachteten Verstärker ergibt sich}
    U_x &= \SI{e-4}{\volt}
\end{align*}

\FloatBarrier
\subsection{Aufgabe C}

\begin{problem}
    Berechnen Sie nun die Verstärkung eines Gleichtaktsignals (common mode
    signal, CM), also $\Delta U_+ = \Delta U_- = +\Delta \Uin$. Betrachten Sie
    dazu wieder die Änderung der Emitterspannungen und die daraus
    resultierenden Änderungen des Stromes durch $R_1$ (näherungsweise: $R_1 \gg
    \RE$). Zeigen Sie, dass $v_\text{CM} \approx -\RC / (2R_1)$. Wie ist die
    Gleichtaktunterdrückung einer solchen Schaltung für $\RE = \SI
    1{\kilo\ohm}$, $\RC = R_1 = \SI{100}{\kilo\ohm}$?
\end{problem}

Ändert man an beiden Eingängen die Spannung gleichermaßen um $+\Delta\Uin$,
hebt man auf beiden Seiten ebenfalls die Emitterspannungen um diesen Betrag.
Der zusätzliche Spannungsabfall an den Widerständen $\RE$ und $R_1$ findet
wegen $R_1 \gg \RE$ fast vollständig in $R_1$ statt. Der Strom, der dadurch
zusätzlich durch $R_1$ fließt, ist $\Delta I_1 = \Delta\Uin / R_1$. Die
Spannung, die über $\RC$ abfällt ändert sich dadurch um

\[
    \Delta\UC = \RC\Delta\IC = \half\RC\Delta I_1 = \frac \RC {2R_1} \Delta\Uin
\]

$\Uout$ sinkt entsprechend um diesen Betrag.

Für die Gleichtaktverstärkung ergibt sich daher 

\[
    v_\text{CM} = \frac{\Uout}{\Uin} = -\frac \RC {2R_1}
\]

Die Gleichtaktunterdrückung ist dabei

\[
    \text{CMRR} = \frac {v_\text{CM}}{v_\text{diff}} = -\frac
    {\frac{\RC}{2R_1}}{\frac{\RC}{2\RE} = -\frac{\RE}{R_1} = -0.01
\]

\FloatBarrier
\subsection{Aufgabe D}

\begin{problem}
    Betrachten Sie diese Schaltung (gemeint ist Schaltung~\ref{fig:5_6-4}) mit
    $Z_2 = R = \SI {100}{\kilo\ohm}$, $Z_1 = C = \SI {100}{\nano\farad}$. Wie
    ändert sich der Betrag von $Z_1$ mit der Frequenz und was bedeutet das für
    die Verstärkung? Was passiert für $f = 0$ und $f \to \infty$? Für welche
    Frequenz ist $\abs{Z_1} = R$? Berechnen Sie die Ausgangsspannung und
    daraus $v(f) = \abs{\Uout / \Uin}$, in dem Sie die komplexen Impedanzen für
    $R$ und $C$ benutzen. Stimmt Ihre obige Vorhersage für $v(0)$ und
    $v(\infty)$? Skizzieren Sie, wie $v(f)$ in einem Doppellogarithmischen Plot
    (Bode-Diagramm) aussieht!
\end{problem}

Da gilt $\abs{Z_1}=\frac 1{2\pi f\C$ ist, sinkt dieser mit steigender Frequenz
$f$. Für $f = 0$ gilt daher $\abs{Z_1} = \infty$, dadurch $v = 1$ und für $f
\to \infty$ gilt $\abs{Z_1} = 0$, dadurch $v \to \infty$.

Aus $R = \frac 1{2\pi fC}$ ergibt sich $f = \frac 1{2\pi RC =
\SI{15.9}{\hertz}$.

Mit der gegebenen Beziehung erhält man

\[
    v = \abs{1 + \frac {Z_2}{Z_1}} = \abs{1 + \ii 2\pi RC f} = \sqrt{1 +
    4\pi^2R^2C^2 f^2} 
\]

Die Grenzwertbetrachtungen $f = 0$ und $f \to \infty$ ergeben das gleiche
Resultat, wie unsere obige Vermutung.

Eine Skizze der Abhängigkeit befindet sich in Abbildung~\ref{fig:5_6-D}.

\begin{figure}
    \centering
    \includegraphics{5_6-D.pdf}
    \caption{%
        Abhängigkeit der Verstärkung von der Frequenz
    }
    \label{fig:5_6-D}
\end{figure}

%%%%%%%%%%%%%%%%%%%%%%%%%%%%%%%%%%%%%%%%%%%%%%%%%%%%%%%%%%%%%%%%%%%%%%%%%%%%%%%
%                                  Literatur                                  %
%%%%%%%%%%%%%%%%%%%%%%%%%%%%%%%%%%%%%%%%%%%%%%%%%%%%%%%%%%%%%%%%%%%%%%%%%%%%%%%

\FloatBarrier
\IfFileExists{\bibliographyfile}{
	\bibliography{\bibliographyfile}
}{}

\end{document}

% vim: spell spelllang=de tw=79
