% Copyright © 2013 Martin Ueding <dev@martin-ueding.de>

\input{header.tex}

\ihead{physik313 – Versuch 2}
\ifoot{Lino Lemmer}

\hypersetup{
	pdftitle={Diodenkennlinien}
}

\subject{Praktikumsprotokoll}
\title{Diodenkennlinien}
\subtitle{physik313 – Versuch 2}
\author{
	Lino Lemmer \footnote{\href{mailto:s6lilemm@uni-bonn.de}{s6lilemm@uni-bonn.de}}
}

\setcounter{tocdepth}{2}

\begin{document}

\maketitle

Der \LaTeX-Quelltext zu allen Protokollen in diesem Praktikum kann auf
\ref{it:mu} eingesehen werden. Die Quellen für dieses Protokoll können auf
\ref{it:github/alles} eingesehen werden. Die \LaTeX-Datei wird aus
\ref{it:github/template} generiert.

\begin{enumerate}
	\item
		\label{it:mu}
		\url{http://martin-ueding.de/de/university/physik313/}
	\item
		\label{it:github/alles}
		\url{https://github.com/martin-ueding/physik313-2/}
	\item
		\label{it:github/template}
		\url{https://github.com/martin-ueding/physik313-2/blob/master/Template.tex}
\end{enumerate}

\newpage
\tableofcontents
\newpage

%%%%%%%%%%%%%%%%%%%%%%%%%%%%%%%%%%%%%%%%%%%%%%%%%%%%%%%%%%%%%%%%%%%%%%%%%%%%%%%
%                                 Einleitung                                  %
%%%%%%%%%%%%%%%%%%%%%%%%%%%%%%%%%%%%%%%%%%%%%%%%%%%%%%%%%%%%%%%%%%%%%%%%%%%%%%%

\section{Einleitung}

In diesem Versuch untersuchen wir die Kennlinien von Halbleiterdioden. Wir
messen Diodenkennlinien statisch und mit dem Oszillographen. Anschließend
untersuchen wir verschiedene Gleichrichter sowie Spannungsstabilisierung.

%%%%%%%%%%%%%%%%%%%%%%%%%%%%%%%%%%%%%%%%%%%%%%%%%%%%%%%%%%%%%%%%%%%%%%%%%%%%%%%
%                                   Theorie                                   %
%%%%%%%%%%%%%%%%%%%%%%%%%%%%%%%%%%%%%%%%%%%%%%%%%%%%%%%%%%%%%%%%%%%%%%%%%%%%%%%

\section{Theorie}

Dioden bestehen meistens aus Silizium oder Germanium. Ein Teil ist n-dotiert,
der andere p-dotiert. Wenn Plus an die p-Zone gepolt ist, dann ist dieser
pn-Übergang in Durchlassrichtung gepolt
\cite[§~14.1]{beuth/elementare_elektronik}. In der anderen Polung ist die
Diode gesperrt und lässt so gut wie keinen Strom durch.

Bei einer hohen Gegenspannung wird die Diode jedoch durchbrochen, dies ist die
\emph{Durchbruchspannung}. Jenseits dieser Spannung ist die Diode niederohmig
und leitet wieder viel Strom, da die Diode dann zerstört ist. Eine Zenerdiode
wird nicht zerstört und kann so zur Spannungsbegrenzung benutzt werden, siehe
letzte Voraufgaben.

Den Stromverlauf in Abhängigkeit der Spannung wird \emph{Kennlinie} genannt und
taucht in den Voraufgaben einige Male auf. Siehe Abbildung~\ref{fig:2-2}.

\begin{figure}[h]
	\centering
	\caption{%
		Kennlinie einer Diode \cite[Abbildung~2.2]{physik313-Anleitung}
	}
	\label{fig:2-2}
	\includegraphics[width=.45\linewidth]{Bilder_aus_Anleitung/2-2.png}
\end{figure}

Dioden können zur Gleichrichtung benutzt werden. Dabei kann die untere
Halbwelle blockiert werden, oder mit einem Zweiweggleichrichter auch die untere
Halbwelle zu einer positiven Spannung umgewandelt werden. Siehe Abbildung
\ref{fig:2-4}. Mit einem Glättungskondensator kann diese $|\sin(\omega t)|$
Spannung, noch zu einer besseren Gleichspannung mit weniger Restwelligkeit
gemacht werden.

\begin{figure}[h]
	\centering
	\caption{%
		Ein- und Zweiweggleichrichter \cite[Abbildung~2.4]{physik313-Anleitung}
	}
	\label{fig:2-4}
	\includegraphics[width=.7\linewidth]{Bilder_aus_Anleitung/2-4.png}
\end{figure}

Mit einer Zenerdiode kann ein Stromteiler aufgebaut werden, der die
Lastspannung stabilisiert. So können stabilisierte Netzteile gebaut werden.

Für die dynamische Messung der Kennlinien benutzen wir das Oszilloskop. Da es
nur Spannungen anzeigen kann, wandeln wir den Strom mit einem Ohm'schen
Widerstand in eine Spannung um. So kann im $x$-$y$-Betrieb direkt die Kennlinie
sichtbar gemacht werden, wenn das System von einem Sägezahn angetrieben wird.

%%%%%%%%%%%%%%%%%%%%%%%%%%%%%%%%%%%%%%%%%%%%%%%%%%%%%%%%%%%%%%%%%%%%%%%%%%%%%%%
%                                  Aufgaben                                   %
%%%%%%%%%%%%%%%%%%%%%%%%%%%%%%%%%%%%%%%%%%%%%%%%%%%%%%%%%%%%%%%%%%%%%%%%%%%%%%%

\section{Aufgaben}

\subsection{Aufgabe A}

\begin{problem}
	Wieviele Energieniveaus gibt es in den erlaubten Bändern des Bändermodells?
\end{problem}

Bei Silizium gibt es 4 Zustände im Valenzband, keine in der Bandlücke und 4 im
Leitungsband. \cite[Vorlesung~16, Folie~13]{meschede/physik441}

\subsection{Aufgabe B}

\begin{problem}
	Wie uns weshalb dotiert man Halbleiter?
\end{problem}

Dotierung benutzt man, um freie Ladungsträger zu erhalten und so die
Leitfähigkeit zu verbessern. In Halbleitern wäre sonst kein Zustand im
Leitungsband besetzt. Dies erreicht man, in dem man Atome mit anderer
Wertigkeit an Kristallplätze setzt und Si so ersetzt. Dies ist eine gezielte
Verunreinigung.

\subsection{Aufgabe C}

\begin{problem}
	Was sind Donatoren und Akzeptoren?
\end{problem}

Beides sind Atome mit anderer Wertigkeit als die umgebenen Kristallatome. Zum
Beispiel P oder B in einem Si- oder Ge-Gitter. Bei P sind reichen die
Valenzelektronen für die Bindungen zu den Nachbarn aus, es bleibt jedoch noch
ein Elektron übrig, das nur mit einer kleinen Ionisationsenergie an den Kern
gebunden ist. So wird der Kristall einfach leitend.

Donatoren und Akzeptoren bringen also ein zusätzliches, freies Elektron bzw.
Loch in den Kristall. \cite[§~18.4.2]{meschede-gerthsen_24}

\subsection{Aufgabe D}

\begin{problem}
	Was bestimmt die Dicke der Grenzschicht bei einem p-n-Halbleiter?
\end{problem}

Die Schichtdicke wird durch die Diffusionsspannung $U_\text{D}$, die Donator-
und Akzeptorkonzentration $N_\text{D}$ bzw. $N_\text{A}$, die angelegte
Spannung $U$ und die Permittivitätszahl $\varepsilon_\text{r}$ bestimmt.

\subsection{Aufgabe E}

\begin{problem}
	Wie ändert sich die Kapazität einer Diode im Sperrfall mit der angelegten
	Spannung?
\end{problem}

Sie verändert sich nichtlinear. \cite[§~15.2.2]{beuth/elementare_elektronik}

Mit der Diffusionsspannung $U_\text D$ und $n$ von den physikalischen
Eigenschaften der Diode abhängig:
\cite[§~1.1.1.1.2]{antula/schaltungen_mikroelektronik}
\[
	C_R = \frac{C_0}{\del{1- \frac U{U_\text D}}^n}
\]

\subsection{Aufgabe F}

\begin{problem}
	Skizzieren Sie den Kennlinienverlauf, $I = f(U)$, der Zweipole aus
	Abbildung~\ref{fig:2-3} ($R = \SI{100}\ohm$, D = Diode). Erläutern Sie bei
	c) und d) den Einfluss der Widerstände.
\end{problem}

\begin{figure}[h]
	\centering
	\caption{%
		\cite[Abbildung~2.3]{physik313-Anleitung}
	}
	\label{fig:2-3}
	\includegraphics[width=.6\linewidth]{Bilder_aus_Anleitung/2-3.png}
\end{figure}

\subsubsection{Einfacher Widerstand}

Beim einfachen Widerstand $R$ gilt:
\[
	I = \frac 1R U
\]

Dies ist in Abbildung~\ref{fig:F-Widerstand} skizziert.

\begin{figure}[h]
	\centering
	\caption{%
		Kennlinie des Ohm'schen Widerstands
	}
	\label{fig:F-Widerstand}
	\includegraphics{Zeichnungen/F-Widerstand.pdf}
\end{figure}

\subsubsection{Einfache Diode}

Für die Diode gilt die in Abbildung~\ref{fig:2-2} der Aufgabenstellung gezeichnete
Kennlinie, ich habe sie in Abbildung~\ref{fig:F-Diode} selbst gemalt.

\begin{figure}[h]
	\centering
	\caption{%
		Kennlinie der einfachen Diode
	}
	\label{fig:F-Diode}
	\includegraphics{Zeichnungen/F-Diode.pdf}
\end{figure}

\subsubsection{Diode und Widerstand seriell}

Hier gibt die Diode den Stromverlauf vor. Der Widerstand verschlingt jedoch
noch Spannung, wenn Strom fließt. Somit hat die Diode weniger Spannung, es
fließt weniger Strom. Darauf fällt weniger Spannung über dem Widerstand ab, die
Diode hat mehr spannung zur Verfügung. Wir haben versucht, die analytisch zu
lösen, sind jedoch nur anschaulich weitergekommen.

So haben wir uns überlegt, dass der Durchlass erst bei höherer Spannung
einsetzt und dann flacher einsteigt. Dies ist in Abbildung~\ref{fig:F-seriell}
skizziert.

\begin{figure}[h]
	\centering
	\caption{%
		Kennlinie der Reihenschaltung
	}
	\label{fig:F-seriell}
	\includegraphics{Zeichnungen/F-seriell.pdf}
\end{figure}

\subsubsection{Diode und Widerstand parallel}

Hier liegt die gleiche Spannung an Diode und Widerstand an. Die Leitfähigkeiten
addieren sich:
\begin{align*}
	Y_\text{ext} &= Y_\text D + Y_R \\
	I_\text{ext} &= (Y_\text D + Y_R) U \\
	&= \del{\frac{f_\text D(U)}U + \frac 1R} U \\
	&= f_\text D (U) + \frac UR \\
	&= f_\text D (U) + f_R(U) \\
	&= (f_\text D + f_R)(U)
\end{align*}

Somit summieren sich beide Kennlinien auf, siehe Abbildung
\ref{fig:F-parallel}.

\begin{figure}[h]
	\centering
	\caption{%
		Kennlinie der Parallelschaltung
	}
	\label{fig:F-parallel}
	\includegraphics{Zeichnungen/F-parallel.pdf}
\end{figure}

\subsubsection{Ideale Spannungsquelle}

Eine reale Spannungsquelle mit eingestellter Spannung $U_0$ hat einen
Innenwiderstand $R_\text i$. Wenn man die Quelle kurzschließt, fließt der
Strom:
\[
	I = \frac{U_0}{R_\text i}
\]

Wird noch eine externe Spannung angelegt, fließt mehr oder weniger Strom durch
den Innenwiderstand:
\[
	I = \frac{1}{R_\text i} (U_0 + U_\text{ext})
\]

Dies ist in Abbildung~\ref{fig:F-Spannungsquelle} skizziert.

Betrachtet man für die ideale Spannungsquelle den Grenzwert $R_\text{I}\to0$
geht die Steigung in Abbildung~\ref{fig:F-Spannungsquelle} gegen unendlich und
wir erhalten eine Senkrechte.

\begin{figure}[h]
	\centering
	\caption{%
		Kennlinie der realen Spannungsquelle
	}
	\label{fig:F-Spannungsquelle}
	\includegraphics{Zeichnungen/F-Spannungsquelle.pdf}
\end{figure}

\subsubsection{Ideale Stromquelle}

Die ideale Stromquelle hat einen Innenwiderstand $R_\text i$, die Quelle passt
die Spannung aber so an, dass der einstellte Strom $I_0$ fließt. Wenn extern
Spannung $U$ angelegt wird, hat die Stromquelle nur mehr oder weniger zu tun,
der Strom fließt trotzdem.
\[
	I = I_0
\]

Dies haben wir in Abbildung~\ref{fig:F-Stromquelle} skizziert.

\begin{figure}[h]
	\centering
	\caption{%
		Kennlinie der idealen Stromquelle
	}
	\label{fig:F-Stromquelle}
	\includegraphics{Zeichnungen/F-Stromquelle.pdf}
\end{figure}

\subsection{Aufgabe G}

\begin{problem}
	Skizzieren Sie den zeitlichen Verlauf der Ausgangsspannungen der
	Schaltungen in Abbildung~\ref{fig:2-4}~(a) und (b), wenn die
	Eingangsspannung eine weit über der Durchlassspannung der Dioden liegende
	Sinusspannung ist.
\end{problem}

Die eingehende Wechselspannung wird unten abgeschnitten. Da die
Eingangsspannung weit über der Durchlassspannung ist, wird unten nichts
nennenswertes abgeschnitten. Es ergibt sich ein Spannungsverlauf wie in
Abbildung~\ref{fig:G-einfach}.

\begin{figure}[h]
	\centering
	\caption{%
		Spannungsverlauf nach der einfachen Gleichrichtung
	}
	\label{fig:G-einfach}
	\includegraphics{Zeichnungen/G-einfach}
\end{figure}

Bei der zweiten Schaltung wird auch die untere Halbwelle durchgelassen,
allerdings nach oben geklappt. Es kommt zu einem Spannungsverlauf wie in
Abbildung~\ref{fig:G-doppelt}.

\begin{figure}[h]
	\centering
	\caption{%
		Spannungsverlauf nach der doppelten Gleichrichtung
	}
	\label{fig:G-doppelt}
	\includegraphics{Zeichnungen/G-doppelt.pdf}
\end{figure}

\subsection{Aufgabe H}

\begin{problem}
	Wie muss $C$ dimensioniert sein, um die Welligkeit der Spannung über $R$
	möglichst klein zu halten?
\end{problem}

Der Kondensator sollte so groß sein, dass die Kapazität sich in einem Zyklus
nicht komplett entlädt.

Angenommen, das Signal ist ein Rechteck mit Periode T. Dann ist die
Zeitkonstante des Kondensators $\tau = R_\text L C$. Es sollte $\tau \gg T$
gelten. Somit ist $C \gg T R_\text L$.

Wenn $C \to \infty$ geht, muss der Kondensator immer länger aufladen, bis er
die Durchschnittsspannung erreicht hat. Dadurch wird das System träge und zieht
zu beginn beliebig hohe Ströme aus der Diode. Dieser Aspekt wird in einer
späteren Aufgabe noch behandelt.

\subsection{Aufgabe I}
\label{s:I}

\begin{problem}
	Wie würden Sie Strom- und Spannungsmessgerät zur Messung der Kennlinie in
	Durchlassrichtung und in Sperrichtung anordnen? Berücksichten Sie die
	Innenwiderstände der beiden Geräte.
\end{problem}

In Abbildung~\ref{fig:I-Schaltungen} sind zwei Möglichkeiten zur
Kennlinienmessung dargestellt.

\begin{figure}[h]
	\centering
	\caption{%
		Mögliche Schaltungen zur Kennlinienmessung
	}
	\label{fig:I-Schaltungen}
	\includegraphics{Zeichnungen/I-Schaltungen.pdf}
\end{figure}

Schaltung \textcircled 1 hat den Vorteil, dass nur die Spannung, die an der
Diode abfällt gemessen wird. Schaltung \textcircled 2 hat den Vorteil, dass nur
der Strom, der durch die Diode geht, gemessen wird. Der Strom, der durch den
Spannungsmesser geht, wird nicht gemessen.

In \cite[Bild 14.2]{beuth/elementare_elektronik} ist einfach nur Schaltung
\textcircled 1 als „Schaltung zur Aufnahme der Diodenkennlinien $I = f(U)$“
dargestellt.

So ist es wahrscheinlich am sinnvollsten, Schaltung \textcircled 1 für beide
Messungen zu benutzen. Da bei der Sperrung kleine Ströme fließen, allerdings
hohe Spannungen auftreten, ist es vielleicht sinnvoll, dafür \textcircled 2 zu
benutzen, um den Strommessfehler durch den Innenwiderstand des Spannungsmessers
zu vermeiden.

\subsection{Aufgabe J}

\begin{problem}
	Wie kann man sich eine zu einem Strom proportionale Spannung herstellen?
\end{problem}

Man lässt den Strom durch einen Ohm'schen Widerstand laufen, an diesem fällt
dann eine zum Strom proportionale Spannung ab.

Dies erfährt man auch, wenn man etwas weiter ließt und sich
Abbildung~\ref{fig:2-7} aus der Anleitung anschaut.

\begin{figure}[h]
	\centering
	\caption{%
		\cite[Abbildung~2.7]{physik313-Anleitung}
	}
	\label{fig:2-7}
	\includegraphics[width=.6\linewidth]{Bilder_aus_Anleitung/2-7.png}
\end{figure}

\subsection{Aufgabe K}

\begin{problem}
	Für Abbildung~2.8: Berechnen Sie größenordnungsmäßig die größte Kapazität,
	die benutzt werden darf, ohne die Grenzwerte der Si-Diode zu überschreiten.
	Nehmen Sie dazu an, dass sich $U$ beim Einschalten um $\SI{1}\volt$ in
	$\SI{100}{\micro\second}$ ändert und vernachlässigen Sie den Einfluss von
	$R_\text L$.
\end{problem}

Der maximale Durchlassstrom ist \SI{1000}{\milli\ampere}. Wenn $C$ zu groß ist,
zieht $C$ zu viel Strom. In $\Deltaup t := \SI{100}{\micro\second}$ geht die
Spannung um $\Deltaup U := \SI1\volt$ hoch. Die Ladungszunahme ist $\Deltaup Q
= C \Deltaup U$.

Der Strom ist:
\[
	I = \frac{\Deltaup Q}{\Deltaup t}
	= C \frac{\Deltaup U}{\Deltaup t}
\]

Dies muss kleiner als $I_\text{max}$ sein:
\[
	I_\text{max} \frac{\Deltaup t}{\Deltaup U} > C
	\implies
	C < \SI{100}{\micro\farad}
\]

\subsection{Aufgabe L}

\begin{problem}
	Welche Sperrbelastung erfährt die Diode in der Sperrsphase für eine
	Eingangswechselspannung mit einer Amplitude $U_0$? Die Amplitude $U_0$ ist
	der Betrag der maximalen Spannung, der zwischen den beiden Polen der
	Wechselspannungsquelle autritt.
\end{problem}

Wenn die Wechselspannungsquelle gerade ganz negativ ist, so wirkt auf die Diode
einmal die Spannung $- U_0$ von der Spannungsquelle. Außerdem wird noch einmal
eine Spannung $-U_0$ durch den Kondensator auf die Diode. Es liegen also
$-2U_0$ an.

\subsection{Aufgabe M}

\begin{problem}
	Skizzieren Sie den zeitlichen Verlauf der Spannung am Ausgang der
	Schaltungen in Abbildung~\ref{fig:2-9}.
\end{problem}

\begin{figure}[h]
	\centering
	\caption{%
		\cite[Abbildung~2.9]{physik313-Anleitung}
	}
	\label{fig:2-9}
	\includegraphics[width=\linewidth]{Bilder_aus_Anleitung/2-9.png}
\end{figure}

\subsubsection{Schaltung a}

Unterhalb der Sperrspannung kann hier kein Strom fließen. Die Ausgangsspannung
ist quasi identisch null.

\subsubsection{Schaltung b}

Hier kann Strom fließen, allerdings macht die diagonale Diode nichts. Dies ist
ein einfacher Gleichrichter, der Ausgabestrom ist der gleiche wie in Abbildung
\ref{fig:G-einfach}.

\subsubsection{Schaltung c}

Dies ist nur eine andere Darstellung von Schaltung b, so dass die gleichen
Überlegungen auch hier zutreffen. Die Umformung ist in Abbildung
\ref{fig:M-Schaltung_drei} dargestellt.

\begin{figure}[h]
	\centering
	\caption{%
		Umformung von Schaltung c
	}
	\label{fig:M-Schaltung_drei}
	\includegraphics{Zeichnungen/M-Schaltung_drei.pdf}
\end{figure}

\subsection{Aufgabe N}

$U'$ ist die Spannung an der Last. Der Gesamtstrom, der fließt ist:
\[
	I = \frac{U_0}{R + R_\text L}
\]

\newcommand\RL{R_\text L}

Die Spannung $U'$ ist:
\begin{align*}
	U'
	&= \RL I \\
	&= \RL \frac{U_0}{R + R_\text L} \\
	&= \frac{U_0 \RL}{R + \RL} \\
	&= \frac{U_0}{1 + \frac R\RL}
\end{align*}

Die Funktion ist in Abbildung~\ref{fig:N-Plot} skizziert.

\begin{figure}[h]
	\centering
	\caption{%
		Lastspannung in Abhängigkeit vom Lastwiderstand
	}
	\label{fig:N-Plot}
	\includegraphics{Zeichnungen/N-Plot.pdf}
\end{figure}

Die Extremwerte sind $I_\text{max} = U_0/R$, $U'_\text{min} = 0$, wenn
$\RL = 0$, sowie $I_\text{min} = 0$ und $U'_\text{max} = U_0$ wenn $\RL \to \infty$.

\subsection{Aufgabe O}

\newcommand\IZmax{I_\text{Z,max}}
\newcommand\IZmin{I_\text{Z,min}}
\newcommand\IZ{I_\text Z}
\newcommand\UZ{U_\text Z}

Der Strom $\IZmax$ darf nicht überschritten werden. Mit dem Ansatz aus 
\cite[§~15.1.3]{beuth/elementare_elektronik} betrachte ich zuerst $\RL =
\infty$. Dann fällt an der Zenerdiode gerade $\UZ$ ab, über $R$ fällt $U - \UZ$
ab. Damit der Strom nicht überschritten wird, muss gelten:
\[
	R > \frac{U - \UZ}\IZmax
\]

%Diese Bedingung ist erfüllt:
%\[
	%\SI{200}\ohm > \frac{\SI{8.2}\volt}{\SI{100}{\milli\ampere}} = \SI{82}\ohm
%\]

Der extremste Fall tritt ein, wenn $U$ maximal wird. Also ist die untere
Schranke für $R$:
\[
	R > \frac{\SI{22}\volt - \SI{8.2}\volt}{\SI{100}{\milli\ampere}}
	= \SI{138}\ohm
\]

Dies ist die untere Grenze. Wenn weniger als $\IZmin$ durch die Zenerdiode
fließen, ist die Stabilisierung auch weg. Also im unbelasteten Fall:
\[
	\frac U\IZmin > R
\]

Nehme minimales $U$ für Extremwert:
\[
	R < \SI{8}{\kilo\ohm}
\]

Wenn jetzt die Last dazukommt, wird der Zenerstrom allerdings kleiner. Sobald
dann jedoch die Spannung wieder ansteigt, sollte der Zenerstrom wieder groß
genug werden.

Somit ist der Bereich \SIrange{138}{8000}{\ohm}.

%%%%%%%%%%%%%%%%%%%%%%%%%%%%%%%%%%%%%%%%%%%%%%%%%%%%%%%%%%%%%%%%%%%%%%%%%%%%%%%
%                      Versuchsaufbau und -durchführung                      %
%%%%%%%%%%%%%%%%%%%%%%%%%%%%%%%%%%%%%%%%%%%%%%%%%%%%%%%%%%%%%%%%%%%%%%%%%%%%%%%

\section{Versuchsaufbau und -durchführung}

\subsection{Versuchsaufgabe 1: Statische Messung der Diodenkennlinie}

In diesem Versuch soll mit Hilfe eines Gleichspannungsnetzgerätes und zweier
Multimeter die Kennlinien der Silizium-Diode MRA4004 und der Schottky-Diode
10BG015. Wir verwenden dafür \fehlt und \fehlt. Die Anordnung ist in \ref{s:I}
beschrieben (Abb.~\ref{fig:I-Schaltungen}). Die Messung in Durchlass- und in
Sperrrichtung werden dabei nacheinander durchgeführt.

\subsubsection{Diode D1}

Unsere Messwerte für die Diode~D1 sind in Tabelle~\ref{table:D1}. Der
entsprechende Plot der Daten ist in Abbildung~\ref{fig:D1}.

\begin{table}[h]
	\centering
	\caption{%
		Messdaten zur Diode D1
	}
	\label{table:D1}
	\begin{tabular}{SSS}
		{$U / \si\volt$} & {$I_1 / \si{\milli\ampere}$} & {$I_2 / \si{\milli\ampere}$} \\
		\hline
		%< for u, i1, u2 in a1_D1: >%
		<< u >> & << i1 >> & << u2 >> \\
		%< endfor >%
	\end{tabular}
\end{table}

\begin{figure}[h]
	\centering
	\caption{%
		Plot zur Diode D1
	}
	\label{fig:D1}
	\includegraphics[width=\textwidth]{Daten_D1.pdf}
\end{figure}

\subsubsection{Diode D2}

Unsere Messwerte für die Diode~D2 sind in Tabelle~\ref{table:D2}. Der
entsprechende Plot der Daten ist in Abbildung~\ref{fig:D2}

\begin{table}[h]
	\centering
	\caption{%
		Messdaten zur Diode D2
	}
	\label{table:D2}
	\begin{tabular}{SS}
		{$U / \si\volt$} & {$I_1 / \si{\milli\ampere}$} \\
		\hline
		%< for u, i in a1_D2: >%
		<< u >> & << i >> \\
		%< endfor >%
	\end{tabular}
\end{table}

\begin{figure}[h]
	\centering
	\caption{%
		Plot zur Diode D2
	}
	\label{fig:D2}
	\includegraphics[width=\textwidth]{Daten_D2.pdf}
\end{figure}

\subsection{Versuchsaufgabe 2: Oszillogramm der Diodenkennlinie}

Die Kennlinien der in Versuch 1 statisch vermessenen Dioden und einer Zener-Diode sollen mit Hilfe eines Osszilographen vermessen werden. Dazu wird der in Abbildung~\ref{fig:2-6} gezeigte Aufbau verwendet.

\begin{figure}
	\centering
	\caption{%
		\cite[Abbildung~2.6]{physik313-Anleitung}
	}
	\label{fig:2-6}
	\includegraphics[width=.8\linewidth]{Bilder_aus_Anleitung/2-6.png}
\end{figure}

Ein auf Dreieck \SI{200}{\kilo\hertz} gestellter Signalgenerator wird an BNC~1 angeschlossen. Je nach zu messender Diode werden B3 mit B8 (MRA4004), B4 mit B9 (10BG015) oder B5 mit B10 (Zener) verbunden. Die Kanäle CH1 und CH2 werden an BNC~2 bzw. BNC~3 angeschlossen um die zum Strom proportionale Spannung abzunehmen. Im Oszillographen werden die Kanäle invers addiert. 

%%%%%%%%%%%%%%%%%%%%%%%%%%%%%%%%%%%%%%%%%%%%%%%%%%%%%%%%%%%%%%%%%%%%%%%%%%%%%%%
%                                 Auswertung                                  %
%%%%%%%%%%%%%%%%%%%%%%%%%%%%%%%%%%%%%%%%%%%%%%%%%%%%%%%%%%%%%%%%%%%%%%%%%%%%%%%

\section{Auswertung}

%%%%%%%%%%%%%%%%%%%%%%%%%%%%%%%%%%%%%%%%%%%%%%%%%%%%%%%%%%%%%%%%%%%%%%%%%%%%%%%
%                                  Ergebnis                                   %
%%%%%%%%%%%%%%%%%%%%%%%%%%%%%%%%%%%%%%%%%%%%%%%%%%%%%%%%%%%%%%%%%%%%%%%%%%%%%%%

\section{Ergebnis}

\IfFileExists{\bibliographyfile}{
	\bibliography{\bibliographyfile}
}{}

\end{document}

% vim: spell spelllang=de
