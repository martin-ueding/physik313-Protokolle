% Copyright © 2013 Martin Ueding <dev@martin-ueding.de>

\input{header.tex}

\ihead{physik313 – Versuch 2}
\ifoot{Ueding, Lemmer}

\hypersetup{
	pdftitle={Diodenkennlinien}
}

\subject{Praktikumsprotokoll}
\title{Diodenkennlinien}
\subtitle{physik313 – Versuch 2}
\author{
	Martin Ueding \footnote{\href{mailto:mu@martin-ueding.de}{mu@martin-ueding.de}}
	\and
	Lino Lemmer \footnote{\href{mailto:s6lilemm@uni-bonn.de}{s6lilemm@uni-bonn.de}}
}

\begin{document}

\maketitle

%\newpage
\tableofcontents
\newpage

%%%%%%%%%%%%%%%%%%%%%%%%%%%%%%%%%%%%%%%%%%%%%%%%%%%%%%%%%%%%%%%%%%%%%%%%%%%%%%%
%                                 Einleitung                                  %
%%%%%%%%%%%%%%%%%%%%%%%%%%%%%%%%%%%%%%%%%%%%%%%%%%%%%%%%%%%%%%%%%%%%%%%%%%%%%%%

\section{Einleitung}

%%%%%%%%%%%%%%%%%%%%%%%%%%%%%%%%%%%%%%%%%%%%%%%%%%%%%%%%%%%%%%%%%%%%%%%%%%%%%%%
%                                   Theorie                                   %
%%%%%%%%%%%%%%%%%%%%%%%%%%%%%%%%%%%%%%%%%%%%%%%%%%%%%%%%%%%%%%%%%%%%%%%%%%%%%%%


%%%%%%%%%%%%%%%%%%%%%%%%%%%%%%%%%%%%%%%%%%%%%%%%%%%%%%%%%%%%%%%%%%%%%%%%%%%%%%%
%                                  Aufgaben                                   %
%%%%%%%%%%%%%%%%%%%%%%%%%%%%%%%%%%%%%%%%%%%%%%%%%%%%%%%%%%%%%%%%%%%%%%%%%%%%%%%

\section{Aufgaben}

\subsection{Aufgabe F}

\subsubsection{Einfacher Widerstand}

Beim einfachen Widerstand $R$ gilt:
\[
	I = \frac 1R U
\]

Dies ist in Abbildung \ref{fig:F-Widerstand} skizziert.

\begin{figure}[h]
	\centering
	\caption{%
		Kennlinie des Ohm'schen Widerstands
	}
	\label{fig:F-Widerstand}
	\includegraphics{Zeichnungen/F-Widerstand.pdf}
\end{figure}

\subsubsection{Einfache Diode}

Für die Diode gilt die in Abbildung 2.2 der Aufgabenstellung gezeichnete
Kennlinie, ich habe sie in Abbildung \ref{fig:F-Diode} selbst gemalt.

\begin{figure}[h]
	\centering
	\caption{%
	}
	\label{fig:F-Diode}
	\includegraphics{Zeichnungen/F-Diode.pdf}
\end{figure}

\subsubsection{Diode und Widerstand seriell}

Hier gibt die Diode den Stromverlauf vor. Der Widerstand verschlingt jedoch
noch Spannung, wenn Strom fließt. Somit hat die Diode weniger Spannung, es
fließt weniger Strom. Darauf fällt weniger Spannung über dem Widerstand ab, die
Diode hat mehr spannung zur Verfügung. Wir haben versucht, die analytisch zu
lösen, sind jedoch nur anschaulich weitergekommen.

So haben wir uns überlegt, dass der Durchlass erst bei höherer Spannung
einsetzt und dann flacher einsteigt. Dies ist in Abbildung \ref{fig:F-seriell}
skizziert.

\begin{figure}[h]
	\centering
	\caption{%
	}
	\label{fig:F-seriell}
	\includegraphics{Zeichnungen/F-seriell.pdf}
\end{figure}

\subsubsection{Diode und Widerstand parallel}

Hier liegt die gleiche Spannung an Diode und Widerstand an. Die Leitfähigkeiten
addieren sich:
\begin{align*}
	Y_\text{ext} &= Y_\text D + Y_R \\
	I_\text{ext} &= (Y_\text D + Y_R) U \\
	&= \del{\frac{f_\text D(U)}U + \frac 1R} U \\
	&= f_\text D (U) + \frac UR
\end{align*}

Somit summieren sich beide Kennlinien auf, siehe Abbildung
\ref{fig:F-parallel}.

\begin{figure}[h]
	\centering
	\caption{%
	}
	\label{fig:F-parallel}
	\includegraphics{Zeichnungen/F-parallel.pdf}
\end{figure}

\subsubsection{Ideale Stromquelle}

Abbildung \ref{fig:F-Spannungsquelle}

\begin{figure}[h]
	\centering
	\caption{%
	}
	\label{fig:F-Stromquelle}
	%\includegraphics{Zeichnungen/F-Stromquelle.pdf}
\end{figure}

\subsubsection{Ideale Spannungsquelle}

Abbildung \ref{fig:F-Stromquelle}

\begin{figure}[h]
	\centering
	\caption{%
	}
	\label{fig:F-Spannungsquelle}
	%\includegraphics{Zeichnungen/F-Diode.pdf}
\end{figure}

%%%%%%%%%%%%%%%%%%%%%%%%%%%%%%%%%%%%%%%%%%%%%%%%%%%%%%%%%%%%%%%%%%%%%%%%%%%%%%%
%                      Versuchsaufbau und -durchführung                      %
%%%%%%%%%%%%%%%%%%%%%%%%%%%%%%%%%%%%%%%%%%%%%%%%%%%%%%%%%%%%%%%%%%%%%%%%%%%%%%%

\section{Versuchsaufbau und -durchführung}


%%%%%%%%%%%%%%%%%%%%%%%%%%%%%%%%%%%%%%%%%%%%%%%%%%%%%%%%%%%%%%%%%%%%%%%%%%%%%%%
%                                 Auswertung                                  %
%%%%%%%%%%%%%%%%%%%%%%%%%%%%%%%%%%%%%%%%%%%%%%%%%%%%%%%%%%%%%%%%%%%%%%%%%%%%%%%

\section{Auswertung}

%%%%%%%%%%%%%%%%%%%%%%%%%%%%%%%%%%%%%%%%%%%%%%%%%%%%%%%%%%%%%%%%%%%%%%%%%%%%%%%
%                                  Ergebnis                                   %
%%%%%%%%%%%%%%%%%%%%%%%%%%%%%%%%%%%%%%%%%%%%%%%%%%%%%%%%%%%%%%%%%%%%%%%%%%%%%%%

\section{Ergebnis}

\IfFileExists{\bibliographyfile}{
	\bibliography{\bibliographyfile}
}{}

\end{document}

% vim: spell spelllang=de
