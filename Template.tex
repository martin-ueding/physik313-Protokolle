% Copyright © 2013 Martin Ueding <dev@martin-ueding.de>

\input{header.tex}

\ihead{physik313 – Versuch 1}
\ifoot{Ueding, Lemmer}

\hypersetup{
	pdftitle={Ausbreitung von Signalen auf Leitungen}
}

\subject{Praktikumsprotokoll}
\title{Ausbreitung von Signalen auf Leitungen}
\subtitle{physik313 – Versuch 1}
\author{
	Martin Ueding \footnote{\href{mailto:mu@martin-ueding.de}{mu@martin-ueding.de}}
	\and
	Lino Lemmer \footnote{\href{mailto:s6lilemm@uni-bonn.de}{s6lilemm@uni-bonn.de}}
}

\begin{document}

\maketitle

%\newpage
\tableofcontents
\newpage

%%%%%%%%%%%%%%%%%%%%%%%%%%%%%%%%%%%%%%%%%%%%%%%%%%%%%%%%%%%%%%%%%%%%%%%%%%%%%%%
%                                 Einleitung                                  %
%%%%%%%%%%%%%%%%%%%%%%%%%%%%%%%%%%%%%%%%%%%%%%%%%%%%%%%%%%%%%%%%%%%%%%%%%%%%%%%

\section{Einleitung}

In diesem Versuch experimentieren wir mit der Ausbreitung von Pulsen auf
Koaxialkabeln. Dabei betrachten wir Verzögerung, Reflexion und
Impulsdeformation.

%%%%%%%%%%%%%%%%%%%%%%%%%%%%%%%%%%%%%%%%%%%%%%%%%%%%%%%%%%%%%%%%%%%%%%%%%%%%%%%
%                                   Theorie                                   %
%%%%%%%%%%%%%%%%%%%%%%%%%%%%%%%%%%%%%%%%%%%%%%%%%%%%%%%%%%%%%%%%%%%%%%%%%%%%%%%

\section{Theorie}

\subsection{Phasengeschwindigkeit}

Formel (1.13) aus der Anleitung:
\begin{equation}
	\label{eq:1.13}
	v_\text{ph} = c_0 \frac{1}{\sqrt{\varepsilon_0 \mu_0}}
\end{equation}

Formel (1.15) aus der Anleitung:
\begin{equation}
	\label{eq:1.15}
	Z = \sqrt{\frac{\mu_r \mu_0}{\varepsilon_r \varepsilon_0}} \frac{\ln(R_\text a/R_\text i)}{2\piup}
\end{equation}

%%%%%%%%%%%%%%%%%%%%%%%%%%%%%%%%%%%%%%%%%%%%%%%%%%%%%%%%%%%%%%%%%%%%%%%%%%%%%%%
%                                  Aufgaben                                   %
%%%%%%%%%%%%%%%%%%%%%%%%%%%%%%%%%%%%%%%%%%%%%%%%%%%%%%%%%%%%%%%%%%%%%%%%%%%%%%%

\section{Aufgaben}

\subsection{Aufgabe A}

\begin{quote}
	Was muss man tun, um große Verzögerungszeiten zu erreichen?
\end{quote}

Eine große Verzögerungszeit bedeutet eine kleine Phasengeschwindigkeit. Diese
ist in Formel \eqref{eq:1.13} gegeben. Es kann entweder $\mu_r$ oder
$\varepsilon_r$ groß gemacht werden.

Außerdem kann das Kabel verlängert werden, dann braucht das Signal auch länger.

\subsection{Aufgabe B}

\begin{quote}
	Welche Konsequenz für den Wellenwiderstand haben die verschiedenen
	Möglichkeiten, die Verzögerungszeiten zu verändern?
\end{quote}

\paragraph{Permittivität erhöhen}

Durch Einfügen eines Dielektrikums kann die Permittivität erhöht werden,
dadurch werden die Kabel schwerer und teurer. Nach \eqref{eq:1.15} sinkt die
Impedanz $Z$ des Kabels mit größer werdender Permittivität.

\paragraph{Permeabilität erhöhen}

Durch Erhöhen der Permeabilität wird ebenfalls mehr Material gebraucht. Die
Impedanz steigt nach \eqref{eq:1.15} jetzt allerdings an.

Durch eine geschickte Kombination von Permittivität und Permeabilität kann die
Impedanz gleich gehalten werden und die Phasengeschwindigkeit verringert
werden.

\paragraph{Kabel verlängern}

Ein längeres Kabel bedeutet auch erhöhte normale Ohm'sche Verluste. Es braucht
auch mehr Material und somit mehr Gewicht und Kosten.

%%%%%%%%%%%%%%%%%%%%%%%%%%%%%%%%%%%%%%%%%%%%%%%%%%%%%%%%%%%%%%%%%%%%%%%%%%%%%%%
%                      Versuchsaufbau und -durchführung                      %
%%%%%%%%%%%%%%%%%%%%%%%%%%%%%%%%%%%%%%%%%%%%%%%%%%%%%%%%%%%%%%%%%%%%%%%%%%%%%%%

\section{Versuchsaufbau und -durchführung}

\subsection{Seriennummern}

Die Seriennummern unserer Geräte haben wir in \cref{tb:seriennummern} aufgelistet.

\begin{table}
	\center
	\caption{Seriennummern unserer Geräte}
	\label{tb:seriennummern}
	\begin{tabular}{ll}
		Gerät & Seriennummer \\
		\hline
		Oszillograph & \\
		Funktionsgenerator & \\
		Kabel HH 2500 & \\
		Kabelkasten (RG-58 C/U) & \\
		Differenzierglied & \\
		Differenzierglied mit \SI{2.2}{\kilo\ohm} Anpassung & \\
		Anpasswiderstand: \SI{2.45}{\kilo\ohm} & \\
		Abschlusswiderstand: \SI{2.5}{\kilo\ohm} und \SI{50}{\ohm} & \\
		Einstellbarer Abschlusswiderstand: \SIrange{0}{10}{\kilo\ohm} &
	\end{tabular}
\end{table}

%%%%%%%%%%%%%%%%%%%%%%%%%%%%%%%%%%%%%%%%%%%%%%%%%%%%%%%%%%%%%%%%%%%%%%%%%%%%%%%
%                                 Auswertung                                  %
%%%%%%%%%%%%%%%%%%%%%%%%%%%%%%%%%%%%%%%%%%%%%%%%%%%%%%%%%%%%%%%%%%%%%%%%%%%%%%%

\section{Auswertung}

%%%%%%%%%%%%%%%%%%%%%%%%%%%%%%%%%%%%%%%%%%%%%%%%%%%%%%%%%%%%%%%%%%%%%%%%%%%%%%%
%                                  Ergebnis                                   %
%%%%%%%%%%%%%%%%%%%%%%%%%%%%%%%%%%%%%%%%%%%%%%%%%%%%%%%%%%%%%%%%%%%%%%%%%%%%%%%

\section{Ergebnis}

%\IfFileExists{\bibliographyfile}{
%	\bibliography{\bibliographyfile}
%}{}

\end{document}

% vim: spell spelllang=de
