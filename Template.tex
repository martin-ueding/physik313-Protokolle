% Copyright © 2013 Martin Ueding <dev@martin-ueding.de>

\input{header.tex}

\ihead{physik313 – Versuch 2}
\ifoot{Ueding, Lemmer}

\hypersetup{
	pdftitle={Diodenkennlinien}
}

\subject{Praktikumsprotokoll}
\title{Diodenkennlinien}
\subtitle{physik313 – Versuch 2}
\author{
	Martin Ueding \footnote{\href{mailto:mu@martin-ueding.de}{mu@martin-ueding.de}}
	\and
	Lino Lemmer \footnote{\href{mailto:s6lilemm@uni-bonn.de}{s6lilemm@uni-bonn.de}}
}

\begin{document}

\maketitle

%\newpage
\tableofcontents
\newpage

%%%%%%%%%%%%%%%%%%%%%%%%%%%%%%%%%%%%%%%%%%%%%%%%%%%%%%%%%%%%%%%%%%%%%%%%%%%%%%%
%                                 Einleitung                                  %
%%%%%%%%%%%%%%%%%%%%%%%%%%%%%%%%%%%%%%%%%%%%%%%%%%%%%%%%%%%%%%%%%%%%%%%%%%%%%%%

\section{Einleitung}

%%%%%%%%%%%%%%%%%%%%%%%%%%%%%%%%%%%%%%%%%%%%%%%%%%%%%%%%%%%%%%%%%%%%%%%%%%%%%%%
%                                   Theorie                                   %
%%%%%%%%%%%%%%%%%%%%%%%%%%%%%%%%%%%%%%%%%%%%%%%%%%%%%%%%%%%%%%%%%%%%%%%%%%%%%%%

%%%%%%%%%%%%%%%%%%%%%%%%%%%%%%%%%%%%%%%%%%%%%%%%%%%%%%%%%%%%%%%%%%%%%%%%%%%%%%%
%                                  Aufgaben                                   %
%%%%%%%%%%%%%%%%%%%%%%%%%%%%%%%%%%%%%%%%%%%%%%%%%%%%%%%%%%%%%%%%%%%%%%%%%%%%%%%

\section{Aufgaben}

\subsection{Aufgabe A}

Bei Silizium gibt es 4 Zustände im Valenzband, keine in der Bandlücke und 4 im
Leitungsband.

\subsection{Aufgabe B}

Dotierung benutzt man, um freie Ladungsträger zu erhalten. In Halbleitern wäre
sonst kein Zustand im Leitungsband besetzt.

\subsection{Aufgabe C}

\fehlt

\subsection{Aufgabe D}

\fehlt

\subsection{Aufgabe E}

\fehlt

\subsection{Aufgabe F}

\subsubsection{Einfacher Widerstand}

Beim einfachen Widerstand $R$ gilt:
\[
	I = \frac 1R U
\]

Dies ist in Abbildung \ref{fig:F-Widerstand} skizziert.

\begin{figure}[h]
	\centering
	\caption{%
		Kennlinie des Ohm'schen Widerstands
	}
	\label{fig:F-Widerstand}
	\includegraphics{Zeichnungen/F-Widerstand.pdf}
\end{figure}

\subsubsection{Einfache Diode}

Für die Diode gilt die in Abbildung 2.2 der Aufgabenstellung gezeichnete
Kennlinie, ich habe sie in Abbildung \ref{fig:F-Diode} selbst gemalt.

\begin{figure}[h]
	\centering
	\caption{%
		Kennlinie der einfachen Diode
	}
	\label{fig:F-Diode}
	\includegraphics{Zeichnungen/F-Diode.pdf}
\end{figure}

\subsubsection{Diode und Widerstand seriell}

Hier gibt die Diode den Stromverlauf vor. Der Widerstand verschlingt jedoch
noch Spannung, wenn Strom fließt. Somit hat die Diode weniger Spannung, es
fließt weniger Strom. Darauf fällt weniger Spannung über dem Widerstand ab, die
Diode hat mehr spannung zur Verfügung. Wir haben versucht, die analytisch zu
lösen, sind jedoch nur anschaulich weitergekommen.

So haben wir uns überlegt, dass der Durchlass erst bei höherer Spannung
einsetzt und dann flacher einsteigt. Dies ist in Abbildung \ref{fig:F-seriell}
skizziert.

\begin{figure}[h]
	\centering
	\caption{%
		Kennlinie der Reihenschaltung
	}
	\label{fig:F-seriell}
	\includegraphics{Zeichnungen/F-seriell.pdf}
\end{figure}

\subsubsection{Diode und Widerstand parallel}

Hier liegt die gleiche Spannung an Diode und Widerstand an. Die Leitfähigkeiten
addieren sich:
\begin{align*}
	Y_\text{ext} &= Y_\text D + Y_R \\
	I_\text{ext} &= (Y_\text D + Y_R) U \\
	&= \del{\frac{f_\text D(U)}U + \frac 1R} U \\
	&= f_\text D (U) + \frac UR \\
	&= f_\text D (U) + f_R(U) \\
	&= (f_\text D + f_R)(U)
\end{align*}

Somit summieren sich beide Kennlinien auf, siehe Abbildung
\ref{fig:F-parallel}.

\begin{figure}[h]
	\centering
	\caption{%
		Kennlinie der Parallelschaltung
	}
	\label{fig:F-parallel}
	\includegraphics{Zeichnungen/F-parallel.pdf}
\end{figure}

\subsubsection{Ideale Stromquelle}

Die ideale Stromquelle hat einen Innenwiderstand $R_\text i$, die Quelle passt
die Spannung aber so an, dass der einstellte Strom $I_0$ fließt. Wenn extern
Spannung $U$ angelegt wird, hat die Stromquelle nur mehr oder weniger zu tun,
der Strom fließt trotzdem.
\[
	I = I_0
\]

Dies haben wir in Abbildung \ref{fig:F-Stromquelle} skizziert.

\begin{figure}[h]
	\centering
	\caption{%
		Kennlinie der idealen Stromquelle
	}
	\label{fig:F-Stromquelle}
	\includegraphics{Zeichnungen/F-Stromquelle.pdf}
\end{figure}

\subsubsection{Ideale Spannungsquelle}

Eine ideale Spannungsquelle mit eingestellter Spannung $U_0$ hat einen
Innenwiederstand $R_\text i$. Wenn man die Quelle kurzschließt, fließt der
Strom:
\[
	I = \frac{U_0}{R_\text i}
\]

Wir noch eine externe Spannung angelegt, fließt mehr oder weniger Strom durch den Innenwiderstand:
\[
	I = \frac{1}{R_\text i} (U_0 + U_\text{ext})
\]

Dies ist in Abbildung \ref{fig:F-Spannungsquelle} skizziert.

\begin{figure}[h]
	\centering
	\caption{%
		Kennlinie der idealen Spannungsquelle
	}
	\label{fig:F-Spannungsquelle}
	\includegraphics{Zeichnungen/F-Spannungsquelle.pdf}
\end{figure}

\subsection{Aufgabe G}

Die eingehende Wechselspannung wird unten abgeschnitten. Da die Eingangsspannung weit über der Durchlassspannung ist, wird unten nichts nennenswertes abgeschnitten. Es ergibt sich ein Spannungsverlauf wie in Abbildung \ref{fig:G-einfach}.

\begin{figure}[h]
	\centering
	\caption{%
		Spannungsverlauf nach der einfachen Gleichrichtung
	}
	\label{fig:G-einfach}
	\includegraphics{Zeichnungen/G-einfach}
\end{figure}

Bei der zweiten Schaltung wird auch die untere Halbwelle durchgelassen, allerdings nach oben geklappt. Es kommt zu einem Spannungsverlauf wie in Abbildung \ref{fig:G-doppelt}.

\begin{figure}[h]
	\centering
	\caption{%
		Spannungsverlauf nach der doppelten Gleichrichtung
	}
	\label{fig:G-doppelt}
	\includegraphics{Zeichnungen/G-doppelt.pdf}
\end{figure}

\subsection{Aufgabe H}

Der Kondensator sollte so groß sein, dass die Kapazität sich in einem Zyklus nicht komplett entlädt.

Angenommen, das Signal ist ein Rechteck mit Periode T. Dann ist die
Zeitkonstante des Kondensators $\tau = R_\text L C$. Es sollte $\tau \gg T$
gelten. Somit ist $C \gg T R_\text L$.

Wenn $C \to \infty$ geht, muss der Kondensator immer länger aufladen, bis er
die Durchschnittsspannung erreicht hat. Dadurch wird das System träge und zieht
zu beginn beliebig hohe Ströme aus der Diode. Dieser Aspekt wird in einer
späteren Aufgabe noch behandelt.

\subsection{Aufgabe I}

In Abbildung \ref{fig:I-Schaltungen} sind zwei Möglichkeiten zur
Kennlinienmessung dargestellt.

\begin{figure}[h]
	\centering
	\caption{%
		Mögliche Schaltungen zur Kennlinienmessung
	}
	\label{fig:I-Schaltungen}
	\includegraphics{Zeichnungen/I-Schaltungen.pdf}
\end{figure}

Schaltung \textcircled 1 hat den Vorteil, dass nur die Spannung, die an der
Diode abfällt gemessen wird. Schaltung \textcircled 2 hat den Vorteil, dass nur
der Strom, der durch die Diode geht, gemessen wird. Der Strom, der durch den
Spannungsmesser geht, wird nicht gemessen.

In \cite[Bild 14.2]{beuth/elementare_elektronik} ist einfach nur Schaltung
\textcircled 1 als „Schaltung zur Aufnahme der Diodenkennlinien $I = f(U)$“
dargestellt.

So ist es wahrscheinlich am sinnvollsten, Schaltung \textcircled 1 für beide
Messungen zu benutzen. Da bei der Sperrung kleine Ströme fließen, allerdings
hohe Spannungen auftreten, ist es vielleicht sinnvoll, dafür \textcircled 2 zu
benutzen, um den Strommessfehler durch den Innenwiderstand des Spannungsmessers
zu vermeiden.

\subsection{Aufgabe J}

Man lässt den Strom durch einen Ohm'schen Widerstand laufen, an diesem fällt
dann eine zum Strom proportionale Spannung ab.

Dies erfährt man auch, wenn man etwas weiter ließt und sich Abbildung 2.7 aus
der Anleitung anschaut.

\subsection{Aufgabe K}

Der maximale Durchlassstrom ist \SI{1000}{\milli\ampere}. Wenn $C$ zu groß ist, zieht $C$ zu viel Strom. In $\Deltaup t := \SI{100}{\micro\second}$ geht die Spannung um $\Deltaup U := \SI1\volt$ hoch. Die Ladungszunahme ist $\Deltaup Q = C \Deltaup U$.

Der Strom ist:
\[
	I = \frac{\Deltaup Q}{\Deltaup t}
	= C \frac{\Deltaup U}{\Deltaup t}
\]

Dies muss kleiner als $I_\text{max}$ sein:
\[
	I_\text{max} \frac{\Deltaup t}{\Deltaup U} > C
	\implies
	C < \SI{100}{\micro\farad}
\]

\subsection{Aufgabe L}

Wenn die Wechselspannungsquelle gerade ganz negativ ist, so wirkt auf die Diode
einmal die Spannung $- U_0$ von der Spannungsquelle. Außerdem wird noch einmal
eine Spannung $-U_0$ durch den Kondensator auf die Diode. Es liegen also
$-2U_0$ an.

\subsection{Aufgabe M}

\fehlt

\subsection{Aufgabe N}

$U'$ ist die Spannung an der Last. Der Gesamtstrom, der fließt ist:
\[
	I = \frac{U_0}{R + R_\text L}
\]

\newcommand\RL{R_\text L}

Die Spannung $U'$ ist:
\begin{align*}
	U'
	&= \RL I \\
	&= \RL \frac{U_0}{R + R_\text L} \\
	&= \frac{U_0 \RL}{R + \RL} \\
	&+ \frac{U_0}{1 + \frac R\RL}
\end{align*}

\subsection{Aufgabe O}

\fehlt

%%%%%%%%%%%%%%%%%%%%%%%%%%%%%%%%%%%%%%%%%%%%%%%%%%%%%%%%%%%%%%%%%%%%%%%%%%%%%%%
%                      Versuchsaufbau und -durchführung                      %
%%%%%%%%%%%%%%%%%%%%%%%%%%%%%%%%%%%%%%%%%%%%%%%%%%%%%%%%%%%%%%%%%%%%%%%%%%%%%%%

\section{Versuchsaufbau und -durchführung}

%%%%%%%%%%%%%%%%%%%%%%%%%%%%%%%%%%%%%%%%%%%%%%%%%%%%%%%%%%%%%%%%%%%%%%%%%%%%%%%
%                                 Auswertung                                  %
%%%%%%%%%%%%%%%%%%%%%%%%%%%%%%%%%%%%%%%%%%%%%%%%%%%%%%%%%%%%%%%%%%%%%%%%%%%%%%%

\section{Auswertung}

%%%%%%%%%%%%%%%%%%%%%%%%%%%%%%%%%%%%%%%%%%%%%%%%%%%%%%%%%%%%%%%%%%%%%%%%%%%%%%%
%                                  Ergebnis                                   %
%%%%%%%%%%%%%%%%%%%%%%%%%%%%%%%%%%%%%%%%%%%%%%%%%%%%%%%%%%%%%%%%%%%%%%%%%%%%%%%

\section{Ergebnis}

\IfFileExists{\bibliographyfile}{
	\bibliography{\bibliographyfile}
}{}

\end{document}

% vim: spell spelllang=de
