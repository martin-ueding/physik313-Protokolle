% Copyright © 2013 Martin Ueding <dev@martin-ueding.de>

\input{../header.tex}

\usepackage{placeins}
\usepackage{minted}

\ihead{physik313 – Versuch 8}
\ifoot{Lino Lemmer, Martin Ueding}

\subject{Praktikumsprotokoll}
\title{Mikroprozessor}
\subtitle{physik313 – Versuch 8}
\author{
	Lino Lemmer
    \footnote{\href{mailto:s6lilemm@uni-bonn.de}{s6lilemm@uni-bonn.de}}
	\and
	Martin Ueding
    \footnote{\href{mailto:mu@martin-ueding.de}{mu@martin-ueding.de}}
}

%\setcounter{tocdepth}{2}

\newcommand\fT{f_\text{T}}
\newcommand\IB{I_\text{B}}
\newcommand\IC{I_\text{C}}
\newcommand\ID{I_\text{D}}
\newcommand\IE{I_\text{E}}
\newcommand\IS{I_\text{S}}
\newcommand\RC{R_\text{C}}
\newcommand\RD{R_\text{D}}
\newcommand\RE{R_\text{E}}
\newcommand\UBE{U_\text{BE}}
\newcommand\UB{U_\text{B}}
\newcommand\UCE{U_\text{CE}}
\newcommand\UC{U_\text{C}}
\newcommand\UD{U_\text{D}}
\newcommand\UDS{U_\text{DS}}
\newcommand\UE{U_\text{E}}
\newcommand\UGS{U_\text{GS}}
\newcommand\UG{U_\text{G}}
\newcommand\Uin{U_\text{in}}
\newcommand\Uout{U_\text{out}}

\newcommand\UEH{U_\text{E H}}
\newcommand\UEL{U_\text{E L}}
\newcommand\UH{U_\text{H}}
\newcommand\UL{U_\text{L}}
\newcommand\UQH{U_\text{Q H}}
\newcommand\UQL{U_\text{Q L}}

\newcommand\mand {\wedge}
\newcommand\mhigh{\top}
\newcommand\mlow {\bot}
\newcommand\mnand{\bar\wedge}
\newcommand\mnor {\bar\vee}
\newcommand\mnot {\neg}
\newcommand\mor  {\vee}
\newcommand\mxor {\veebar}
\newcommand\tand {\textsc{and}}
\newcommand\thigh{\textsc{high}}
\newcommand\tlow {\textsc{low}}
\newcommand\tnand{\textsc{nand}}
\newcommand\tnor {\textsc{nor}}
\newcommand\tnot {\textsc{not}}
\newcommand\tor  {\textsc{or}}
\newcommand\txor {\textsc{xor}}

\begin{document}

\maketitle

Der \LaTeX-Quelltext zu allen Protokollen in diesem Praktikum kann auf
\ref{it:mu} eingesehen werden. Die Quellen für dieses Protokoll können auf
\ref{it:github/alles} eingesehen werden. Die \LaTeX-Datei wird aus
\ref{it:github/template} generiert.

\begin{enumerate}
	\item
		\label{it:mu}
		\url{http://martin-ueding.de/de/university/physik313/}
	\item
		\label{it:github/alles}
		\url{https://github.com/martin-ueding/physik313-8/}
	\item
		\label{it:github/template}
		\url{https://github.com/martin-ueding/physik313-8/blob/master/Template.tex}
\end{enumerate}

\tableofcontents
\newpage

%%%%%%%%%%%%%%%%%%%%%%%%%%%%%%%%%%%%%%%%%%%%%%%%%%%%%%%%%%%%%%%%%%%%%%%%%%%%%%%
%                                 Einleitung                                  %
%%%%%%%%%%%%%%%%%%%%%%%%%%%%%%%%%%%%%%%%%%%%%%%%%%%%%%%%%%%%%%%%%%%%%%%%%%%%%%%

\FloatBarrier
\section{Einleitung}

%%%%%%%%%%%%%%%%%%%%%%%%%%%%%%%%%%%%%%%%%%%%%%%%%%%%%%%%%%%%%%%%%%%%%%%%%%%%%%%
%                                  Theorie                                    %
%%%%%%%%%%%%%%%%%%%%%%%%%%%%%%%%%%%%%%%%%%%%%%%%%%%%%%%%%%%%%%%%%%%%%%%%%%%%%%%

\FloatBarrier
\section{Theorie}

%%%%%%%%%%%%%%%%%%%%%%%%%%%%%%%%%%%%%%%%%%%%%%%%%%%%%%%%%%%%%%%%%%%%%%%%%%%%%%%
%                                  Aufgaben                                   %
%%%%%%%%%%%%%%%%%%%%%%%%%%%%%%%%%%%%%%%%%%%%%%%%%%%%%%%%%%%%%%%%%%%%%%%%%%%%%%%

\FloatBarrier
\section{Aufgaben}

\subsection{Aufgabe A}

\begin{problem}
	Wandeln Sie die zwei nachfolgenden Dualzahlen in das Hexadezimal- und das
	Dezimalsystem um:
	\begin{gather*}
		1101\,1111\,0010\,1110_2 \\
		1111\,1111_2
	\end{gather*}
\end{problem}

An dieser Stelle möchte ich mich als guter Programmierer meiner Faulheit
bedienen (Aussage von Larry Wall, \cite{threevirtues.com}). Daher benutze ich
Python 3, um die Zahlen zu verrechnen. Mir ist durchaus bekannt, wie man die
Zahlen umrechnet, jedoch möchte ich meine Zeit lieber für die interessanten
Teile dieses Versuchs aufwenden.

\begin{minted}{pycon}
>>> 0b1101111100101110
57134
\end{minted}

\subsection{Aufgabe B}

\begin{problem}
	Wandeln Sie die nachfolgenden Dezimalzahlen in Binär- und Hexadezimalzahlen
	um:
	\[
		2115_{10}
	\]
\end{problem}

\begin{minted}{pycon}
>>> bin(2115)
'0b100001000011'
\end{minted}

\subsection{Aufgabe C}

\begin{problem}
	Wandeln Sie die nachfolgenden Hexadezimalzahl in Binär- und
	Dezimaldarstellung um:
	\[
		\mathrm{B75F}_{16}
	\]
\end{problem}

\begin{minted}{pycon}
>>> 0xb75f
46943
>>> bin(_)
'0b1011011101011111'
\end{minted}

\subsection{Aufgabe D}

\begin{problem}
	Führen Sie die nachfolgenden Operationen zwischen Dualzahlen durch:
\end{problem}

\paragraph{Addition}

\begin{minted}{pycon}
>>> bin(0b01011011 + 0b01101011)
'0b11000110'
>>> bin(0b11111111 + 0b00000001)
'0b100000000'
\end{minted}

\paragraph{Subtraktion}

\begin{minted}{pycon}
>>> bin(0b11000000 - 0b10110101)
'0b1011'
\end{minted}

\paragraph{Multiplikation}

\begin{minted}{pycon}
>>> bin(0b1101 * 0b1001)
'0b1110101'
\end{minted}

\paragraph{Division}

Hier entsteht offensichtlich ein Rest, so dass ganzzahlige Division benutzt
werden muss:

\begin{minted}{pycon}
>>> bin(0b1110111 / 0b101)
Traceback (most recent call last):
  File "<stdin>", line 1, in <module>
TypeError: 'float' object cannot be interpreted as an integer
>>> 0b1110111 / 0b101
23.8
>>> bin(0b1110111 // 0b101)
'0b10111'
\end{minted}

\subsection{Aufgabe E}

\begin{problem}
	Erklären Sie den Unterschied von ROM und RAM. Wo liegen die Vorteile der
	beiden Typen?
\end{problem}

ROM ist nur lesbar, meistens langsam jedoch ohne Stromversorgung stabil. RAM
ist auch schreibbar, sehr schnell, jedoch gehen ohne kontinuierliche
Stromversorgung die Daten verloren.

\subsection{Aufgabe F}

\begin{problem}
	Kann man mit Digitalrechnern analoge Signale verarbeiten? Was brauchen Sie
	hierfür? Wodurch wird die Genaugikeit begrenzt?
\end{problem}

Man braucht ADC, die jedoch bei $b$ Bytes pro Messwert nur $2^b$ Stufen
abbilden können.

%%%%%%%%%%%%%%%%%%%%%%%%%%%%%%%%%%%%%%%%%%%%%%%%%%%%%%%%%%%%%%%%%%%%%%%%%%%%%%%
%                                  Literatur                                  %
%%%%%%%%%%%%%%%%%%%%%%%%%%%%%%%%%%%%%%%%%%%%%%%%%%%%%%%%%%%%%%%%%%%%%%%%%%%%%%%

\FloatBarrier
\IfFileExists{\bibliographyfile}{
	\bibliography{\bibliographyfile}
}{}

\end{document}

% vim: spell spelllang=de tw=79
