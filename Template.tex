% Copyright © 2013 Martin Ueding <dev@martin-ueding.de>

\input{header.tex}

\usepackage{placeins}

\ihead{physik313 – Versuch 3/4}
\ifoot{Lino Lemmer}

\hypersetup{
	pdftitle={Transistor und Transistorverstärker}
}

\subject{Praktikumsprotokoll}
\title{Transistor und Transistorverstärker}
\subtitle{physik313 – Versuch 3/4}
\author{
	Lino Lemmer \footnote{\href{mailto:s6lilemm@uni-bonn.de}{s6lilemm@uni-bonn.de}}
}

%\setcounter{tocdepth}{2}

\begin{document}

\maketitle

Der \LaTeX-Quelltext zu allen Protokollen in diesem Praktikum kann auf
\ref{it:mu} eingesehen werden. Die Quellen für dieses Protokoll können auf
\ref{it:github/alles} eingesehen werden. Die \LaTeX-Datei wird aus
\ref{it:github/template} generiert.

\begin{enumerate}
	\item
		\label{it:mu}
		\url{http://martin-ueding.de/de/university/physik313/}
	\item
		\label{it:github/alles}
		\url{https://github.com/martin-ueding/physik313-3,4/}
	\item
		\label{it:github/template}
		\url{https://github.com/martin-ueding/physik313-3,4/blob/master/Template.tex}
\end{enumerate}

\newpage
\tableofcontents
\newpage

%%%%%%%%%%%%%%%%%%%%%%%%%%%%%%%%%%%%%%%%%%%%%%%%%%%%%%%%%%%%%%%%%%%%%%%%%%%%%%%
%                                 Einleitung                                  %
%%%%%%%%%%%%%%%%%%%%%%%%%%%%%%%%%%%%%%%%%%%%%%%%%%%%%%%%%%%%%%%%%%%%%%%%%%%%%%%

\FloatBarrier
\section{Einleitung}

%%%%%%%%%%%%%%%%%%%%%%%%%%%%%%%%%%%%%%%%%%%%%%%%%%%%%%%%%%%%%%%%%%%%%%%%%%%%%%%
%                                  Aufgaben                                   %
%%%%%%%%%%%%%%%%%%%%%%%%%%%%%%%%%%%%%%%%%%%%%%%%%%%%%%%%%%%%%%%%%%%%%%%%%%%%%%%

\FloatBarrier
\section{Aufgaben}

\subsection{Aufgabe A}

\begin{problem}
	Welche Ströme treten beim Transistor außer dem Basis-Emitter-Durchlassstrom
	und dem Emitter-Kollektor-Strom auf?
\end{problem}

\fehlt

\subsection{Aufgabe B}

\begin{problem}
	Wie sieht der Potentialverlauf im npn-Transistor aus
	\begin{enumerate}
		\item
			ohne äußere Spannung
		\item
			bei außen angelegter Spannung?
	\end{enumerate}
\end{problem}

\fehlt

\subsection{Aufgabe C}

\begin{problem}
	Wie sehen die Ladungsträgerkonzentrationen für Löcher und Elektronen im
	npn-Transistor aus?
\end{problem}

\fehlt

\subsection{Aufgabe D}

\begin{problem}
	Verifizieren Sie die Relationen zwischen $\alpha$, $\beta$ und $\gamma$.
\end{problem}

\fehlt

\FloatBarrier
\subsection{Aufgabe E}

\begin{problem}
	Welchen Ausgangsspannunsbereich ($U_\text{out min}, U_\text{out max}$)
	(Aussteuerbereich) hat die Schaltung in Abbildung~\ref{fig:3_4-5}?
	Vernachlässigen Sie hier $U_\text{CE sat}$.
\end{problem}

\begin{figure}[htbp]
	\centering
	\includegraphics[width=.4\textwidth]{Anleitung/3_4-5.png}
	\caption{%
		\cite[Abbildung~3/4.5]{physik313-Anleitung}
	}
	\label{fig:3_4-5}
\end{figure}

\fehlt

\FloatBarrier
\subsection{Aufgabe F}

\begin{problem}
	Welche Form hat die \emph{Eingangskennlinie} eines Transistors in
	Emitterschaltung ($I_\text B$ als Funktion von $I_\text{BE}$)?
\end{problem}

\fehlt

\subsection{Aufgabe G}

\begin{problem}
	Leiten Sie \eqref{eq:3_4-7} her!
\end{problem}

\fehlt

\subsection{Aufgabe H}

\begin{problem}
	Was passiert, wenn man den Spannungsteiler zu niederohmig macht?
\end{problem}

\fehlt

\subsection{Aufgabe I}

\begin{problem}
	Wie sieht die entsprechende Kennlinie beim bipolaren Transistor aus?
	Welcher Spannung entspricht dort $U_\text{thr}$?
\end{problem}

\fehlt

\subsection{Aufgabe J}

\begin{problem}
	Was ändert sich, wenn man $I_\text S$ anstelle von $I_\text D$ aufträgt?
\end{problem}

\fehlt

\subsection{Aufgabe K}

\begin{problem}
	Zeigen Sie, dass genauer gilt:
	\begin{equation}
		\label{eq:3_4-11}
		v = \frac{\gamma R_\text E}{r_\text{BE} + \gamma R_\text E},
	\end{equation}
	wobei der differentielle Widerstand der Emitter-Basis-Diode $r_\text{BE} =
	\mathrm d U_\text{BE} / \mathrm d I_\text B$ ist.
\end{problem}

\fehlt

\subsection{Aufgabe L}

\begin{problem}
	Welchen Zweck könnte der Kollektorwiderstand $R_\text C$ beim Emitterfolger
	haben? Hinweis: Am Ausgang könnte eine niederohmige Last angeschlossen
	sein.
\end{problem}

\fehlt

\subsection{Aufgabe M}

\begin{problem}
	Beweisen Sie \eqref{eq:3_4-12}.
\end{problem}

Die zitierte Gleichung ist:
\begin{equation}
	\label{eq:3_4-12}
	\frac{r_\text{out}}{r_\text{in}}
	= \frac{\gamma R_\text E}{r_\text{BE} + \gamma R_\text E}
	\approx \frac 1\gamma
\end{equation}

\fehlt

\subsection{Aufgabe N}

\begin{problem}
	Wie groß ist der Eingangswiderstand des unbelasteten Emitterfolgers?
\end{problem}

\fehlt

\subsection{Aufgabe O}

\begin{problem}
	Zeigen Sie, dass genauer gilt:
	\begin{equation}
		\label{eq:3_4-14}
		v = - \frac{\beta R_\text C}{r_\text{BE} + \gamma R_\text E}
	\end{equation}
\end{problem}

\fehlt

\subsection{Aufgabe P}

\begin{problem}
	Beweisen Sie \eqref{eq:3_4-17}.
\end{problem}

Die zitierte Gleichung ist:
\begin{equation}
	\label{eq:3_4-17}
	\frac{\mathrm d v} v
	= \frac{\mathrm d v_0}{v_0} \frac{1}{k v_0 + 1}
	= \frac{\mathrm d v_0}{v_0} \frac{v}{v_0}
\end{equation}

\fehlt

\subsection{Aufgabe Q}

\begin{problem}
	Erklären sie, wieso die Kapazität $C_\text{CB}$ Einfluss auf die
	Verstärkung hat.
\end{problem}

\fehlt

\FloatBarrier
\subsection{Aufgabe R}

\begin{problem}
	Erklären Sie die Funktionsweise der Schaltung in
	Abbildung~\ref{fig:3_4-15}! Wie groß ist die Spannungsänderung im Punkt P
	bei einer Stromänderung $\mathrm d I_\text E(T2)$ und welche
	Transistorgröße bestimmt diesen Wert?
\end{problem}

\begin{figure}[htbp]
	\centering
	\includegraphics[width=.6\textwidth]{Anleitung/3_4-15.png}
	\caption{%
		\cite[Abbildung~3/4.15]{physik313-Anleitung}
	}
	\label{fig:3_4-15}
\end{figure}

\fehlt

\FloatBarrier
\subsection{Aufgabe S}

\begin{problem}
	Leiten Sie \eqref{eq:3_4-18} her. Hinweise: Da die Gegenkopplung bei
	Betrachtung im Frequenzraum auf der Addition von Sinusschwingungen beruht
	und wie in diesem Kapitel die Phasen ignoriert haben, ist hier
	$v(f_\text{grenz gk}) = 2v(f=0)$ statt korrekterweise $\sqrt 2 v (f = 0)$.
\end{problem}

Die zitierte Gleichung ist:
\begin{equation}
	\label{eq:3_4-18}
	f_\text{grenz gk} = f_\text{grenz} \frac{v_0}{v(f=0)}
\end{equation}

\fehlt

\FloatBarrier
\subsection{Aufgabe T}

\begin{problem}
	Erläutern Sie die Wirkungsweise der Art der Stabilisierung des
	Basispotentials durch den Widerstand $R$ in Abbildung~\ref{fig:3_4-16}.
	Überlegen Sie dazu, was passiert, wenn das Basispotential aus irgend einem
	(äußeren) Grund „wegläuft“!
\end{problem}

\begin{figure}[htbp]
	\centering
	\includegraphics[width=.6\textwidth]{Anleitung/3_4-16.png}
	\caption{%
		\cite[Abbildung~3/4.16]{physik313-Anleitung}
	}
	\label{fig:3_4-16}
\end{figure}

\fehlt

%%%%%%%%%%%%%%%%%%%%%%%%%%%%%%%%%%%%%%%%%%%%%%%%%%%%%%%%%%%%%%%%%%%%%%%%%%%%%%%
%                   Durchführung: Transistoreigenschaften                    %
%%%%%%%%%%%%%%%%%%%%%%%%%%%%%%%%%%%%%%%%%%%%%%%%%%%%%%%%%%%%%%%%%%%%%%%%%%%%%%%

\FloatBarrier
\section{Durchführung Tag 2: Transistoreigenschaften}

\FloatBarrier
\subsection{Kennlinien und Arbeitspunkt}

\subsubsection{Kennlinienschreiber}

\fehlt

Das Schaltbild des Kennlinienschreibers ist in Abbildung~\ref{fig:3-1} dargestellt.

\begin{figure}[htbp]
	\centering
	\includegraphics[width=\textwidth]{Anleitung/3-1.png}
	\caption{
		\cite[Abbildung~3.1]{physik313-Anleitung}
	}
	\label{fig:3-1}
\end{figure}

\fehlt

\subsubsection{Inbetriebnahme des Kennlinienschreibers}

\begin{figure}[htbp]
	\centering
	\includegraphics[width=\textwidth]{Anleitung/3-2.png}
	\caption{
		\cite[Abbildung~3.2]{physik313-Anleitung}
	}
	\label{fig:3-2}
\end{figure}

\fehlt

\subsubsection{Bipolarer Transistor}

\fehlt

\subsubsection{FET}

\fehlt

\subsection{Emitterfolger}

\subsubsection{Aufbau}

\fehlt

\begin{figure}[htbp]
	\centering
	\includegraphics[width=\textwidth]{Anleitung/3-4.png}
	\caption{
		\cite[Abbildung~3.4]{physik313-Anleitung}
	}
	\label{fig:3-4}
\end{figure}

\subsubsection{Spannungsverstärkung}

\fehlt

\subsubsection{Aussteuergrenzen}

\fehlt

\subsection{FET}

\begin{figure}[htbp]
	\centering
	\includegraphics[width=\textwidth]{Anleitung/3-5.png}
	\caption{
		\cite[Abbildung~3.5]{physik313-Anleitung}
	}
	\label{fig:3-5}
\end{figure}

\fehlt

\subsubsection{Eingangswiderstand}

\fehlt

%%%%%%%%%%%%%%%%%%%%%%%%%%%%%%%%%%%%%%%%%%%%%%%%%%%%%%%%%%%%%%%%%%%%%%%%%%%%%%%
%                    Durchführung: Transistorverstärker                     %
%%%%%%%%%%%%%%%%%%%%%%%%%%%%%%%%%%%%%%%%%%%%%%%%%%%%%%%%%%%%%%%%%%%%%%%%%%%%%%%

\FloatBarrier
\section{Durchführung Tag 2: Transistorverstärker}

\subsection{Fortsetung Emitterfolger}

\subsubsection{Spannungsverstärkung des Emitterfolgers}

\subsubsection{Emitterfolger als Impedanzwandler}

\subsection{Invertierender Transistorverstärker (Emitterschaltung)}

\subsubsection{Phasenbeziehung zwischen Ein- und Ausgang}

\subsubsection{Spannungsverstärkung des inverteierenden Verstärkers}

\subsubsection{Bestimmung des Transistoreingangswiderstands}

\subsection{Wechselstrommäßige Aufhebung der Gegenkopplung}

\subsection{Frequenzverhalten und Kaskodenschaltung}

\subsection{Verstärker mit Spannungsgegenkopplung}

\begin{figure}[htbp]
	\centering
	\includegraphics[width=.6\textwidth]{Anleitung/4-1.png}
	\caption{
		\cite[Abbildung~4.1]{physik313-Anleitung}
	}
	\label{fig:4-1}
\end{figure}

%%%%%%%%%%%%%%%%%%%%%%%%%%%%%%%%%%%%%%%%%%%%%%%%%%%%%%%%%%%%%%%%%%%%%%%%%%%%%%%
%                                  Literatur                                  %
%%%%%%%%%%%%%%%%%%%%%%%%%%%%%%%%%%%%%%%%%%%%%%%%%%%%%%%%%%%%%%%%%%%%%%%%%%%%%%%

\FloatBarrier
\IfFileExists{\bibliographyfile}{
	\bibliography{\bibliographyfile}
}{}

\end{document}

% vim: spell spelllang=de
