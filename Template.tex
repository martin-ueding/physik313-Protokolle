% Copyright © 2013 Martin Ueding <dev@martin-ueding.de>

\input{header.tex}

\hypersetup{
	pdftitle={Einführung und Vorversuch}
}

\subject{Praktikumsprotokoll}
\title{Einführung und Vorversuch}
\subtitle{physik313 – Versuch 0}
\author{
	Martin Ueding \footnote{\href{mailto:mu@martin-ueding.de}{mu@martin-ueding.de}}
	\and
	Lino Lemmer \footnote{\href{mailto:s6lilemm@uni-bonn.de}{s6lilemm@uni-bonn.de}}
}

\begin{document}

\maketitle

%\newpage
%\tableofcontents
%\newpage

\section{Einleitung}

\section{Theorie}

\section{Aufgaben}

\subsection{Aufgabe A}

\begin{align*}
	U_\text{SS}=&2U_0\\
	U_\text{S}=&U_0\\
	U_\text{eff}=&\frac{U_0}{\sqrt{2}}
\end{align*}

\subsection{Aufgabe B}

\[U_\text{S}=\SI{10}{V}\]

\subsection{Aufgabe C}

Aus
\begin{align}
	U_1=&U_0\frac{R_1}{R_1+R_i}\notag
	\intertext{folgt}
	R_i=&\frac{\del{U_0-U_1}R_1}{U_1}\notag\\
	=&\frac{U_0-U_1}{I_1}\label{eq:R_i}
	\intertext{und aus}
	U_2=&U_0\frac{R_2}{R_2+R_i}\notag
	\intertext{folgt}
	U_0=&\frac{\del{R_2+R_i}U_2}{R_2}\notag\\
	=&\del{R_2+R_i}I_2\label{eq:U_0}
	\intertext{Setzt man nun \eqref{eq:U_0} ind \eqref{eq:R_i} ein erhält man}
	R_i=&\frac{\del{R_2+R_i}I_2}{I_1}-\frac{U_1}{I_1}\notag\\
	\iff\qquad \del{\frac{I_1-I_2}{I_1}}R_i=&\frac{U_2-U_1}{I_1}\notag\\
	\iff\qquad R_i=&\frac{U_2-U_1}{I_1-U_2}\notag
\end{align}
Dies ist die gesuchte Gleichung.

\section{Versuchsaufbau und -durchführung}

\section{Messung}

\section{Auswertung}

\section{Ergebnis}

%\IfFileExists{\bibliographyfile}{
%	\bibliography{\bibliographyfile}
%}{}

\end{document}

% vim: spell spelllang=de
