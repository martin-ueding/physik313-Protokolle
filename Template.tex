% Copyright © 2013 Martin Ueding <dev@martin-ueding.de>

\input{../header.tex}

\usepackage{placeins}

\ihead{physik313 – Versuch 7}
\ifoot{Lino Lemmer, Martin Ueding}

\subject{Praktikumsprotokoll}
\title{Logische Schaltungen}
\subtitle{physik313 – Versuch 7}
\author{
	Lino Lemmer
    \footnote{\href{mailto:s6lilemm@uni-bonn.de}{s6lilemm@uni-bonn.de}}
	\and
	Martin Ueding
    \footnote{\href{mailto:mu@martin-ueding.de}{mu@martin-ueding.de}}
}

%\setcounter{tocdepth}{2}

\newcommand\fT{f_\text{T}}
\newcommand\IB{I_\text{B}}
\newcommand\IC{I_\text{C}}
\newcommand\ID{I_\text{D}}
\newcommand\IE{I_\text{E}}
\newcommand\IS{I_\text{S}}
\newcommand\RC{R_\text{C}}
\newcommand\RD{R_\text{D}}
\newcommand\RE{R_\text{E}}
\newcommand\UBE{U_\text{BE}}
\newcommand\UB{U_\text{B}}
\newcommand\UCE{U_\text{CE}}
\newcommand\UC{U_\text{C}}
\newcommand\UD{U_\text{D}}
\newcommand\UDS{U_\text{DS}}
\newcommand\UE{U_\text{E}}
\newcommand\UGS{U_\text{GS}}
\newcommand\UG{U_\text{G}}
\newcommand\Uin{U_\text{in}}
\newcommand\Uout{U_\text{out}}

\newcommand\UEH{U_\text{E H}}
\newcommand\UEL{U_\text{E L}}
\newcommand\UH{U_\text{H}}
\newcommand\UL{U_\text{L}}
\newcommand\UQH{U_\text{Q H}}
\newcommand\UQL{U_\text{Q L}}

\newcommand\mand {\wedge}
\newcommand\mhigh{\top}
\newcommand\mlow {\bot}
\newcommand\mnand{\bar\wedge}
\newcommand\mnor {\bar\vee}
\newcommand\mnot {\neg}
\newcommand\mor  {\vee}
\newcommand\mxor {\veebar}
\newcommand\tand {\textsc{and}}
\newcommand\thigh{\textsc{high}}
\newcommand\tlow {\textsc{low}}
\newcommand\tnand{\textsc{nand}}
\newcommand\tnor {\textsc{nor}}
\newcommand\tnot {\textsc{not}}
\newcommand\tor  {\textsc{or}}
\newcommand\txor {\textsc{xor}}

\begin{document}

\maketitle

\[
	\mand
	\mhigh
	\mlow
	\mnand
	\mnor
	\mnot
	\mor
	\mxor
\]

Der \LaTeX-Quelltext zu allen Protokollen in diesem Praktikum kann auf
\ref{it:mu} eingesehen werden. Die Quellen für dieses Protokoll können auf
\ref{it:github/alles} eingesehen werden. Die \LaTeX-Datei wird aus
\ref{it:github/template} generiert.

\begin{enumerate}
	\item
		\label{it:mu}
		\url{http://martin-ueding.de/de/university/physik313/}
	\item
		\label{it:github/alles}
		\url{https://github.com/martin-ueding/physik313-7/}
	\item
		\label{it:github/template}
		\url{https://github.com/martin-ueding/physik313-7/blob/master/Template.tex}
\end{enumerate}

\tableofcontents
\newpage

%%%%%%%%%%%%%%%%%%%%%%%%%%%%%%%%%%%%%%%%%%%%%%%%%%%%%%%%%%%%%%%%%%%%%%%%%%%%%%%
%                                 Einleitung                                  %
%%%%%%%%%%%%%%%%%%%%%%%%%%%%%%%%%%%%%%%%%%%%%%%%%%%%%%%%%%%%%%%%%%%%%%%%%%%%%%%

\FloatBarrier
\section{Einleitung}

%%%%%%%%%%%%%%%%%%%%%%%%%%%%%%%%%%%%%%%%%%%%%%%%%%%%%%%%%%%%%%%%%%%%%%%%%%%%%%%
%                                  Theorie                                    %
%%%%%%%%%%%%%%%%%%%%%%%%%%%%%%%%%%%%%%%%%%%%%%%%%%%%%%%%%%%%%%%%%%%%%%%%%%%%%%%

\FloatBarrier
\section{Theorie}

%%%%%%%%%%%%%%%%%%%%%%%%%%%%%%%%%%%%%%%%%%%%%%%%%%%%%%%%%%%%%%%%%%%%%%%%%%%%%%%
%                                  Aufgaben                                   %
%%%%%%%%%%%%%%%%%%%%%%%%%%%%%%%%%%%%%%%%%%%%%%%%%%%%%%%%%%%%%%%%%%%%%%%%%%%%%%%

\FloatBarrier
\section{Aufgaben}

\FloatBarrier
\subsection{Aufgabe K}

\begin{problem}
	Wechle logische Funktion wir[d] durch diese Schaltung [in
	Abbildung~\ref{fig:7-7}] realisiert? Welche Aufgabe habben die Dioden?
	Überprüfen Sie, ob auch hier noch die Ausgangspegel korrekt sind.
\end{problem}

\begin{figure}[htbp]
	\centering
	\includegraphics[width=.5\linewidth]{../Anleitung/7-7.png}
	\caption{%
		\cite[Abbildung~7.7]{physik313-Anleitung}
	}
	\label{fig:7-7}
\end{figure}

Die die logische Funktion des Schaltung ist eine \tnor-Funktion.

Die Dioden
haben die Funktion die einkommenden Spannungen nur an das Gatter
weiterzuleiten. Die Eingangsspannung soll von beiden Eingängen auf $\UEH$
gezogen werden können, jedoch sollen die Eingänge voneinander entkoppelt sein,
sie sollen ja Eingänge sein.

\paragraph{Spannungslevel}

Wenn nur $\UL$ auf beiden Eingängen ist, liegt eine maximale Spannung von
\SI{.2}{\volt} an, da die Siliziumdioden schon \SI{.6}{\volt} verschlingen.
Dies reicht nicht mehr aus, um den Transistor zu schalten, am Ausgang werden
fast die vollen \SI{5}{\volt} anliegen.

Wenn einer der der Eingänge auf $\UL$ geschaltet ist, dann mindestens
\SI{2.4}{\volt}. Nach der Diode sind immer noch \SI{1.8}{\volt} da. Dies sollte
reichen, um den Transistor zu schalten. $\UBE$ ist dann nur noch
\SI{.7}{\volt}, so dass folgender Strom fließt:
\[
	\IB = \frac{\SI{1.8}\volt - 2 \cdot \SI{.7}\volt}{\SI1{\kilo\ohm}}
	= \SI{.4}{\milli\ampere}
\]

Mit einer Stromverstärkung von $\beta = 100$ sind dies $\IB =
\SI{40}{\milli\ampere}$. Dabei würde an $\RC$ eine Spannung von \SI{40}{\volt}
abfallen. Da nur \SI{5}{\volt} anliegen, wird sicher ein \thigh{} am Ausgang
anliegen.

\FloatBarrier
\subsection{Aufgabe L}

\fehlt

\FloatBarrier
\subsection{Aufgabe M}

\begin{problem}
	Wie funktioniert der CMOS-Inverter (Abbildung~\ref{fig:7-9})? Benutzen Sie
	die angegebenen Kennlinien.
\end{problem}

\begin{figure}[htbp]
	\centering
	\includegraphics[width=.3\linewidth]{../Anleitung/7-9.png}
	\caption{%
		\cite[Abbildung~7.9]{physik313-Anleitung}
	}
	\label{fig:7-9}
\end{figure}

Wenn eine Eingangsspannung anliegt, also ein \thigh, sperrt der obere FET, der
untere FET wird durchgeschaltet. Somit fällt die Betriebsspannung am oberen FET
ab, Ausgang ist ein \tlow.

Wenn keine Eingangsspannung anliegt, also ein \tlow, wird der Obere
durchgeschaltet, der Untere gesperrt. Damit fällt die Betriebsspannung am
unteren ab, Ausgang ist ein \thigh.

Die Kennlinien sind so extrem, dass sich hier ein digitales Verhalten zeigt.

\FloatBarrier
\subsection{Aufgabe N}

\begin{problem}
	Welche logische Funktion ist mit dem Gatter in Abbildung~\ref{fig:7-10}
	realisiert?
\end{problem}

\begin{figure}[htbp]
	\centering
	\includegraphics[width=.5\linewidth]{../Anleitung/7-10.png}
	\caption{%
		\cite[Abbildung~7.10]{physik313-Anleitung}
	}
	\label{fig:7-10}
\end{figure}

Legen wir doch eine Tabelle an. Dazu gehen wir Schritt für Schritt durch:

\paragraph{A und B \tlow}

n-MOS links und rechts sind durchgeschaltet. Daher fällt über ihnen keine
Spannung ab. Die p-MOS oben und unten sind gesperrt, daher fällt über ihnen
Spannung ab. Ausgang ist \thigh.

\paragraph{A und B \thigh}

Genau invers, also fällt die Spannung über den ersten MOSFETs ab, womit der
Ausgang \tlow ist.

\paragraph{A \thigh{} und B \tlow}

n-MOS links ist offen, n-MOS rechts ist gesperrt. p-MOS oben ist offen, p-MOS
unten ist gesperrt.

Somit fällt viel Spannung unten ab, oben ist ein Weg offen. Ausgang ist \thigh.

\paragraph{A \tlow{} und B \thigh}

n-MOS links ist gesperrt, rechts ist offen. p-MOS oben ist gesperrt, unten ist
offen.

Somit wie oben, \thigh

\paragraph{Zusammenfassung}

Wir erhalten:

\begin{tabular}{c|cc}
	A \textbackslash{} B & 0 & 1 \\
	\hline
	0 & 1 & 1 \\
	1 & 1 & 0
\end{tabular}

Somit handelt es sich um ein \tnand-Gatter.

\FloatBarrier
\subsection{Aufgabe O}

\begin{problem}
	Wie lauten die Bool'schen Ausdrücke für Summe und Übertrag eines
	Halbaddierers?
\end{problem}

Einen Übertrag gibt es nur, wenn beide Summanden \thigh{} sind. Somit ist der
Übertrag:
\[
	U := a \mand b
\]

Die Summe ist ein \txor. Also:
\[
	S := a \mxor b
\]

\FloatBarrier
\subsection{Aufgabe P}

\begin{problem}
	Schreiben Sie die Funktionstafel auf für einen Volladdierer, der auch den
	Übertrag des vorhergehenden Bits mit verarbeitet.
\end{problem}

Der Volladdierer bekommt nicht nur die beiden Stellen $a$ und $b$ geliefert,
sondern auch $\Uin$, den Übertrag der vorherigen Rechnung. Der Volladdierer
addiert diese drei Zahlen auf. Da sie im Bereich 0 bis 3 liegen, braucht er
zwei Stellen als Ausgabe. Das ist einmal Summe und $\Uout$, der Übertrag des
Ergebnisses.

\begin{tabular}{ccc|cc}
	$a$ & $b$ & $\Uin$ & $\Uout$ & $S$ \\
	\hline
	0 & 0 & 0 & 0 & 0 \\
	0 & 0 & 1 & 0 & 1 \\
	0 & 1 & 0 & 0 & 1 \\
	0 & 1 & 1 & 1 & 0 \\
	1 & 0 & 0 & 0 & 1 \\
	1 & 0 & 1 & 1 & 0 \\
	1 & 1 & 0 & 1 & 0 \\
	1 & 1 & 1 & 1 & 1 \\
\end{tabular}

\begin{small}
	Für die Bearbeitung dieser Aufgabe habe ich mir
	\cite{wikipedia/Volladdierer} angeschaut. Die Tabelle habe ich jedoch nicht
	kopiert, sondern aus der Erklärung im ersten Absatz des Artikels selbst
	erstellt.
\end{small}

\FloatBarrier
\subsection{Aufgabe Q}

\begin{problem}
	Geben Sie ein Blockschaltbild an für einen Volladdierer, der aus zwei
	Halbaddierern zusammengesetzt ist.
\end{problem}

Der Volladdierer soll, wie in der vorherigen Aufgabe beschrieben, drei Zahlen
addieren. Dazu werden erst einmal $a$ und $b$ addiert, zur Zwischensumme $Z$
und ihrem Übertrag $U_Z$. Anschließend werden werden der vorherige Übertrag
$\Uin$ mit der Zwischensumme addiert. Als Ergebnis kommen die Summe $S$ und ein
weiterer Übertrag $U_S$ heraus. Die beiden Überträge können jedoch nicht beide
\thigh{} sein. Somit reicht es, sie mit einem \tor-Gatter zu verknüpfen.

Dies ist in Abbildung~\ref{fig:Q-Volladdierer} gezeigt.

\begin{figure}[htbp]
	\centering
	\includegraphics{../Zeichnungen/Q-Volladdierer.pdf}
	\caption{%
		Blockschaltbild für den Volladdierer. Mit Denkanstoß von
		\cite{wikipedia/Volladdierer}.
	}
	\label{fig:Q-Volladdierer}
\end{figure}

%%%%%%%%%%%%%%%%%%%%%%%%%%%%%%%%%%%%%%%%%%%%%%%%%%%%%%%%%%%%%%%%%%%%%%%%%%%%%%%
%                                  Literatur                                  %
%%%%%%%%%%%%%%%%%%%%%%%%%%%%%%%%%%%%%%%%%%%%%%%%%%%%%%%%%%%%%%%%%%%%%%%%%%%%%%%

\FloatBarrier
\IfFileExists{\bibliographyfile}{
	\bibliography{\bibliographyfile}
}{}

\end{document}

% vim: spell spelllang=de tw=79
