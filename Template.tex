% Copyright © 2013 Martin Ueding <dev@martin-ueding.de>

\input{../header.tex}

\usepackage{placeins}

\ihead{physik313 – Versuch 7}
\ifoot{Lino Lemmer, Martin Ueding}

\subject{Praktikumsprotokoll}
\title{Logische Schaltungen}
\subtitle{physik313 – Versuch 7}
\author{
	Lino Lemmer
    \footnote{\href{mailto:s6lilemm@uni-bonn.de}{s6lilemm@uni-bonn.de}}
	\and
	Martin Ueding
    \footnote{\href{mailto:mu@martin-ueding.de}{mu@martin-ueding.de}}
}

%\setcounter{tocdepth}{2}

\newcommand\fT{f_\text{T}}
\newcommand\IB{I_\text{B}}
\newcommand\IC{I_\text{C}}
\newcommand\ID{I_\text{D}}
\newcommand\IE{I_\text{E}}
\newcommand\IS{I_\text{S}}
\newcommand\RC{R_\text{C}}
\newcommand\RD{R_\text{D}}
\newcommand\RE{R_\text{E}}
\newcommand\UBE{U_\text{BE}}
\newcommand\UB{U_\text{B}}
\newcommand\UCE{U_\text{CE}}
\newcommand\UC{U_\text{C}}
\newcommand\UD{U_\text{D}}
\newcommand\UDS{U_\text{DS}}
\newcommand\UE{U_\text{E}}
\newcommand\UGS{U_\text{GS}}
\newcommand\UG{U_\text{G}}
\newcommand\Uin{U_\text{in}}
\newcommand\Uout{U_\text{out}}

\newcommand\UEH{U_\text{E H}}
\newcommand\UEL{U_\text{E L}}
\newcommand\UH{U_\text{H}}
\newcommand\UL{U_\text{L}}
\newcommand\UQH{U_\text{Q H}}
\newcommand\UQL{U_\text{Q L}}

\newcommand\tand{\textsc{and}}
\newcommand\tnor{\textsc{nor}}
\newcommand\tnot{\textsc{not}}
\newcommand\tor {\textsc{or}}
\newcommand\txor{\textsc{xor}}

\newcommand\mand{\mathsc{and}}
\newcommand\mnor{\mathsc{nor}}
\newcommand\mnot{\mathsc{not}}
\newcommand\mor {\mathsc{or}}
\newcommand\mxor{\mathsc{xor}}

\begin{document}

\maketitle

Der \LaTeX-Quelltext zu allen Protokollen in diesem Praktikum kann auf
\ref{it:mu} eingesehen werden. Die Quellen für dieses Protokoll können auf
\ref{it:github/alles} eingesehen werden. Die \LaTeX-Datei wird aus
\ref{it:github/template} generiert.

\begin{enumerate}
	\item
		\label{it:mu}
		\url{http://martin-ueding.de/de/university/physik313/}
	\item
		\label{it:github/alles}
		\url{https://github.com/martin-ueding/physik313-7/}
	\item
		\label{it:github/template}
		\url{https://github.com/martin-ueding/physik313-7/blob/master/Template.tex}
\end{enumerate}

\tableofcontents
\newpage

%%%%%%%%%%%%%%%%%%%%%%%%%%%%%%%%%%%%%%%%%%%%%%%%%%%%%%%%%%%%%%%%%%%%%%%%%%%%%%%
%                                 Einleitung                                  %
%%%%%%%%%%%%%%%%%%%%%%%%%%%%%%%%%%%%%%%%%%%%%%%%%%%%%%%%%%%%%%%%%%%%%%%%%%%%%%%

\FloatBarrier
\section{Einleitung}

%%%%%%%%%%%%%%%%%%%%%%%%%%%%%%%%%%%%%%%%%%%%%%%%%%%%%%%%%%%%%%%%%%%%%%%%%%%%%%%
%                                  Theorie                                    %
%%%%%%%%%%%%%%%%%%%%%%%%%%%%%%%%%%%%%%%%%%%%%%%%%%%%%%%%%%%%%%%%%%%%%%%%%%%%%%%

\FloatBarrier
\section{Theorie}

%%%%%%%%%%%%%%%%%%%%%%%%%%%%%%%%%%%%%%%%%%%%%%%%%%%%%%%%%%%%%%%%%%%%%%%%%%%%%%%
%                                  Aufgaben                                   %
%%%%%%%%%%%%%%%%%%%%%%%%%%%%%%%%%%%%%%%%%%%%%%%%%%%%%%%%%%%%%%%%%%%%%%%%%%%%%%%

\FloatBarrier
\section{Aufgaben}

\FloatBarrier
\subsection{Aufgabe K}

\begin{problem}
	Wechle logische Funktion wir[d] durch diese Schaltung [in
	Abbildung~\ref{fig:7-7}] realisiert? Welche Aufgabe habben die Dioden?
	Überprüfen Sie, ob auch hier noch die Ausgangspegel korrekt sind.
\end{problem}

Die die logische Funktion des Schaltung ist eine \tnor-Funktion.

Die Dioden
haben die Funktion die einkommenden Spannungen nur an das Gatter
weiterzuleiten. Die Eingangsspannung soll von beiden Eingängen auf $\UEH$
gezogen werden können, jedoch sollen die Eingänge voneinander entkoppelt sein,
sie sollen ja Eingänge sein.

\paragraph{Spannungslevel}

Wenn nur $\UL$ auf beiden Eingängen ist, liegt eine maximale Spannung von
\SI{.2}{\volt} an, da die Siliziumdioden schon \SI{.6}{\volt} verschlingen.
Dies reicht nicht mehr aus, um den Transistor zu schalten, am Ausgang werden
fast die vollen \SI{5}{\volt} anliegen.

Wenn einer der der Eingänge auf $\UL$ geschaltet ist, dann mindestens
\SI{2.4}{\volt}. Nach der Diode sind immer noch \SI{1.8}{\volt} da. Dies sollte
reichen, um den Transistor zu schalten. 

%%%%%%%%%%%%%%%%%%%%%%%%%%%%%%%%%%%%%%%%%%%%%%%%%%%%%%%%%%%%%%%%%%%%%%%%%%%%%%%
%                                  Literatur                                  %
%%%%%%%%%%%%%%%%%%%%%%%%%%%%%%%%%%%%%%%%%%%%%%%%%%%%%%%%%%%%%%%%%%%%%%%%%%%%%%%

\FloatBarrier
\IfFileExists{\bibliographyfile}{
	\bibliography{\bibliographyfile}
}{}

\end{document}

% vim: spell spelllang=de tw=79
