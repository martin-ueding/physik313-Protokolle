% Copyright © 2013 Martin Ueding <dev@martin-ueding.de>

\input{header.tex}

\ihead{physik313 – Versuch 2}
\ifoot{Ueding, Lemmer}

\hypersetup{
	pdftitle={Diodenkennlinien}
}

\subject{Praktikumsprotokoll}
\title{Diodenkennlinien}
\subtitle{physik313 – Versuch 2}
\author{
	Martin Ueding \footnote{\href{mailto:mu@martin-ueding.de}{mu@martin-ueding.de}}
	\and
	Lino Lemmer \footnote{\href{mailto:s6lilemm@uni-bonn.de}{s6lilemm@uni-bonn.de}}
}

\begin{document}

\maketitle

%\newpage
\tableofcontents
\newpage

%%%%%%%%%%%%%%%%%%%%%%%%%%%%%%%%%%%%%%%%%%%%%%%%%%%%%%%%%%%%%%%%%%%%%%%%%%%%%%%
%                                 Einleitung                                  %
%%%%%%%%%%%%%%%%%%%%%%%%%%%%%%%%%%%%%%%%%%%%%%%%%%%%%%%%%%%%%%%%%%%%%%%%%%%%%%%

\section{Einleitung}

%%%%%%%%%%%%%%%%%%%%%%%%%%%%%%%%%%%%%%%%%%%%%%%%%%%%%%%%%%%%%%%%%%%%%%%%%%%%%%%
%                                   Theorie                                   %
%%%%%%%%%%%%%%%%%%%%%%%%%%%%%%%%%%%%%%%%%%%%%%%%%%%%%%%%%%%%%%%%%%%%%%%%%%%%%%%


%%%%%%%%%%%%%%%%%%%%%%%%%%%%%%%%%%%%%%%%%%%%%%%%%%%%%%%%%%%%%%%%%%%%%%%%%%%%%%%
%                                  Aufgaben                                   %
%%%%%%%%%%%%%%%%%%%%%%%%%%%%%%%%%%%%%%%%%%%%%%%%%%%%%%%%%%%%%%%%%%%%%%%%%%%%%%%

\section{Aufgaben}

\subsection{Aufgabe A}

Bei Silizium gibt es 4 Zustände im Valenzband, keine in der Bandlücke und 4 im
Leitungsband.

\subsection{Aufgabe B}

Dotierung benutzt man, um freie Ladungsträger zu erhalten. In Halbleitern wäre
sonst kein Zustand im Leitungsband besetzt.

\subsection{Aufgabe F}

\subsubsection{Einfacher Widerstand}

Beim einfachen Widerstand $R$ gilt:
\[
	I = \frac 1R U
\]

Dies ist in Abbildung \ref{fig:F-Widerstand} skizziert.

\begin{figure}[h]
	\centering
	\caption{%
		Kennlinie des Ohm'schen Widerstands
	}
	\label{fig:F-Widerstand}
	\includegraphics{Zeichnungen/F-Widerstand.pdf}
\end{figure}

\subsubsection{Einfache Diode}

Für die Diode gilt die in Abbildung 2.2 der Aufgabenstellung gezeichnete
Kennlinie, ich habe sie in Abbildung \ref{fig:F-Diode} selbst gemalt.

\begin{figure}[h]
	\centering
	\caption{%
		Kennlinie der einfachen Diode
	}
	\label{fig:F-Diode}
	\includegraphics{Zeichnungen/F-Diode.pdf}
\end{figure}

\subsubsection{Diode und Widerstand seriell}

Hier gibt die Diode den Stromverlauf vor. Der Widerstand verschlingt jedoch
noch Spannung, wenn Strom fließt. Somit hat die Diode weniger Spannung, es
fließt weniger Strom. Darauf fällt weniger Spannung über dem Widerstand ab, die
Diode hat mehr spannung zur Verfügung. Wir haben versucht, die analytisch zu
lösen, sind jedoch nur anschaulich weitergekommen.

So haben wir uns überlegt, dass der Durchlass erst bei höherer Spannung
einsetzt und dann flacher einsteigt. Dies ist in Abbildung \ref{fig:F-seriell}
skizziert.

\begin{figure}[h]
	\centering
	\caption{%
		Kennlinie der Reihenschaltung
	}
	\label{fig:F-seriell}
	\includegraphics{Zeichnungen/F-seriell.pdf}
\end{figure}

\subsubsection{Diode und Widerstand parallel}

Hier liegt die gleiche Spannung an Diode und Widerstand an. Die Leitfähigkeiten
addieren sich:
\begin{align*}
	Y_\text{ext} &= Y_\text D + Y_R \\
	I_\text{ext} &= (Y_\text D + Y_R) U \\
	&= \del{\frac{f_\text D(U)}U + \frac 1R} U \\
	&= f_\text D (U) + \frac UR \\
	&= f_\text D (U) + f_R(U) \\
	&= (f_\text D + f_R)(U)
\end{align*}

Somit summieren sich beide Kennlinien auf, siehe Abbildung
\ref{fig:F-parallel}.

\begin{figure}[h]
	\centering
	\caption{%
		Kennlinie der Parallelschaltung
	}
	\label{fig:F-parallel}
	\includegraphics{Zeichnungen/F-parallel.pdf}
\end{figure}

\subsubsection{Ideale Stromquelle}

Die ideale Stromquelle hat einen Innenwiderstand $R_\text i$, die Quelle passt
die Spannung aber so an, dass der einstellte Strom $I_0$ fließt. Wenn extern
Spannung $U$ angelegt wird, hat die Stromquelle nur mehr oder weniger zu tun,
der Strom fließt trotzdem.
\[
	I = I_0
\]

Dies haben wir in Abbildung \ref{fig:F-Stromquelle} skizziert.

\begin{figure}[h]
	\centering
	\caption{%
		Kennlinie der idealen Stromquelle
	}
	\label{fig:F-Stromquelle}
	\includegraphics{Zeichnungen/F-Stromquelle.pdf}
\end{figure}

\subsubsection{Ideale Spannungsquelle}

Eine ideale Spannungsquelle mit eingestellter Spannung $U_0$ hat einen
Innenwiederstand $R_\text i$. Wenn man die Quelle kurzschließt, fließt der
Strom:
\[
	I = \frac{U_0}{R_\text i}
\]

Wir noch eine externe Spannung angelegt, fließt mehr oder weniger Strom durch den Innenwiderstand:
\[
	I = \frac{1}{R_\text i} (U_0 + U_\text{ext})
\]

Dies ist in Abbildung \ref{fig:F-Spannungsquelle} skizziert.

\begin{figure}[h]
	\centering
	\caption{%
		Kennlinie der idealen Spannungsquelle
	}
	\label{fig:F-Spannungsquelle}
	\includegraphics{Zeichnungen/F-Spannungsquelle.pdf}
\end{figure}

%%%%%%%%%%%%%%%%%%%%%%%%%%%%%%%%%%%%%%%%%%%%%%%%%%%%%%%%%%%%%%%%%%%%%%%%%%%%%%%
%                      Versuchsaufbau und -durchführung                      %
%%%%%%%%%%%%%%%%%%%%%%%%%%%%%%%%%%%%%%%%%%%%%%%%%%%%%%%%%%%%%%%%%%%%%%%%%%%%%%%

\section{Versuchsaufbau und -durchführung}


%%%%%%%%%%%%%%%%%%%%%%%%%%%%%%%%%%%%%%%%%%%%%%%%%%%%%%%%%%%%%%%%%%%%%%%%%%%%%%%
%                                 Auswertung                                  %
%%%%%%%%%%%%%%%%%%%%%%%%%%%%%%%%%%%%%%%%%%%%%%%%%%%%%%%%%%%%%%%%%%%%%%%%%%%%%%%

\section{Auswertung}

%%%%%%%%%%%%%%%%%%%%%%%%%%%%%%%%%%%%%%%%%%%%%%%%%%%%%%%%%%%%%%%%%%%%%%%%%%%%%%%
%                                  Ergebnis                                   %
%%%%%%%%%%%%%%%%%%%%%%%%%%%%%%%%%%%%%%%%%%%%%%%%%%%%%%%%%%%%%%%%%%%%%%%%%%%%%%%

\section{Ergebnis}

\IfFileExists{\bibliographyfile}{
	\bibliography{\bibliographyfile}
}{}

\end{document}

% vim: spell spelllang=de
