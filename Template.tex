% Copyright © 2013 Martin Ueding <dev@martin-ueding.de>

\input{header.tex}

\ihead{physik313 – Versuch 1}
\ifoot{Ueding, Lemmer}

\hypersetup{
	pdftitle={Ausbreitung von Signalen auf Leitungen}
}

\subject{Praktikumsprotokoll}
\title{Ausbreitung von Signalen auf Leitungen}
\subtitle{physik313 – Versuch 1}
\author{
	Martin Ueding \footnote{\href{mailto:mu@martin-ueding.de}{mu@martin-ueding.de}}
	\and
	Lino Lemmer \footnote{\href{mailto:s6lilemm@uni-bonn.de}{s6lilemm@uni-bonn.de}}
}

\begin{document}

\maketitle

%\newpage
\tableofcontents
\newpage

%%%%%%%%%%%%%%%%%%%%%%%%%%%%%%%%%%%%%%%%%%%%%%%%%%%%%%%%%%%%%%%%%%%%%%%%%%%%%%%
%                                 Einleitung                                  %
%%%%%%%%%%%%%%%%%%%%%%%%%%%%%%%%%%%%%%%%%%%%%%%%%%%%%%%%%%%%%%%%%%%%%%%%%%%%%%%

\section{Einleitung}

In diesem Versuch experimentieren wir mit der Ausbreitung von Pulsen auf
Koaxialkabeln. Dabei betrachten wir Verzögerung, Reflexion und
Impulsdeformation.

%%%%%%%%%%%%%%%%%%%%%%%%%%%%%%%%%%%%%%%%%%%%%%%%%%%%%%%%%%%%%%%%%%%%%%%%%%%%%%%
%                                   Theorie                                   %
%%%%%%%%%%%%%%%%%%%%%%%%%%%%%%%%%%%%%%%%%%%%%%%%%%%%%%%%%%%%%%%%%%%%%%%%%%%%%%%

\section{Theorie}

\subsection{Phasengeschwindigkeit}

Formel (1.6) aus der Anleitung:
\begin{equation}
	\label{eq:1.6}
	\Upsilon^2 = Z' Y'
\end{equation}

Formel (1.7) aus der Anleitung:
\begin{equation}
	\label{eq:1.7}
	U_\text h(l) + U_\text r(l)
	= \del{U_\text h(0) \eup^{-\Upsilon x} + U_\text r(0) \eup^{\Upsilon x}} \eup^{\iup \omega t}
\end{equation}

Formel (1.13) aus der Anleitung:
\begin{equation}
	\label{eq:1.13}
	v_\text{ph} = c_0 \frac{1}{\sqrt{\varepsilon_0 \mu_0}}
\end{equation}

Formel (1.14) aus der Anleitung:
\begin{equation}
	\label{eq:1.14}
	Z = \frac{U_\text h (x)}{I_\text h (x)}
\end{equation}

Formel (1.15) aus der Anleitung:
\begin{equation}
	\label{eq:1.15}
	Z = \sqrt{\frac{\mu_r \mu_0}{\varepsilon_r \varepsilon_0}} \frac{\ln(R_\text a/R_\text i)}{2\piup}
\end{equation}

Formel (1.17) aus der Anleitung:
\begin{equation}
	\label{eq:1.17}
	R_\text A
	= \frac{U_\text h (l) + U_\text r (l)}{I_\text h (l) + I_\text r (l)}
	= Z \frac{U_{\text h l} + U_{\text r l}}{U_{\text h l} + I_{\text r l}}
	= Z \frac{1+r}{1-r}
\end{equation}

%%%%%%%%%%%%%%%%%%%%%%%%%%%%%%%%%%%%%%%%%%%%%%%%%%%%%%%%%%%%%%%%%%%%%%%%%%%%%%%
%                                  Aufgaben                                   %
%%%%%%%%%%%%%%%%%%%%%%%%%%%%%%%%%%%%%%%%%%%%%%%%%%%%%%%%%%%%%%%%%%%%%%%%%%%%%%%

\section{Aufgaben}

\subsection{Aufgabe A}

\begin{quote}
	Was muss man tun, um große Verzögerungszeiten zu erreichen?
\end{quote}

Eine große Verzögerungszeit bedeutet eine kleine Phasengeschwindigkeit. Diese
ist in Formel \eqref{eq:1.13} gegeben. Es kann entweder $\mu_r$ oder
$\varepsilon_r$ groß gemacht werden.

Außerdem kann das Kabel verlängert werden, dann braucht das Signal auch länger.

\subsection{Aufgabe B}

\begin{quote}
	Welche Konsequenz für den Wellenwiderstand haben die verschiedenen
	Möglichkeiten, die Verzögerungszeiten zu verändern?
\end{quote}

\paragraph{Permittivität erhöhen}

Durch Einfügen eines Dielektrikums kann die Permittivität erhöht werden,
dadurch werden die Kabel schwerer und teurer. Nach \eqref{eq:1.15} sinkt die
Impedanz $Z$ des Kabels mit größer werdender Permittivität.

\paragraph{Permeabilität erhöhen}

Durch Erhöhen der Permeabilität wird ebenfalls mehr Material gebraucht. Die
Impedanz steigt nach \eqref{eq:1.15} jetzt allerdings an.

Durch eine geschickte Kombination von Permittivität und Permeabilität kann die
Impedanz gleich gehalten werden und die Phasengeschwindigkeit verringert
werden.

\paragraph{Kabel verlängern}

Ein längeres Kabel bedeutet auch erhöhte normale Ohm'sche Verluste. Es braucht
auch mehr Material und somit mehr Gewicht und Kosten.

\subsection{Aufgabe C}

\begin{quote}
	Sei ein Kabel abgeschlossen mit $R_A = Z$. Wie hängt der Eingangswiderstand
	$R_\text{in}$ des Kabels von seiner Länge ab?
\end{quote}

Der Eingangswiderstand ist der Widerstand, den der Signalgenerator am Kabel
erfährt, also das Verhältnis von $U_\text h(0)$ und $I_\text h(0)$. Der
Wellenwiderstand $Z$ ist ebenfalls dieses Verhältnis, wenn es keine rücklaufende Welle gibt, siehe \eqref{eq:1.14}.

Bei einer Länge $l$ hat es die spezifische Impedanz von $Z' = Z/l$. Die
Impedanz bei einer Länge $l$ ist somit $Z(l) = Z' l$. Für die
Längenabhängigkeit erhalten wir also:
\[
	R_\text{in}(l) = Z' l = Z
\]

\subsection{Aufgabe D}

\begin{quote}
	Wie groß ist die Eingangsimpedanz $Z$ eines verlustfreien Kabels mit
	Wellenwiderstand $Z = \SI{50}\ohm$ und einem offenen Leitungsende für eine
	Sinusspannung der Frequenz $\omega$ und der Wellenlänge auf dem Kabel
	$\lambda$ in Abhängigkeit der Länge $l$ des Kabels ($0 < l < \lambda$)?
\end{quote}

Nach \eqref{eq:1.14} ist der Wellenwiderstand:
\begin{equation}
	\label{eq:D1}
	Z := \frac{U_\text h(x)}{I_\text h(x)}
\end{equation}

Wäre nicht die rücklaufende Welle, wäre hier einfach $Z_\text{in} = Z$. In dieser Aufgabe habe ich zwar schon $Z$, allerdings reicht das nicht aus. Ich brauche:
\begin{equation}
	\label{eq:D2}
	Z_\text{in}
	= \frac{U_\text h (l) + U_\text r (l)}{I_\text h (l) + I_\text r (l)}
\end{equation}

In der Anleitung steht für die offene Leitung
\[
	I_\text r(l) + I_\text h(l) = 0
\]
sowie
\[
	U_\text r(l) - U_\text h(l) = 0.
\]

Mit \eqref{eq:1.7} können wir herleiten:
\begin{align*}
	U_\text r(l) &= U_\text h(l) \\
	U_\text h(0) \eup^{-\Upsilon x} \eup^{\iup \omega t} &= U_\text r(0) \eup^{\Upsilon x}) \eup^{\iup \omega t} \\
	U_\text h(0) \eup^{-\Upsilon x} &= U_\text r(0) \eup^{\Upsilon x}) \\
	U_\text h(0) \eup^{-2 \Upsilon x} &= U_\text r(0) \\
\end{align*}

Für den Strom geht dies analog und führt auf die gleiche Phase zwischen ein-
und ausgehender Welle. Damit können wir \eqref{eq:D2} umschreiben und erhalten:
\[
	Z_\text{in}
	= \frac{U_\text h (l)}{I_\text h (l)} (1 + \eup^{-2 \Upsilon x}
\]

In diese Relation setzen wir wiederum \eqref{eq:D1} ein und vereinfachen so zu:
\[
	Z_\text{in}
	= Z (1 + \eup^{-2 \Upsilon x}
\]

Das Kabel war als verlustfrei vorgegeben, es gibt also keinen Strom zwischen Leiter und Masse. Damit ist $Y = Y' = 0$ Aus \eqref{eq:1.6} folgt damit, dass $\Upsilon = 0$ ist. Dadurch ist der Phasenfaktor in der obigen Formel einfach eins.

Wir erhalten das Ergebnis:
\[
	Z_\text{in} = 2 Z
\]

Dies ist unabhängig von der Länge und der Wellenlänge sowie der Frequenz.

%%%%%%%%%%%%%%%%%%%%%%%%%%%%%%%%%%%%%%%%%%%%%%%%%%%%%%%%%%%%%%%%%%%%%%%%%%%%%%%
%                      Versuchsaufbau und -durchführung                      %
%%%%%%%%%%%%%%%%%%%%%%%%%%%%%%%%%%%%%%%%%%%%%%%%%%%%%%%%%%%%%%%%%%%%%%%%%%%%%%%

\section{Versuchsaufbau und -durchführung}

\subsection{Seriennummern}

Die Seriennummern unserer Geräte haben wir in Tabelle \ref{tb:seriennummern}
aufgelistet.

\begin{table}[hb]
	\center
	\caption{Seriennummern unserer Geräte}
	\label{tb:seriennummern}
	\begin{tabular}{ll}
		Gerät & Seriennummer \\
		\hline
		Oszillograph & \\
		Funktionsgenerator & \\
		Kabel HH 2500 & \\
		Kabelkasten (RG-58 C/U) & \\
		Differenzierglied & \\
		Differenzierglied mit \SI{2.2}{\kilo\ohm} Anpassung & \\
		Anpasswiderstand: \SI{2.45}{\kilo\ohm} & \\
		Abschlusswiderstand: \SI{2.5}{\kilo\ohm} und \SI{50}{\ohm} & \\
		Einstellbarer Abschlusswiderstand: \SIrange{0}{10}{\kilo\ohm} &
	\end{tabular}
\end{table}

\subsection{Versuchsaufgabe 1: Differenzierglied}

\subsection{Versuchsaufgabe 2: Impulse auf Kabeln}

\subsection{Versuchsaufgabe 3: Leitungsabschluss, Verzögerungszeit}

\subsection{Versuchsaufgabe 4: Klippkabel, Dämpfung}

\subsection{Versuchsaufgabe 5: \SI{50}{\ohm} Kabel RG-58 C/U}

\subsubsection{Teil a}

\begin{quote}
	Warum macht sich eine kleine Bandbreite besonders bei der Übertragung von
	Rechtecksignalen bemerkbar?
\end{quote}

Entwickelt man ein Rechtecksignal in eine Fourierreihe, so fallen die
Amplituden der hohen Frequenzen nur langsam ab, weil die Funktion nicht
besonders glatt ist. Dies bedeutet, dass sehr viele hohe Frequenzen mit
einbezogen werden müssen, damit die Summe wie ein Rechteck aussieht. Bei einer
kleinen Bandbreite ist allerdings genau dies das Problem: Hohe Frequenzen
werden abgeschnitten, da sie besonders gedämpft werden. So wird das Rechteck an
den Ecken etwas abgerundet.

%%%%%%%%%%%%%%%%%%%%%%%%%%%%%%%%%%%%%%%%%%%%%%%%%%%%%%%%%%%%%%%%%%%%%%%%%%%%%%%
%                                 Auswertung                                  %
%%%%%%%%%%%%%%%%%%%%%%%%%%%%%%%%%%%%%%%%%%%%%%%%%%%%%%%%%%%%%%%%%%%%%%%%%%%%%%%

\section{Auswertung}

\subsection{Versuchsaufgabe 3: Leitungsabschluss, Verzögerungszeit}

\subsubsection{Teil d}



%%%%%%%%%%%%%%%%%%%%%%%%%%%%%%%%%%%%%%%%%%%%%%%%%%%%%%%%%%%%%%%%%%%%%%%%%%%%%%%
%                                  Ergebnis                                   %
%%%%%%%%%%%%%%%%%%%%%%%%%%%%%%%%%%%%%%%%%%%%%%%%%%%%%%%%%%%%%%%%%%%%%%%%%%%%%%%

\section{Ergebnis}

\IfFileExists{\bibliographyfile}{
	\bibliography{\bibliographyfile}
}{}

\end{document}

% vim: spell spelllang=de
