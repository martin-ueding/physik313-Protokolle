% Copyright © 2013 Martin Ueding <dev@martin-ueding.de>

\input{header.tex}

\hypersetup{
	pdftitle={Einführung und Vorversuch}
}

\subject{Praktikumsprotokoll}
\title{Einführung und Vorversuch}
\subtitle{physik313 – Versuch 0}
\author{
	Martin Ueding \footnote{\href{mailto:mu@martin-ueding.de}{mu@martin-ueding.de}}
	\and
	Lino Lemmer \footnote{\href{mailto:s6lilemm@uni-bonn.de}{s6lilemm@uni-bonn.de}}
}

\begin{document}

\maketitle

%\newpage
%\tableofcontents
%\newpage

\section{Einleitung}

\section{Theorie}

\section{Aufgaben}

\subsection{Aufgabe A}

\begin{align*}
	U_\text{SS}&=2U_0\\
	U_\text{S}&=U_0
\end{align*}

Für die effektive Spannung:
\begin{align*}
	U_\text{eff}^2
	&= U_0^2 \frac{\omega}{2\piup/\omega} \int_0^{2\piup} \dif t \, \sin^2(\omega t) \\
	&= U_0^2 \frac{\omega}{2\piup/\omega} \int_0^{2\piup} \dif t \, \half \del{1- \cos(2\omega t)} \\
	&= \half U_0^2 \frac{\omega}{2\piup} \eval{\del{1- \frac{1}{2\omega} \sin(2\omega t)}}_0^{2\piup/\omega}  \\
	&= \half U_0^2
\end{align*}

Somit folgt:
\[
	U_\text{eff} = \frac{U_0}{\sqrt{2}}
\]

\subsection{Aufgabe B}

Bei der Rechteckspannung müssen wir genauso vorgehen und über eine ganze Periode mitteln. Die Periode sei $T$ und die Maximalspannung $U_0$.
\begin{align*}
	U_\text{eff}^2
	&= \frac 1T U_0^2 \int_0^T \dif T \, \Theta^2\del{t - \frac T2} \\
	&= \frac 1T U_0^2 \int_{T/2}^T \dif T \\
	&= \frac 1T U_0^2 \frac T/2 \\
	&= \frac{U_0^2}2
\end{align*}

Somit folgt:
\[
	U_\text{eff} = \frac{U_0}{\sqrt{2}}
\]

Bei einer Eingangsspannung von $U_0 = \SI{10}\volt$ ist der Effektivwert \SI{<< B_U_eff >>}\volt.

\subsection{Aufgabe C}

Aus
\begin{align}
	U_1&=U_0\frac{R_1}{R_1+R_i}\notag
	\intertext{folgt}
	R_i&=\frac{\del{U_0-U_1}R_1}{U_1}\notag\\
	&=\frac{U_0-U_1}{I_1}\label{eq:R_i}
	\intertext{und aus}
	U_2&=U_0\frac{R_2}{R_2+R_i}\notag
	\intertext{folgt}
	U_0&=\frac{\del{R_2+R_i}U_2}{R_2}\notag\\
	&=\del{R_2+R_i}I_2\label{eq:U_0}
	\intertext{Setzt man nun \eqref{eq:U_0} ind \eqref{eq:R_i} ein erhält man}
	R_i&=\frac{\del{R_2+R_i}I_2}{I_1}-\frac{U_1}{I_1}\notag\\
	\iff\qquad \del{\frac{I_1-I_2}{I_1}}R_i&=\frac{U_2-U_1}{I_1}\notag\\
	\iff\qquad R_i&=\frac{U_2-U_1}{I_1-U_2}\notag
\end{align}
Dies ist die gesuchte Gleichung.

\section{Versuchsaufbau und -durchführung}

\section{Messung}

\section{Auswertung}

\subsection{Aufgabe b}

Aus der
Anleitung\footnote{\url{http://www.hameg.com/manuals.0.html?&no_cache=1&L=1&tx_hmdownloads_pi1[mode]=download&tx_hmdownloads_pi1[uid]=987}}
haben wir den Frequenzbereich entnommen, dieser wird mit $B = \SI{<< B
>>}\hertz$ angegeben. Daraus folgt nach der Formal (0.4) aus der Anleitung für
die Anstiegszeit: $\Deltaup t_\text{Oszi} = \SI{<< Delta_t >>}\second$.

\section{Ergebnis}

%\IfFileExists{\bibliographyfile}{
%	\bibliography{\bibliographyfile}
%}{}

\end{document}

% vim: spell spelllang=de
